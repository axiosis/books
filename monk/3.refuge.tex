\section{Прихисток}

Якщо ви дочитали до третього посту --- це означає, що рекламна
компанія працює, і вас справді зацікавив Просвітлений Майндстрім.
Потрібно сказати, що буддійський світогляд це не
місіонерська релігія, нам все одно чи хочете ви бути буддистом
чи ні, рано чи пізно і так усі будуть буддистами, ми нікуди
не поспішаємо. Припустимо ви вирішили долучитися до ордену Благородних,
що потрібно для цього робити?

Існують кілька ступенів входу в рух, для простоти
будемо говорити, що їх три: 1) для тупих, 2) середнячків та 3)
розвинених фізично та розумово. Взагалі у буддизмі багато
дискримінації та провокацій, так що звикайте. Умовно,
чим людина розвиненіша, тим більше значення надається Вчителю.
Значення Вчителя переоцінити складно, тому буддизм
класифікується як ламаїзм, гуруїзм, паханат або дідівщина,
називайте як хочете, деякі лами жартують і називають це
Ламським Рабством. Але оскільки я не певен, що серед вас
є навіть середнячки, то матимемо на увазі, що всі читачі
цього блогу тупі, чи інакше кажучи, перебувають на першому рівні.

Як увійти в рух Благородних та ознайомитися з Просвітленим
Майндстрімом? Потрібно прийняти притулок у Будді, Дхармі та Санзі!
Перші два слова ви знаєте, але що таке Санха? Якщо
Дхарма --- це благе мислення вашого майндстріму, то Санха --- це
благе мислення інших майндстрімів, ваших друзів, що йдуть по
шляху Благородних. Що означає Прийняття Прихистку?
Прихисток --- це усвідомлений вибір і намір приєднатися
до руху Благородних, тобто бажання особистого повного
і безумовного звільнення і такого ж абсолютного та безповоротного
визволення для інших. Не досягнення миттєвого задоволення,
після якого часто (а може і завжди) буває відхідняк,
а повної Нірвани --- абсолютного і кінцевого максимального
щастя (без жодних мінімаксів).

Чому це називається Прихистком? Тому що шлях цей, як і
ваше життя, небезпечний і складний. На дорозі може відбуватися що
завгодно, ви будь-якої секунди можете померти від того, що розірвався
тромб в мозку або божевільний мотоцикліст на дорозі скалічив ваше життя.
А від небезпек добре мати парасольку чи іншуранс. Така страховка
і є Прихисток. Воно має видаватися лише сертифікованими
філіями, і передбачає усвідомлене підписання договору.
У контракті ви зобов'язуєтесь свято шанувати та поважати об'єкти
притулку або як їх називають Три Корені: 1) Будду Шак'ямуні ---
першого Вчителя (у тимчасових межах цього Великого Вибуху), який
поставив питання про існування Просвітленого Майндстріму
і успішно підтвердив цю гіпотезу; 2) Дхарму --- своє особисте
благе мислення, яке є дуже сильною зброєю в
боротьбі за абсолютну свободу; 3) Санху --- благе мислення
інших істот, які прийняли такий самий контракт. Зі свого боку
Рух Благородних, особисто під заступництвом Будди Шак'ямуні,
зобов'язується допомагати вам на шляху досягнення вашої шляхетної
Цілі --- Просвітленого Майндстріму. Стережіться підробок,
на сертифікаті мають бути всі знаки Будди Шак'ямуні: Чотири
Благородні істини, Вчитель, підходящий Час, Місце, ваша
підпис кров'ю та підпис кров'ю Вчителя.

Автопритулок. Зрозуміло, що через цей стрім ви не отримаєте
притулок, але деяке уявлення про нього ви вже отримали,
і якщо дуже сильно захотіти, то можна прийняти його автономно
(Автопритулок) і вже потім зустрітися з Вчителем, який
має сертифікат.

Сертифікати. Є список філій, які видають такі сертифікати:
Бон (небуддійський рух, або китайська підробка Буддизму,
але працююча), Норбупа (сертифікати, що видаються Намкаєм Норбу
Рінпоче), та офіційні школи тибетського Буддизму (ведуть
початок від Будди Шак'ямуні) --- Гелугпи, Джонанги, Кагьюпи,
Сак'япи, Ньінгмапи. Божевільний Монах має сертифікат Ньінгмапи
--- або в перекладі російською --- Старих Пердунів (кличка
з негативним забарвленням, яким обзивали ньінгмапінців
інші школи, і яка згодом стала іменем школи, адже найкращий захист --- це відкритість).

Ритуал. Ясно, що Автоприхисток не має жодного ритуалу,
ви вирішуєте для себе самі і потім вже, в цьому чи в наступному
життях зустрічається з Вчителем. Можна далі ускладнювати ритуал,
знаходити найближчого ламу чи наставника, які особисто дадуть
вам Притулок, або отримати аудієнцію у високого лами та
розкачати ритуал за повною програмою з обрізанням (волосся),
рисом та офіційним святкуванням. Взагалі у буддизмі немає
нічого обов'язкового, немає догм, для будь-якого правила є
виняток. Це трохи ускладнює просування, але також
і відкриває двері до лайфхаків. Тож треба знати закони
та успішно ними користуватися для своєї особистої вигоди на шляху
Просвітленого Манйдстріма.
