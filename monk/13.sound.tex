\section{Реверберації багатовимірного звуку}

Сьогодні згідно з календарем Тибету --- сприятливий день,
отже, є шанс почитати проповідь від Божевільного Монаха.
Згідно з тантрою Дра Талг'юр з розділу Менгаде Дзогчен всі
світи і всі всесвіти, всі Великі Вибухи, в які можуть
(а можуть і не) приходити Будди з'являються з вібрації
багатовимірної мембрани. Тобто ми з вами --- справжні
коливання Єдиного Квантового Вакууму, продукти
вібрацій багатомірного звуку. Тим кому складно уявити
божество використовуючи небесне полотно як канвас, а білі
бігаючі точки як пензель (доля візуалів), можуть увійти
рух через вібрації звуку (доля аудіалів). Тому
розчехляйте ваші ONKYO, DENON, сьогодні буде дхарма-дискотека.

Взагалі, мантри так просто не можна передавати. У традиції
дуже важливо, щоб мантру вам наспівав чи просто формально
продиктував за літерами (з довільною швидкістю) Вчитель,
інакше така передача вважатиметься некошерною. Не надиктував
по телефону (Норбупа стайл), а особисто, так щоб звук
торкнувся ваших вушних мембран і виходив з голосових
зв'язок Вчителя. Але оскільки Монах Божевільний, а в інтернеті
немає нічого кошерного, а в буддизмі немає догм, то дещо я
вам все ж таки розповім. Це буде справжнісінький Дхарма Попсовий
альбом попередніх практик. У ньому будуть зібрані основні
практики у вигляді мантр, які не потребують передачі. Формально
ви можете слухати, але практикувати (самим читати) ненавмисне.
Як і у випадку Автопритулку, валіть все на Божевільного
Монаха (якщо дуже хочеться самим потренуватися)!

Кожен буддист розпочинає свій день з Молитви до Восьми
Тих, Хто приносить Удачу. Це пісня Міпама Рінпоче, втілення
Манджушрі, який не залишив після себе реінкарнацій (повна
реалізація, остаточне здобуття плоду еволюції мислення)
\footnote {\url{https://tonpa.guru/nendro/1. Auspicious.mp3}}
Тибетський текст і переклад тут Виконує на записі керівник Шрі
Сінха центрів по всьому світу, представник Дзогчен монастирів
та Долини Дзогчен на цій землі --- Дзогчен Кенпо Чога Рінпоче.

Відразу після цієї молитви ми згадуємо притулок, який уже давали.
\footnote{\url{https://tonpa.guru/nendro/2. Take Refuge.mp3}}.
Цю мелодію я вперше почув по Радіо Lounge FM (United Peace Voices),
її можна грати в кафе і добре шифруватися (ніхто і не запідозрить, що ви сектант).
Трансрипція була дана в перших постах, а сенс вірша полягає в
прийняття притулку в Будді, Дхармі та Санзі.

Формально перші два записи ---- це вірші Тибету, а не мантри
на божественному санскриті. Але сила їх настільки величезна, що
вони давно вже перекочували до розряду мантр, віршів-мантр. Після
цих записів ми наводимо своє мислення на заспокоєння Стоскладової
Мантри Ваджрасаттви. Ваджрасаттва --- це принц, що увійшов до Мандали.
Переможців зі сходу, який мав дуже багато неслухняних
учнів і він дав обітницю, що його мантра допомагатиме всім у кого
проблеми із чистотою мислення. Це свого роду метапокаяння.
Є звичайно і більш жорсткі види очисних молитов, які можуть
очистити навіть порушені Самаї, такі як Нарак Кончак, але ви поки що
не наробили великих ділов, ви діти, і молитви вам потрібні теж початкового
рівня. Зустрічайте Втілення Ваджрасаттви в звуці від
Сак'япінських Монахинь\footnote{\url{https://tonpa.guru/nendro/3. Vajrasattva.mp3}}.
Ваджрасаттва в перекладі на нашу мову означає Алмазний Розум, по-тибетськи
Дордже Семпа. Мантра Ваджрасаттви:
\\
\\
ОМ БЕНЗА САТО САМА МАНУ ПАЛА БЕНЗА САТО ТЕНОПА ЧИТА ДРІТО
МЕБХАВА СУТО КАЙО МЕБХАВА СУПО КАЙО МЕБХАВА АНУ РАКТО МЕБХАВА
САРВА СІДДХІ МЕМ ПРАЯЦЯ САРВА КАРМА СУЦА МЕ ЧИТАМ ШРІ ЯМ КУРУ
ХУМ ХАХА ХАХА ХО БХАГАВАН САРВА ТАТХАГАТА БЕНЗА МА МЕ МУНЦЯ
БЕНЗІ БХАВА МАХА САМА САТО О.
\\
\\
Мантра Ваджрасаттви зустрічається як у багатьох одкровеннях (терму) так і в тантрах.

Підношення Мандали або Всесвіту. Ця вірш-мантра давалася
Будда Шак'ямуні і записана в Гухьясамаджа Корінний
Тантрі\footnote{\url{https://tonpa.guru/nendro/5. Mandala.mp3}}.
Виконують сак'япінські ченці:
\\
\\
ОМ БЕНЗУ БХУМІ А ХУМ СІ ЙОНГ СУ ДАКПА ВАНГ ЧЕН СЕРЗІ
САСІ ОМ БЕНЗУ РЕКЕ А ХУМ ЧІ ЧАК РІ КХОР ЮК ГІ КОРВЕ
У СУ ХУМ НРІ Г'ЯЛПО РІ РАБ ШАР ЛУ ПАПКО ЛО ДЗАМБУ
ЛІНГ НУБ БАЛАНГ ЧО ЧАНГ ДРА МІ НЬЄН ЛУ ДАНГ ЛУ ПАК
НДА ЯБ ДАНГ НДА ЯБ ШЕН ЙО ДЕН ДАНГ ЛАМ ЧОК ДРО ДРА
МІ НЬЄН ДАНГ ДРА МІ НЬЄН ГІ ТА РИНПОЧЕ РИВО ПАК САМ
ГІ ШИНГ ДО ДЖО І БА МА МО ПЕЛО ТОК КХОРЛО РІНПОЧЕ
НОРБУ РІНПОЧЕ ЦУНМО РІНПОЧЕ ЛОНПО РІНПОЧЕ ЛАНГПО
РІНПОЧЕ ТАЧОК РІНПОШЕ МАКПОН РІНПОЧЕ ТЕР ЧЕНПО
БУМПА ГЕКПА МА ТРЕНГВА МА ЛУ МА ГАР МА МІТОК МА
ДУКПО МА НАНГСЕЛ МА ДРІЧАП МА НЬІНГ ДАВА РІНПОЧЕ
ДУК ЧОК ЛЕ НАМ ПАР Г'ЯЛВІ Г'ЯЛЦЕН ЛА ДАНГ МІЄ ПАЛДЖОР
ПУНСУМ ЦОКПА МА ЦАНГВА МЕПА
\\
\\
Детальні пояснення шукайте в Інтернеті, наприклад: "lamayeshe mandala offering".

Після того як ми піднесли всесвіт ми виконуємо
Чод або Накопичення Енергії Кусалі. Відсікання
прихильності до власного Я. Є ціла школа
буддизму Чодпа, яка веде свій початок від Мачіг Лабдрон,
жінки, яка стала Буддою тільки завдяки одній
цій практиці. Внесектарний Чод\footnote{\url{https://tonpa.guru/nendro/6. Chod.mp3}}.
Транскрипція:
\\
\\
ОМ МАЧИГ МА ЛА СОЛВА ДЕБ КАРПО ОМ ГІ ДЖИНКИ ЛОБ
МАРПО А ГІ ДЖИНКИ ЛОБ НОНГПО ХУМ ГІ ДЖИНКИ ЛОБ
КУ СУМ ТУК КИ ДЖИН ЧЕН ПОП МА ЮМ ЧЕН ГО ПАНГ ТОБ ПАР ШОК
\\
\\
Тільки після того, як ви очистилися в SPA у самого Ваджрасаттви,
виконали відсікання прив'язанності до власного Я у Мачінг Лабдрон
і піднесли Весь Всесвіт у вигляді Мандали, вам нарешті дозволяється
звернутись до Засновника Корпорації, який створив Службу
Безпеки та весь тибетський Буддизм разом із тибетським
алфавітов, Великим Тантричним Майстром Падмасамбхавою.
Але є нюанс, перш ніж почати читати його мантру
потрібно спочатку обов'язково прочитати семирядковий мантро-вірш,
закликання Падмасабхави
\footnote{\url{https://tonpa.guru/nendro/7. Guru Rinpoche.mp3}}.
не менше 21 (108 для затятих) разів.
\\
\\
ОРДЖЕН ЮЛДЖІ НУБ ДЖАНГ ЦАМ ПЕМА ГЕСАР ДОНГ ПОЛА
ЯМЦЕН ЧОКІ НЕДРУБ НЕ ПЕМА ДЖУНЕ ШЕСУ ДРАГ КХОРДУ
КХАНДРО МАНПО КЕР КЕКІ ДЖЕСУ ДАГ ДРУБ ДЖІ ДЖИНДЖІ
ЛАБЧИР ШЕКСУ СОЛ ГУРУ ПАДМА СІДДХІ ХУМ.
\\
\\
Не багато серед релігійних фанатиків знають, що семирядкова
молитва --- це терма Гуру Чованга (1212-1270). Після закликання
Гуру Рінпоче дозволяється начитувати мантру, яку ви повинні
почути особисто від Учителя (без винятків):
\\
\\
ОМ А ХУМ БЕНЗУ ГУРУ ПЕМА СІДДХІ ХУМ.
\\
\\
Для зручності всі мантри зібрані тут\footnote{\url{https://tonpa.guru/nendro/}} у вигляді альбому
разом із бонусами: молитва Манджушрі\footnote{\url{https://tonpa.guru/nendro/4. Manjushri.mp3}},
Молитва Устремління в Девачен\footnote{\url{https://tonpa.guru/nendro/A. Dechen Monlam.mp3}},
Пісня трапези Джігме Лінгпи\footnote{\url{https://tonpa.guru/nendro/9. Leymon Tendrel.mp3}},
Молитва Дзамбалі та тантра Самантабхадри\footnote{\url{https://tonpa.guru/nendro/8. Kunzan Monlam.mp3}} для
визволення шляхом слухання.
