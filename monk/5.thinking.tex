\section{Мислення поточної миттєвості}

Що ж таке Майндстрім чи наше Мислення? Мислення --- це центральний предмет
вивчення в  буддійській науці. Ми, живі мислячі істоти --- є мисленнями. Немає нічого,
крім мислення. Все, про що ви думаєте, все, що ви бачите, відчуваєте, сприймаєте,
всі ваші емоції, явища, концепції, теорії, припущення, включаючи вас самих, --- це мислення.
Якщо ви будете щось довго і уважно розглядати, ви виявите своє мислення.

Істота. Зазвичай мислення називають істотою. Якщо, щось і існує по-справжньому, то це мислення.
Якщо ви зараз дивитесь, наприклад, навколо себе, то немає жодного іншого мислення,
крім того, що ви бачите і відчуваєте, немає якогось великого розуму, пам'яті.
Пам'ять, думки, звички, емоції, знання, свідомість, усвідомлення, підсвідомість, почуття,
любов, щастя та страждання --- все це знаходиться у вашому мисленні тут і зараз, завжди та скрізь.
Все ваше мислення знаходиться прямо в цій миті і називається Мисленням Миті.

Існування. Мислення по-тибетськи називається Сем, може виступати в ролі дієслова
та іменника. Зазвичай, коли ми говоримо про мислення, ми говоримо про найтоншу
реальну субстанцію, яка формує решту нашого всесвіту, тому з цього погляду буддизм
можна називати ідеалістичним матеріалізмом. Виступаючи основним об'єктом дослідження,
звівши все існуюче і неіснуюче до Мислення Миті буддизм можна сприймати як ідеалізм.
Якщо ми в буддизмі говоримо, що щось існує, воно не може змінювати свою якість існування.

Відкритість. Якщо атом --- це хвиля, яка поширюється на весь видимий всесвіт,
то безглуздо було б думати, що ваше мислення обмежене вашим черепом. Відкритість і відчиненість
вашого мислення має радіальну природу і як Сонце світить для всіх істот у всіх напрямках.
Мислення не обмежене простором, формою та образи виникають у самому мисленні і є його частиною,
якщо ми намагатимемося перераховувати характеристики мислення, то ми виявимо, що всі вони
відповідають характеристикам простору. Мислення --- це Простір, простів вашого сприйнятта та реалізації.

Свобода вибору. У кожну конкретну мить, ваше мислення вільно вибирає яким шляхом йти,
бути на стороні Джидаїв або Сітхів. Навіть якщо ви знаходитесь в нижньому дні
нижнього пекла, ви можете мислити позитивно, що саме з нижнього пекла почав свій рух
на шляху Будда Шак'ямуні. Приклади коли боги вибирають шлях Сітхів наводити не потрібно,
достатньо спробувати припаркуватися в годину пік. Кожна істота в Єдиному Квантовому
Вакуумі має зерно Просвітленого Майндстріму. Остаточно на жодній істоті не можна поставити хрест,
адже вона переродиться в наступних життях і там отримає свій другий шанс.

Неприборканість. На які б тренінги з зупинки мислення ви не ходили,
яких майстрів дзен медитації ви не обирали б --- зупинити і досягти повного
спокою мислення неможливо в жодному з чотирьох (шести) бардо. Близько 80 видів медитації
Шаматха призначені для тренування вашого мислення: з опорою, без опори, опора на порожнечу,
на дихання, неконцептуальна медитація --- все це дано вам, щоб зрозуміти одне:
мислення як Інтерстелар Двигун --- зупинити його неможливо. Тому медитуйте спокійно
і не чекайте знаків, нічого не повинно статися, просто сидіть і рахуйте дихання.
Принаймні перші п'ять років. Майндфулнес Медитація, ага, Calm, Headspace.

Карма. Головна характеристика мислення, що є показником, чи рівнем мислення
називається кармою. Карма --- це інтеграл з Мислення. Поняття карми було ще
до Будди Шак'ямуні. Карма --- це сила мислення, звички мислення, патерни мислення.
Чим більше ви повторюєте якесь мислення, тим більшу ви накопичуєте карму.
Мислення повторене три рази --- це емоція, якщо більше --- це Колесо Медитації.
Медитативні практики полягають у тренуванні доброго мислення, створенні благих емоцій,
які ведуть до переживань і реалізації.

Намір. Щоб накопичена карма дозріла має статися таке. Ваше подвійне мислення
має розділити Мить на Об'єкт, Суб'єкт та Дію. Після завершення дії у цьому стані
ви повинні відчути задоволення від цього процесу. Коли це сталося, вітаємо,
ви щойно віддалилися від Будди на одну йокто-Карму. Є і хороша новина, карма
зазвичай дозріває у хворобі, смерть і всякій гидоті, так що непомітити якщо
ви нашкоднічали не вийде. Наприклад, ви тримаєте чашку в руках і хочете випити чаю,
ви підносите об'єкт чашку до свого суб'єктного рота і чините дію випивання чаю,
отримуючи від цього задоволення у вигляді тепла, що розтікається по вашому тілу.
Просвітлений Майндстрім не поділяє своє мислення на себе, чашку, підношення та
задоволення, тобто Просвітлений Майндстрім не накопичує карми.