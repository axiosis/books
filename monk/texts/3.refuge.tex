\section{Прихисток}

Если вы дочитали до третьего поста -- это означает, что рекламная
компания работает, и вас действительно заинтересовал Просветленный
 Майндстрим. Нужно сказать, что буддийское мировозрение — это не
миссионерская религия, нам все равно хотите ли вы быть буддистом
или нет, рано или поздно и так все будут буддистами, мы никуда
не спешим. Допустим вы решили приобщиться к ордену Благородных,
что нужно для этого делать?

Существуют несколько ступеней входа в движение, для простоты
будем говорить, что их три: 1) для тупых, 2) среднячков и 3)
развитых физически и умственно. Вообще в буддизме много
дискриминации и провокаций, так что привыкайте. Условно
чем человек развитее, тем большее значение придается Учителю.
Значение Учителя переоценить сложно, поэтому буддизм
классифицируется как ламаизм, гуруизм, паханат или дедовщина,
называйте как хотите, некоторые ламы шутят и называют это
Ламским Рабством. Но посколько я не уверен, что среди вас
есть даже среднячки, то будем подразумевать, что все читатели
этого блога тупые, или иначе говоря, находятся на первом уровне. 

Как войти в движение Благородных и ознакомиться с Просветленным
Майндстримом? Нужно принять прибежище в Будде, Дхарме и Санхе!
Первые два слова вы уже знаете, но что такое Санха? Если
Дхарма — это благое мышление вашего майндстрима, то Санха — это
благое мышление других майндстримов, ваших друзей, идущих по
пути Благородных. Что означает Принятие Прибежища?
Прибежище — это осознанный выбор и намерение присоединиться
к движению Благородных, другими словами желание личного полного
и безусловного освобождения и такого же абсолютного и бесповоротного
освобождения для других. Не достижения сиюминутного удовольствия,
после которого часто (а может и всегда) бывает отходняк, а полной
Нирваны — абсолютного и конечного максимального счастья (без всяких минимаксов).

Почему это называется Прибежищем? Потому что путь этот, как и
ваша жизнь, опасен и сложен. На пути может происходить что
угодно, вы в любую секунду можете умереть от разорвавшегося
тромба в мозгу или сумасшедшего мотоциклиста на дороге. А от
опасностей хорошо иметь зонтик или иншуранс. Такая страховка
и есть Прибежище. Оно должно выдаваться только сертифицированными
филиалами, и подразумевает осознанное подписание контракта.
В контракте вы обязуетесь свято чтить и уважать объекты
прибежища или как их называют Три Корня: 1) Будду Шакьямуни —
первого Учителя (во временных пределах этого Большого Взрыва)
поставившего вопрос о существовании Просветленного Майндстрима
и успешно подтвердившего эту гипотезу; 2) Дхарму — свое личное
благое мышление, которое является очень сильным оружием в
борьбе за абсолютную свободу; 3) Санху — благое мышление
других существ принявших такой же контракт. Со своей стороны
Движение Благородных, лично под покровительством Будды Шакьямуни,
обязуется помогать вам на пути достижения вашей благородной
цели — Просветленного Майндстрима. Остерегайтесь подделок,
на сертификате должны быть все знаки Будды Шакьямуни: Четыре
Благородные истины, Учитель, подходящее Время, Место, ваша
подпись кровью и подпись кровью Учителя. 

Автоприбежище. Ясно, что через этот стрим вы не получите
прибежище, но некоторое представление о нем вы уже получили,
и если очень сильно захотеть, то можно принять его автономно
(Автоприбежище) и уже потом встретиться с Учителем, который
обладает Сертификатом. 

Сертификаты. Есть список филиалов выдающих такие сертификаты:
Бон (небуддийское движение, или китайская подделка Буддизма,
но работающая), Норбупа (сертификаты выдаваемые Намкаем Норбу
Ринпоче), и официальные школы тибетского Буддизма (ведущие
начала от Будды Шакьямуни) — Гелугпы, Джонанги, Кагьюпы,
Сакьяпы, Ньингмапы. Безумный Монах имеет сертификат Ньингмапы
— или в переводе на русский — Старпёров (кличка с негативным
окрасом, которой обзывали ньингмапинцев другие школы, и которая
впоследствии стала именем школы, ведь лучшая защита — это открытость).

Ритуал. Ясно, что Автоприбежище не обладает никаким ритуалом,
вы решаете для себя сами и потом уже, в этой или следующих
жизнях встречаетесь с Учителем. Можно дальше усложнять ритуал,
находить ближайшего ламу или наставника, которые лично дадут
вам Прибежище, или получить аудиенцию у высокого ламы и
раскатать ритуал по полной программе с обрезанием (волос),
рисом и официальным празднованием. Вообще в буддизме нет
ничего обязательного, нет догм, для любого правила есть
исключение. Это немного усложняет продвижение, но также
и открывает дверь в лайфхаки. Так что нужно знать законы
и успешно ими пользоваться для своей личной выгоды на пути
Просветленного Манйдстрима.

