\section{Зібрання тантр Старої Школи}
 
В прошлых жизнях я был книжным червем, который ползал и ел святые индийские свитки с Дхарма текстами. Поэтому сегодня мы поговорим о тантрическом каноне. Согласно традиции Ньингма (rnying ma) тибетского буддизма Собрание Тантр Древних (rnying ma rgyud 'bum) представляет собой коллекцию драгоценных текстов, переведенных в первый период распространения буддизма в Тибете, Ранний Период (bstan pa snga dar), который длился с 7 по 10 век нашей эры.

Ранний Период. Становление Буддизма в Тибете, его пышное и величественное процветание, и затем закат. К 8 веку, буддизм официально поддерживался правителями Тибета. Монастырь Самье (bsams yas) был открыт в 775 году Трисонгом Деценом (khri srong lde btstan). Первые монахи тибетского буддизма обнаруживаются в 779 году. С широким размахом были начаты переводы текстов учениками монастыря под руководством ученых-филологов Вималамитры и Вайрочаны. Однако эти начинания столкнулись с темным периодом тибетской истории (последняя декада 9 века, начало первой декады 10 века). С убийством царя Рэлпачэна (ral pa can) в 838 году Тибетская Империя вошла в мир гражданской войны и неспокойствия. Рэлпачэн был свергнут своим братом Ландармой в результате дворцового переворота при поддержке бонской аристократи. Последний царь правил недолго, через 4 года был убит монахом, после чего Тибет распался на мелкие княжества и погряз в гражданской войне.

Второй период распространения буддизма. Поздний период (phyi dar) начался, в зависимости от источников, в последней декаде 10 века с приходом учителя Лумэ Шераба Цултима (klu mes shes rab tshul khrims) или в первой декаде 11 века с приходом Атиши.

Конфликт переводов разных периодов и разделение на хипстеров и олдскульный хардкор. Оба периода (snga dar) и (phyi dar) различаются своими переводами Тантр: ранние переводы (snga 'gyur) выполненные под руководством Вайрочаны и современников (8-9 вв.) и поздние переводы (phyi 'gyur) начавшиеся под руководством Ринчена Зангпо (rin chen bzang po). Школа Ньингма признает старые переводы как наиболие значимые драгоценные тексты нетибетских манускриптов датируемых первым периодом становления буддмизма и его расцвета. Другие школы: Сакья (sa skya), Гелуг (dge lugs) и Кагью (bka rgyud) которые принадлежат второму периоду распространения буддизма и известны как новая традиция Сарма (gsar ma) отвергают как сомнительные и неаутентичные многие переводы rnying ma. Основная претензия состоит в невозможности аутентификации ориганальных индийских источников. В новой традиции (gsar ma) единственными канонизированными собраниями текстов являются Канжур и Танжур. bka 'gyur и bstan 'gyur. Они содержат более 2500 компиляций и являются аутентичными индийскими переводами обеих периодов snga dar и phyi dar.

Бутанская редакция прдеставлена манускриптами mtshams brag (Цамдрак, сокр. TB), sgang steng A, sgang steng В, sbra me'i rtse. mtshams brag состоит из 46 томов и является фотокопией сделанной в 1982 году в монастыре mtshams brag в Бутане. Оригинальный автор неизвестен, однако известно, что издание было написано в период 1728—1748 с оригинала Punakha по приказу sprul sku ngag dbang 'brug pa (1682—1748). Похоже, что это издание более всего похоже на тоже 46 томовый sgang steng В, который недавно был каталогизирован Кантвеллом, Майером, Ковалевским и Ахардом. sgang steng A и sbra me'i rtse пока не каталогизированы и не изучены, их след ведет к tshul khrims rdo rje (1598—1669) и ngag dbang kun bzang rdo rje (1680—1723). Две изученые копии могут быть написаны ориентировочно в 1640-1650 гг и таким образом ведут к tshul khrims rdo rje (Кантвелл, Майер). Также маловероятно, что mtshams brag копировался с sgang steng В или наоборот (Кантвелл, Майер). Скорее они копировались с другого источника, которым может быть, а может и не быть sgang steng A или sbra me'i rtse. В то время как все коллекции являются в том или ином смысле копиями редакции Деге, Цамдрак является уникальной и непохожей на другие коллекции с максимальным числом текстов 939.

Редакция южно-центрального Тибета представлена gting skyes (Тинкье, сокр. TK), rig 'dzin rje (Waddell), nub ri (skyid grong) манускриптами и манускриптом из Катманду. Мы знаем мало об их происхождении в Тибете. Является в том или  ином смысле копией редакции Деге или редакцией которая предшествовала Деге. Однако некоторые тексты из Тинкье являются такими которых нету в Деге или они изменены. Количество текстов — 447.

Редакция восточного Тибета sde dge (Деге, сокр. DG), сделанная в период 1794—1798 гг., в районе Деге, в восточном Тибете, является первым Собранием Тантр Древних выполненных в виде деревяной гравюры. Ее составление было поручено царицей Деге Цеванг Ламо (tshe dbang lha mo) и было выполнено Гьюрмед Цвеванг Чогдруб ('gyur med tshe dbang mchog grub) 1761—1829. В 1797 Цвеванг Чогдруб подготовил детализированный каталог состоящий из двух частей (dkar chag). В первой части была история школы Ньингма, а во второй индекс-каталог всех тантр, их названий, номера томов, автора переводов, номера глав для каждого текста и имена мастеров которые совершали передачу текстов. Считается, что [mkhyen brtse rin po che, Introduction], что редакция Деге является ни чем иным как реиндексированной, облоченной в деревянные гравюры редакцией gting skyes (Тинкье, сокр. TK) выполненной Джигме Лингпа ('jigs med gling pa) 1730—1798. Существует 4 копии этих манускриптов. Одна копия каталогизирована Орофино в 1998 году находится в Tucci Fund Collection в Is.IAO Библиотеке Востока в Риме. Она была подарена Далай Ламой XIV Тензином Гьятсо (bstan 'dzin rgya mtsho) в 1949 в Лхасе. Вторая копия находится в Катманду (переведена туда из Национальной Библиотеки в 1992). Третья копия находится в частной библиотеке Гурме Дордже и была куплена в Деге в 1989 году. Четвертая копия сейчас сканируется и может быть доступна со временем на TBRC. Эта копия была каталогизирована Гарсоном, Хиллом, Возом и Вайнбергом под руководством Германо. Количество текстов в коллекции Деге — 448.

Коллекция Вайрочаны (bai ro'i rgyud 'bum) была найдена в северном Ладакхе. Линия передачи этих текстов утрачена. Только линия учителей Ньингма может что-то с увереностью сказать об этих текстах. Коллекция содержит 8 томов и влючает в себя 196 текстов. Насколько известно Безумному Монаху тантры Вайрочаны утратили линию передачи.

