\section{Зібрання тантр Старої Школи}
 
У минулих життях я був книжковим черв'яком, який повзав
і їв святі індійські сувої з текстами Дхарми. Тому сьогодні
ми поговоримо про тантричний канон. Згідно з традицією
Ньінгма (rnying ma) буддизму Тибету Збори Тантр
Древніх (rnying ma rgyud 'bum) являє собою колекцію
дорогоцінних текстів, перекладених в перший період поширення
буддизму в Тибеті, Ранній Період (bstan pa snga dar),
який тривав з 7 по 10 нашої ери.

Ранній період. Становлення Буддизму в Тибеті, його пишне
і величне процвітання, а потім захід сонця. До 8 століття
буддизм офіційно підтримувався правителями Тибету. Монастир
Сам'є (bsams yas) був відкритий у 775 році Трісонгом
Деценом (khri srong lde btstan). Перші ченці буддизму
Тибету виявляються в 779 році. З широким розмахом було
розпочато переклади текстів учнями монастиря під керівництвом
вчених-філологів Вімаламітри та Вайрочани. Однак ці починання
зіткнулися з темним періодом історії Тибету (остання декада
9 століття, початок першої декади 10 століття). Із вбивством
царя Релпачена (ral pa can) у 838 році Тибетська Імперія
увійшла у світ громадянської війни та скорботи. Релпачен
був повалений своїм братом Ландармою внаслідок палацового
перевороту за підтримки бонської аристократі. Останній
цар правив недовго, через 4 роки був убитий ченцем, після
чого Тибет розпався на дрібні князівства та загруз у громадянській війні.

Другий період поширення буддизму. Пізній період (phyi dar) почався,
залежно від джерел, в останній декаді 10 століття з приходом
вчителя Луме Шераба Цултіма (klu mes shes rab tshul khrims) або
в першій декаді 11 століття з приходом Атіші.

Конфлікт перекладів різних періодів та поділ на хіпстерів та олдскульний хардкор.
Обидва періоди (snga dar) і (phyi dar) розрізняються своїми перекладами
Тантр: ранні переклади (snga 'gyur) виконані під керівництвом Вайрочани
і сучасників (8-9 ст.) і пізні переклади (phyi 'gyur), що розпочалися
під керівництвом Рінчена Зангпо ( rin chen bzang po). Школа Ньінгма
визнає старі переклади як найбільш значущі дорогоцінні тексти нетибетських
манускриптів, що датуються першим періодом становлення буддмізму та його
розквіту. Інші школи: Сакья (sa skya), Гелуг (dge lugs) та Каг'ю (bka rgyud),
належать до другого періоду поширення буддизму і відомі як нова традиція
Сарма (gsar ma) відкидають (відкладали) як сумнівні та неавтентичні багато
перекладів rnying ma. Основна претензія полягає у неможливості
автентифікації оригінальних індійських джерел. У новій традиції (gsar ma)
єдиними канонізованими зборами текстів є Канжур та Танжур. bka 'gyur та bstan 'gyur.
Вони містять понад 2500 компіляцій і є автентичними індійськими
перекладами обох періодів snga dar та phyi dar.

Бутанська редакція представлена манускриптами mtshams brag (Цамдрак, скор. TB),
sgang steng A, sgang steng В, sbra me'i rtse. mtshams brag складається з
46 томів і є фотокопією зробленою в 1982 році в монастирі mtshams brag у
Бутані. Оригінальний автор невідомий, проте відомо, що видання було написано
в період 1728—1748 з оригіналу Punakha за наказом sprul sku ngag dbang 'brug
pa (1682—1748). Схоже, що це видання найбільше схоже також на 46 томовий sgang
steng В, який нещодавно був каталогізований Кантвеллом, Майєром, Ковалевським
і Ахардом. sgang steng A і sbra me'i rtse доки каталогізовані і вивчені,
їх слід веде до tshul khrims rdo rje (1598—1669) і ngag dbang kun
bzang rdo rje (1680—1723). Дві вивчені копії можуть бути написані
орієнтовно в 1640-1650 рр. і таким чином ведуть до tshul khrims rdo rje (Кантвелл, Майєр).
Також малоймовірно, що mtshams brag копіювався з sgang steng або
навпаки (Кантвелл, Майєр). Швидше, вони копіювалися з іншого джерела,
яким може бути, а може і не бути sgang steng A або sbra me'i rtse.
У той час як усі колекції є в тому чи іншому сенсі копіями редакції Деге,
Цамдрак є унікальним і несхожим на інші колекції з максимальним числом текстів 939.

Редакція південно-центрального Тибету представлена gting skyes (Тінк'є, скор. TK),
rig'dzin rje (Waddell), nub ri (skyid grong) манускриптами та манускриптом з Катманду.
Ми знаємо мало про їхнє походження в Тибеті. Є у тому чи іншому сенсі копією редакції
Деге чи редакцією яка передувала Деге. Однак деякі тексти з Тінк'є є такими яких
немає в Деге або вони змінені. Кількість текстів --- 447.

Редакція східного Тибету sde dge (Деге, скор. DG), зроблена в період 1794-1798 рр..,
В районі Деге, в східному Тибеті, є першими Зборами Тантр Древніх виконаних у
вигляді дерев'яної гравюри. Її складання було доручено царицею Деге Цевангом
Ламо (tshe dbang lha mo) і було виконано Г'юрмед Цвеванг Чогдруб ('gyur med
tshe dbang mchog grub) 1761-1829. У 1797 Цвеванг Чогдруб підготував деталізований
каталог, що складається з двох частин (dkar chag). У першій частині була
історія школи Ньінгма, а у другій індекс-каталог усіх тантр, їх назв, номери
томів, автора перекладів, номери розділів для кожного тексту та імена майстрів,
які здійснювали передачу текстів. Вважається, що [mkhyen brtse rin po che, Introduction],
що редакція Деге є ні чим іншим як реіндексованою, облаченою в дерев'яні гравюри
редакцією gting skyes (Тінк'є, скор. TK) виконаної Джигме Лінгпа (jigs9 g3 pa).
Існує 4 копії цих манускриптів. Одна копія каталогізована Орофіно в 1998 році
знаходиться в Tucci Fund Collection в Бібліотеці Сходу в Римі. Вона була
подарована Далай Ламою XIV Тензін Г'ятсо (bstan 'dzin rgya mtsho) в 1949 в Лхасі.
Друга копія знаходиться в Катманду (переведена туди з Національної Бібліотеки у 1992).
Третя копія знаходиться у приватній бібліотеці Гурме Дордже і була куплена у Деге у 1989 році.
Четверта копія зараз сканується та може бути доступна згодом на TBRC. Ця копія
була каталогізована Гарсоном, Хіллом, Возом та Вайнбергом під керівництвом Германо.
Кількість текстів у колекції Деге --- 448.

Колекція Вайрочани (bai ro'i rgyud' bum) була знайдена у північному Ладакху.
Лінію передачі цих текстів втрачено. Тільки лінія вчителів Ньінгма може щось
впевнено сказати про ці тексти. Колекція містить 8 томів і включає 196 текстів.
Наскільки відомо Божевільному ченцю тантри Вайрочани втратили лінію передачі.
