\section{Філософьські школи Буддизму}

Немного информации для задротов и любителей истории. Сегодня мы поговорим о важности воззрения, так как  оно закладывает фундамент мышления. Только изменив воззрение можно уже избавится от половины неблагой кармы. В этом смысле полезно иметь гибкое воззрение, и абсолютно гибкое воззрение исходит из Естественного Состояния. Что это не спрашивайте, так как ответа не существует. Это воззрение настолько прекрасно, что оно трудно выражается словами. Даже Будда Шакьямуни не смог дать подходящего сосудам учения о воззрении сокрыв Сутру Сердца, ожидая прихода Нагарджуны и взращивания сосудов для понимания объяснений. Какое счастье родится в век, когда мы можем воспринимать учение о воззрении. 

Буддизм находится посреди двух вопросов: главного вопроса западной философии сводящийся с дихотомии идеализм-материализм, а также вопроса связанный с принципиальной познаваемостью. С точки зрения современных воззрений буддизм напоминает смесь субъективного идеализма основоположниками которого можно считать Канта, Фихте, Беркли, а также безатомного относительного материализма, постулирующего материальность мышления.

Нагарджуна, комментируя Сутру Сердца, развивает линию срединного пути начерченой Буддой Шакьямуни, запредельную не только мирской и аскетический жизни, но и запредельную существованию и не-существованию феноменов. На примере трех времен показывается, что настоящее обусловлено несуществующими понятиями прошлого и будущего, таким образом все существующие феномены являются причинно-условно существующими, существование которых поддерживается взаимозависимым возникновением. Кроме того, Нагарджуна показывает в Мадхьямике, что невозможно описать абсолютную истину концептуальными понятиями, которые всегда оказываются относительными друг другу. Таким образом пустотность феноменов и их взаимозависимость являются двумя качествами одного.

С абсолютной же точки зрения, Нагарджуна описывает истину, как единство нирваны и самсары. Понимание этой истины подобно орлу, который планирует в небе, не взмахивая крыльями, однако попасть на высоту можно только научившись махать.

Ограниченно трактовав Нагарджуну, Мадхьямика pазделилась на две школы — Мадхьямика-Сватантpикy (Бхававивека, Камалашила, Шантаpакшита) и Мадхьямика-Пpасангикy (Аpьядэва, Бyддхапалита, Чандpакиpти). Последняя, благодаря адвокатской поддержке Чандракирти, послужила основой философской школы Гелуг (и Кагью), фокусируя фоззрение на абсолютной истине и применяя критический подход к относительному мышлению, доказывая его несостоятельность и противоречивость.

Йогачара, в отличии же от Прасангики, активно использовала интрументы мышления, для его освобождения, чем породила не только множество эзотерических конструкций, свойственных школе Ньингма, но и разработала систему категоризации абсолютной истины (10 ступеней бодхисатв). Благодаря Шантракшите, Йогачара вместе с воззрением Сватантрики-Мадхьямики, допускающим конструктивные концептуальные "философские" разработки, послужили основой для производной — Йогачарьи-Сватантрики-Мадхьямики, которая преподавалась Шантракшитой в Тибете во времена Трисонг Децена и Падмасамбхавы.

Именно эта сутрическая традиция, Йогачара-Сватантрика-Мадхьямика используется в Ньингма. Относительная истина здесь трактуется согласно воззрению Читтаматры. Категоризированая абсолютная истина здесь трактуется согласно Саутантрике-Сватантрика-Мадхьямики. Некатегоризированная абсолютная истина здесь трактуется здесь согласно Прасангике. Все это кажется полным безумием и схоластикой, но написано лишь для того, чтобы показать тиртикам-скептикам, что автор имеет буддийское образование. Задроты бегут читать википедию, а коллекционеры покупать рукопись Нагарджуны на аукционе Сотби.

