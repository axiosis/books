\section{Філософьські школи Буддизму}

Небагато інформації для задротів і любителів історії. Сьогодні
ми поговоримо про важливість погляду, оскільки він закладає фундамент мислення.
Тільки змінивши думку можна вже позбутися половини неблагої карми.
У цьому сенсі корисно мати гнучку думку, і гнучке думка виходить з
Природного Стану. Що це не питайте, тому що відповіді не існує.
Ця думка настільки красива, що вона складно виражається словами.
Навіть Будда Шакьямуні не зміг дати відповідного судинам вчення про думку
сховавши Сутру Серця, чекаючи приходу Нагарджуни і вирощування плодів, що здатні зрозуміти пояснення.
Яке щастя народиться у вік, коли ми можемо сприймати вчення про думку.

Буддизм перебуває між двох питань: головного питання західної філософії,
що зводиться в дихотомії ідеалізм-матеріалізм, а також питання пов'язані
з принциповою пізнаваністю. З точки зору сучасних світоглядів буддизм
нагадує суміш суб'єктивного ідеалізму основоположниками якого вважатимуться
Канта, Фіхте, Берклі, і навіть безатомного відносного матеріалізму, що постулює матеріальність мислення.

Нагарджуна, коментуючи Сутру Серця, розвиває лінію серединного шляху
накреслену Буддою Шак'ямуні, позамежну не тільки для мирського та аскетичного життя,
а й позамежну існуванню та неіснуванню феноменів. На прикладі трьох часів
показується, що сьогодення зумовлене неіснуючими поняттями минулого і майбутнього,
таким чином, всі існуючі феномени є причинно-умовно існуючими, існування яких
підтримується взаємозалежним виникненням. Крім того, Нагарджуна показує в Мадхіміці,
що неможливо описати абсолютну істину концептуальними поняттями, які завжди
виявляються відносними один до одного. Таким чином порожнеча феноменів
та їх взаємозалежність є двома якостями одного.

З абсолютної точки зору, Нагарджуна описує істину, як єдність нірвани і самсари.
Розуміння цієї істини подібне до орла, який планує в небі, не змахуючи крилами,
проте потрапити на висоту можна лише навчившись махати.

Обмежено трактувавши Нагарджуну, Мадхіміка розділилася на дві школи ---
Мадхіміка-Сватантріку (Бхававівека, Камалашила, Шантаракшыта) і Мадхіміка-Прасангіку
(Арьядева, Будхапіта, Чандракірті). Остання завдяки адвокатській підтримці Чандракірті
послужила основою філософської школи Гелуг (і Каг'ю), фокусуючи фозгляд на абсолютній
істині і застосовуючи критичний підхід до відносного мислення, доводячи його неспроможність і суперечливість.

Йогачара, на відміну ж від Прасангіки, активно використовувала інструменти мислення,
для його звільнення, чим породила не тільки безліч езотеричних конструкцій,
властивих школі Нінгма, але й розробила систему категоризації абсолютної істини (10 ступенів бодхісатв).
Завдяки Шантракшиті, Йогачара разом із думкою Сватантрики-Мадхьяміки, що припускає
конструктивні концептуальні «філософські» розробки, послужили основою
для похідної --- Йогачари-Сватантрікі-Мадх'яміки, яка викладалася Шантракшитою в Тибеті за часів Трисонга Децена та Падмасабхави.

Саме ця сутрична традиція Йогачара-Сватантрика-Мадх'яміка використовується в Ньінгма.
Відносна істина тут трактується на думку Чіттаматри. Категоризована абсолютна
істина тут трактується згідно Саутантріка-Сватантріка-Мадх'яміки. Некатегоризована
абсолютна істина тут трактується згідно Прасангіки. Все це здається повним
безумством і схоластикою, але написано лише для того, щоб показати
тиртикам-скептикам, що автор має буддійську освіту. Задроти біжать
читати вікіпедію, а колекціонери купують рукопис Нагарджуни на аукціоні Сотбі.