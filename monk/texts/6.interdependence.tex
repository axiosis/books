\section{Взаємозалежнє виникнення}

Второгодники и любители почитать наперед википедию выкрикивают из зала что-то про Ваджрную Сеть и мантру Взаимозависимого Возникновения. Еще до Мадхьямики принято давать учение об относительности Всех Майндстримов. Если кратко то суть такая, что если нет условий для счастья или страдания, т.е. других существ, то и вашего счастья или страдания бы не существовало. Поэтому если гипотетически в Едином Квантовом Вакууме останутся два существа, то Самсара еще может существовать, но если останется только один, то он становится Буддой автоматически, так как у него не остается зеркал-условий. И наоборот, только два существа вместе могут родить Самсару, создавая взаимную относительность существования.

Переоценить взаимозависимое возникновение как понятие в буддизме очень тяжело, оно связано со следующими вещами:  процесс реинкарнации мышления, пустотность, непрерывность, несуществование автономного майндстрима, Четыре Истины Благородных, Единый Квантовый Вакуум.

Реинкарнация. Взаимозависимое возникновение — это базовый принцип по которому вообще возникают любые мысли в Едином Квантовом Вакууме. Каждая моя следующая мысль зависит от мыслей всех существ. Все они являются условиями. Мое мышление является причиной. Кроме этого в формулу еще входит благословение всех Будд как условие и которое является необходимой причиной возникновения следующей мысли. Если будды перестанут благословлять взаимозависимое возникновение то реинкарнация мышления живых существ попросту прекратится.

Пустотность. Взаимозависимое возникновение — также является ключом к пониманию пустотности всех явлений. Ведь все явления, которые существуют в наших мышлениях, не самобытны, а зависят от мышлений других существ, поэтому нельзя говорит что что-то существует само по себе. Все существует взаимозависимо и с благословения будд.

Непрерывность. Взаимозависимое возникновение и счетное бесконечное множество всех живых существ также связано с непрерывностью мышления. Мощность множества всех подномножест всех существ всех вселенных выражается как множество континуума. Это означает что мышление каждого отдельного существа можно представить как вещественное число или Коиндуктивный Майндстрим.

Если хотите помедитировать над Взаимозависимым Возникновением — вот вам его мантра:
\\
\\
ОМ ЙЕДХАРМА ХЕТУПРАБХАВА ХЕТУНТЕКХАН  ТАТХАГАТО ХАЙОВАДАТ ТЕКХАНЦАЙО  НИРОДХА ЭВАМВАДИ МАХАШРАМАНА СОХА.

