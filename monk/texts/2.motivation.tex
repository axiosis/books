\section{Мотивація}

Несмотря на то, что информация в интернете обесценена,
раз вы попали на этот канал, значит у вас есть веские
причины тратить свое драгоценное время на мысли Безумного
Монаха. Спросите себя, какая у вас мотивация. Мотивация
в буддизме самое главное, мы еще не раз будем возвращаться
к этому вопросу. Если вы хотите просто избавится от
страдания — это еще не настоящий, топовый буддизм или
как говорят, Алмазная Колесница. Возможно вам будет
интереснее и полезнее с квалифицированным психоаналитиком
или даже психиатром. Мотивация же Алмазной Колесницы
заключается в двух вещах: первая — это желать окончательного
избавления от страданий навсегда для себя, вторая —
окончательное и бесповоротное освобождение от страданий
других. Если хоть одна часть отсутствует — это не мотивация
Алмазной Колесницы. Такая мотивация называется Бодхичиттой
или Просветленным Майндстримом. Она и есть смысл всего
буддизма, да что там буддизма, большинство религий и
философских систем совместимы с этой мотивацией. Это
универсальная песня про мир во всем мире, которая проста
и доступна на словах, но так тяжело реализуема в мировом
масштабе. Что означает Окончательное Освобождение? Разве
такое возможно, хотя бы в пределах одной жизни, чтобы никогда
больше не страдать? Именной такой вопрос поставил Будда
Шакьямуни, и все говорит о том, что он получил положительный ответ на этот вопрос, несмотря на то, что ушел из жизни он от обычного пищевого отравления, без всяких радуг и тел света. Всех, кто ставил такой вопрос и пытался найти в себе эту мотивацию, мы будем называть Благородные.

Многие из вас, тиртиков-интеллектуалов, ищут магию, хотят постичь тайны бытия, увлекаются каббалой, шиваизмом и другими лямбда-системами, которые тоже декларируют Глубокое Знание. Возможно вы даже успешный выживший психонавт или просто хипарь, но будьте с собой откровенны: насколько близко вы приблизились к состоянию Просветленного Майндстрима? Когда последний раз вы страдали? Достойны ли вы быть причисленными к Семье Благородных? Многие начнут приводить в пример святых, в том числе и Иисуса Иосифовича Христа, но смог ли он прекратить страдания всех христиан или хотя бы одного из них раз и навсегда?

Мотивация Алмазной Колесницы или Просветленный Майндстрим — это то, чем обладают все существа, они все хотят счастья, в любое время, в любом месте, без какого либо обучения, с самого своего рождения и до самой смерти. Конечно есть BDSM и мазохисты, их счастье заключается в получении максимальной боли или максимального унижения (на самом деле функция минимакса), но это лишь показывает относительность счастья, в абсолютном же смысле ничего не меняется — у них свои представления о счастье и они хотят его. Согласитесь, что было бы гораздо сложнее объяснять людям про Счастье и Страдание, если бы они не понимали этого на инстинктивном уровне. Осталось разобраться с вопросом как добиться абсолютного Счастья и навсегда избавится от Страданий. Как сделать это на время мы все знаем, и это не экстраполируется на бесконечность. По крайней мере у Иисуса Иосифовича не получилось, закончилось всё войнами, крестовыми походами, средневековым мраком и харасментом молодых послушников.

