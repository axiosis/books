\section{Мотивація}

Незважаючи на те, що інформація в інтернеті знецінена,
раз ви вже потрапили на цей стрім, значить у вас є вагомі
причини витрачати свій дорогоцінний час на думки Божевільного
Монаха. Запитайте себе, яка у вас є мотивація. Мотивація
у буддизмі найголовніше, ми ще не раз повертатимемося
до цього питання. Якщо ви хочете просто позбавиться від
страждання --- це ще не справжній, топовий буддизм або
як кажуть, Алмазна Візниця. Можливо вам буде
цікавіше та корисніше з кваліфікованим психоаналітиком
чи навіть психіатром. Мотивація ж Алмазної Колісниці
полягає у двох речах: перша --- це бажати остаточного
звільнення від страждань назавжди для себе, друга ---
остаточне та безповоротне звільнення від страждань
для інших. Якщо хоч одна частина відсутня --- це не мотивація
Алмазної Візниці. Така мотивація називається Бодхічіттою
або Просвітленим Майндстрімом. Вона і є сенсом всього
буддизму, та що там буддизму, більшість релігій і
філософські системи сумісні з цією мотивацією. Це
універсальна пісня про мир у всьому світі, яка проста
і доступна на словах, але так важко реалізована у світовому
масштабі. Що означає Остаточне Звільнення? Хіба
таке можливо, хоча б у межах одного життя, щоб ніколи
більше не страждати? Іменне таке питання поставив Будда
Шакьямуні, і все говорить про те, що він отримав позитивну
відповідь на це питання, незважаючи на те, що пішов
з життя він від звичайного харчового отруєння,
без будь-яких веселок і тіл світла. Усіх, хто ставив
таке питання і намагався знайти в собі цю мотивацію,
ми називатимемо Благородні.

Багато хто з вас, тиртиків-інтелектуалів, шукають
магію, хочуть осягнути таємниці буття, захоплюються
каббалою, шиваїзмом та іншими лямбда-системами,
які теж декларують Глибоке Знання. Можливо ви
навіть успішний психонавт, що вижив, або просто хіпар,
але будьте з собою відверті: наскільки близько
ви наблизилися до стану Просвітленого Майндстріму?
Коли востаннє ви страждали? Чи гідні ви бути зарахованими
до Сім'ї Благородних? Багато хто почне наводити
приклад святих, у тому числі й Ісуса Йосиповича Христа,
але чи зміг він припинити страждання всіх християн
або хоча б одного з них раз і назавжди?

Мотивація Алмазної Візниці або Просвітлений Майндстрім ---
це те, що мають усі істоти, вони всі хочуть щастя,
у будь-який час, у будь-якому місці, без будь-якого
навчання, від самого свого народження і до самої смерті. Це бажання вбудоване в істоту. Звичайно, є BDSM і мазохісти, їх щастя полягає
в отриманні максимального болю або максимального
приниження (насправді функція мінімаксу), але це
лише показує відносність щастя, в абсолютному
сенсі нічого не змінюється --- у них свої уявлення
про щастя і вони хочуть його. Погодьтеся, що було б
набагато складніше пояснювати людям про Щастя
та Страждання, якби вони не розуміли цього на
інстинктивному рівні. Залишилося розібратися
з питанням як досягти абсолютного Щастя і назавжди
позбудеться страждань. Як зробити це на якийсь
час ми всі знаємо, і це не екстраполюється на
нескінченність. Принаймні у Ісуса Йосиповича
не вийшло, закінчилося все війнами, хрестовими
походами, середньовічним мороком та харасментом молодих послушників.
