\section{Реверберації багатовимірного звуку}

Сегодня согласно тибетскому календарю — благоприятный день,
а значит есть шанс почитать проповедь от Безумного Монаха.
Согласно тантре Дра Талгьюр из раздела Менгагде Дзогчен все
миры и все вселенные, все Большие Взрывы, в которые могут
(а могут и не) приходить Будды появляются из вибрации
многомерной мембраны. Другими словами мы с вами — самые
настоящие колебания Единого Квантового Вакуума, продукты
вибраций Многомерного Звука. Тем кому сложно вообразить
божество используя небесное плотно как канвас, а белые
бегающие точки как кисть (удел визуалов), могут войти
движение через вибрации звука (удел аудиалов). Поэтому
расчехляйте ваши ONKYO, DENON, сегодня будет дхарма дискотека.

Вообще-то мантры так просто нельзя передавать. В традиции
очень важно, чтобы мантру вам напел или просто формально
продиктовал по буквах (с любой скоростью) Учитель, иначе
такая передача будет считаться некошерной. Не надиктовал
по телефону (Норбупа стайл), а лично, так чтобы звук
коснулся ваших ушных мембран и исходил при этом из голосовых
связок Учителя. Но поскольку Монах Безумен, а в интернете
нет ничего кошерного, а в буддизме нет догм, то кое-что я
вам все же расскажу. Это будет самый настоящий Дхарма Попсовый
альбом предварительных практик. В нем будут собраны основные
практики в виде мантр не нуждающихся в передачи. Формально
вы можете слушать, но практиковать (самим читать) понарошку.
Как и в случае Автоприбежища, валите все на Безумного Монаха (если очень хочется самим потренироваться)!

Каждый буддист начинает свой день с Молитвы к Восьми Приносящим Удачу.
Это песнь Мипама Ринпоче, воплощения Манджушри, не оставившего после
себя реинкарнаций (полная реализация, окончательное обретение плода
эволюции мышления)\footnote{\url{https://tonpa.guru/nendro/1. Auspicious.mp3}}
Тибетский текст и перевод здесь\footnote{\url{https://longchenpa.guru/gter.ma/snying.thig.rtsa.pod/collections/1/prayerbook.pdf}}
Исполняет на записи руководитель Шри Синха центров по всему миру, представитель
Дзогчен монастырей и Долины Дзогчен на этой земле — Дзогчен Кенпо Чога Ринпоче. 

Сразу после этой молитвы мы вспоминаем прибежище, которое уже давали.
\footnote{\url{https://tonpa.guru/nendro/2. Take Refuge.mp3}}. Эту мелодию я впервые
услышал по Радио Lounge FM (United Peace Voices), ее можно играть
в кафе и отлично шифроваться (никто и не заподозрит, что вы сектант).
Трансрипция была дана в первых постах, а смысл стиха заключается в
принятии прибежища в Будде, Дхарме и Сангхе.

Формально первые две записи — это тибетские стихи, не мантры
на божественном санскрите. Но сила их настолько огромна, что
они давно уже перекочевали в разряд мантр, стихо-мантр. После
этих записей мы приводим свое мышление в успокоение Стослоговой
Мантрой Ваджрасаттвы. Ваджрасаттва — это принц вошедший в Мандалу
Победителей с востока, у которого было очень много непослушных
учеников и он дал обет, что его мантра будет помогать всем у кого
проблемы с чистотой мышления. Это своего рода метапокаяние. Есть
конечно и более жесткие виды очистительных молитв, которые могут
очистить даже нарушенные Самаи, такие как Нарак Кончак, но вы пока
не натворили больших делов, вы дети и молитвы вам нужны тоже начального
уровня. Встречайте Воплощение Ваджрасаттвы в звуке от
Сакьяпинских Монахинь\footnote{\url{https://tonpa.guru/nendro/3. Vajrasattva.mp3}}.
Ваджрасаттва в переводе на наш язык означает Алмазный Ум, по-тибетски
Дордже Семпа. Мантра Ваджрасаттвы:
\\
\\
ОМ БЕНЗА САТО САМАЯ МАНУ ПАЛАЯ БЕНЗА САТО ТЕНОПА ЧИТА ДРИТО
МЕБХАВА СУТО КАЙО МЕБХАВА СУПО КАЙО МЕБХАВА АНУ РАКТО МЕБХАВА
САРВА СИДДХИ МЕМ ПРАЯЦА САРВА КАРМА СУЦА МЕ ЧИТАМ ШРИ ЯМ КУРУ
ХУМ ХАХА ХАХА ХО БХАГАВАН САРВА ТАТХАГАТА БЕНЗА МА МЕ МУНЦА
БЕНЗИ БХАВА МАХА САМАЯ САТО А.
\\
\\
Мантра Ваджрасаттвы встречается как во многих откровениях (терма) так и в тантрах.

Подношение Мандалы или Всей Вселенной. Эта стихо-мантра давалась
Буддой Шакьямуни и записана в Гухьясамаджа Коренной
Тантре\footnote{\url{https://tonpa.guru/nendro/5. Mandala.mp3}}.
Исполняют сакьяпинские Монахи
\\
\\
ОМ БЕНЗА БХУМИ А ХУМ СИ ЙОНГ СУ ДАКПА ВАНГ ЧЕН СЕРЗИ
САСИ ОМ БЕНЗА РЕКЕ А ХУМ ЧИ ЧАК РИ КХОР ЮК ГИ КОРВЕ
У СУ ХУМ НРИ ГЬЯЛПО РИ РАБ ШАР ЛУ ПАПКО ЛО ДЗАМБУ
ЛИНГ НУБ БАЛАНГ ЧО ЧАНГ ДРА МИ НЬЕН ЛУ ДАНГ ЛУ ПАК
НГА ЯБ ДАНГ НГА ЯБ ШЕН ЙО ДЕН ДАНГ ЛАМ ЧОК ДРО ДРА
МИ НЬЕН ДАНГ ДРА МИ НЬЕН ГИ ДА РИНПОЧЕ РИВО ПАК САМ
ГИ ШИНГ ДО ДЖО И БА МА МО ПЕ ЛО ТОК КХОРЛО РИНПОЧЕ
НОРБУ РИНПОЧЕ ЦУНМО РИНПОЧЕ ЛОНПО РИНПОЧЕ ЛАНГПО
РИНПОЧЕ ТАЧОК РИНПОЧЕ МАКПОН РИНПОЧЕ ТЕР ЧЕНПО
БУМПА ГЕКПА МА ТРЕНГВА МА ЛУ МА ГАР МА МЕТОК МА
ДУКПО МА НАНГСЕЛ МА ДРИЧАП МА НЬИНГ ДАВА РИНПОЧЕ
ДУК ЧОК ЛЕ НАМ ПАР ГЬЯЛВЕ ГЬЯЛЦЕН ЛА ДАНГ МИЕ ПАЛДЖОР
ПУНСУМ ЦОКПА МА ЦАНГВА МЕПА
\\
\\
Детальные обяснения ищите в интеренете, например: "lamayeshe mandala offering".

После того как мы поднесли вселенную мы выполняем
Чод или Накопление Энергии Кусали. Отсечение
привязанности к собственному Я. Есть целая школа
буддизма Чодпа, которая ведет свое начало от Мачиг Лабдрон,
женжине, которая стала Буддой только благодаря одной
этой практике. Внесектарный
Чод\footnote{\url{https://tonpa.guru/nendro/6. Chod.mp3}}.
Транскрипция:
\\
\\
ОМ МАЧИГ МА ЛА СОЛВА ДЕБ КАРПО ОМ ГИ ДЖИНКИ ЛОБ
МАРПО А ГИ ДЖИНКИ ЛОБ НОНГПО ХУМ ГИ ДЖИНКИ ЛОБ
КУ СУМ ТУК КИ ДЖИН ЧЕН ПОП МА ЮМ ЧЕН ГО ПАНГ ТОБ ПАР ШОК
\\
\\
Только после того, как вы очистились в SPA у Самого Ваджрасаттвы,
выполнили отсечение привязанности к собственному Я у Мачинг Лабдрон
и поднесли Всю Вселенную в виде Мандалы, вам наконец разрешается
обратиться к Основателю Корпорации, который создал Службу
Безопасности и весь тибетский Буддизм вместе с тибетским
алфавитом, Великому Тантрическому Мастеру Падмасамбхаве.
Но есть ньюанс, перед тем как начать читать его мантру
нужно сначала обязательно прочитать семистрочный мантро-стих,
призывание Падмасабхавы\footnote{\url{https://tonpa.guru/nendro/7. Guru Rinpoche.mp3}}.
не меньше 21 (108 для упоротых) раз.
\\
\\
ОРДЖЕН ЮЛДЖИ НУБ ДЖАНГ ЦАМ ПЕМА ГЕСАР ДОНГ ПОЛА
ЯМЦЕН ЧОКИ НЁДРУБ НЕ ПЕМА ДЖУНЭ ШЕСУ ДРАГ КХОРДУ
КХАНДРО МАНПО КЁР КЕКИ ДЖЕСУ ДАГ ДРУБ ДЖИ ДЖИНДЖИ
ЛАБЧИР ШЕКСУ СОЛ ГУРУ ПАДМА СИДДХИ ХУМ.
\\
\\
Не многие среди религиозных фанатиков знают, что семистрочная
молитва — это терма Гуру Чованга (1212—1270). После призывания
Гуру Ринпоче разрешается начитывать мантру, которую вы должны
услышать лично от Учителя (без исключений):
\\
\\
ОМ А ХУМ БЕНЗА ГУРУ ПЕМА СИДДХИ ХУМ.
\\
\\
Для удобства все мантры есть тут\footnote{\url{https://axiosis.top/nendro/}}
вместе с бонусами: молитва Манджушри\footnote{\url{https://tonpa.guru/nendro/4. Manjushri.mp3}},
Молитва Устремления в Девачен\footnote{\url{https://tonpa.guru/nendro/A. Dechen Monlam.mp3}},
Песнь трапезы Джигме Лингпы\footnote{\url{https://tonpa.guru/nendro/9. Leymon Tendrel.mp3}},
Молитва Дзамбале и тантра Самантабхадры\footnote{\url{https://tonpa.guru/nendro/8. Kunzan Monlam.mp3}} для
освобождения путем слушания.

