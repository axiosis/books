\section{Мислення поточної миттєвості}

Что же такое Майндстрим или наше Мышление? Мышление – это центральный предмет изучения в буддийской науке. Мы, живые мыслящие существа, и есть мышление. Нет ничего, помимо мышления.  Все, о чем вы думаете, все, что вы видите, чувствуете, воспринимаете, все ваши эмоции, явления, концепции, теории, предположения, включая вас самих — это мышление. Если вы будете что-то долго и пристально рассматривать, вы обнаружите свое мышление. 

Существо. Обычно мышление называют существом. Если, что то и существует, то это мышление. Если вы сейчас смотрите, например, вокруг себя, то нет никакого другого мышления кроме того, что вы видите и чувствуете, нет какого-то большого ума, памяти. Память, мысли, привычки, эмоции, знание, сознание, осознание, подсознание, чувства, любовь, счастье и страдание – все это находится в вашем мышлении здесь и сейчас, всегда и везде. Все ваше мышление находится прямо в этом мгновении и называется Мышлением Мгновения.

Существование. Мышление по-тибетски называется Сэм, может выступать в роли глагола и существительного. Обычно, когда мы говорим о мышлении, мы говорим о самой тонкой реальной субстанции, которая формирует всю остальную нашу вселенную, поэтому с этой зрения буддизм можно называть материализмом. Выступая основным объектом исследования, сведя всё существующее и несуществующее к Мышлению Мгновения буддизм можно рассматривать как идеализм. Если мы в буддизме говорим, что что-то существует, то оно не может менять свое качество существования.

Открытость. Если атом — это волна, которая распространяется на всю видимую вселенную, то глупо было бы думать, что ваше мышление ограничено вашим черепом. Открытось и распахнусть вашего мышления имеет радиальную природу и как Солнце светит для всех существ во всех направлениях. Мышление не ограничено пространством, формы и образы возникают в самом мышлении и являются его частью, если мы будем пытаться пересчитывать характеристики мышления, то мы обнаружим, что все они соответствуют характеристикам пространства. Мышление — это Пространство.

Свобода выбора. В каждое конкретное мгновение, ваше мышление свободно по какому пути идти, быть на стороне Джидаев или Ситхов. Даже если вы находитесь в нижнем дне нижнего ада, вы можете мыслить позитивно, собственно именно с нижнего ада начал свое движение по пути Будда Шакьямуни. Примеры когда боги выбирают путь Ситхов приводить не нужно, достаточно попробовать припарковаться в час-пик. Каждое существо в Едином Квантовом Вакууме обладает зерном Просветленного Майндстрима. Окончательно ни на одном существе нельзя поставить крест.

Необузданность. На какие бы тренинги по остановке мышления вы не ходили, каких мастеров дзен медитации вы бы не выбирали — остановить и достичь полного покоя мышления невозможно ни в одном из четырех бардо. Около 80 видов медитации Шаматха предназначены для тренировки вашего мышления: с опорой, без опоры, опора на пустоту, на дыхание, неконцептуальная  медитация — все это дано вам, что бы понять одно: мышление как Интерстеллар Двигатель — остановить его невозможно. Поэтому медитируйте спокойно и не ждите знаков, ничего не должно произойти, просто сидите и считайте дыхания. По крайне мере первые пять лет. Майндфулнес Медитация, ага.

Карма. Главная характеристика мышления, которая является показателем, или уровнем мышления называется кармой. Карма — это интеграл по Мышлению. Понятие кармы было еще до Будды Шакьямуни. Карма – это сила мышления, привычки мышления, паттерны мышления. Чем больше вы повторяете какое-то мышление, тем большую вы накапливаете карму. Мышление повторенное три раза – это эмоция, если больше – это Колесо Медитации. Медитативные практики заключаются, в тренировке благого мышления, создании благих эмоций, которые ведут к переживаниям и к реализации.

Намерение. Чтобы накопленная карма созрела должно произойти следующее. Ваше двойственное мышление должно разделить Мгновение на Объект, Субъект и Действие. По завершения действия в этом состоянии вы должны ощутить удовольствие от этого процесса. Когда это произошло, поздравляем, вы только что отдалились от состояния Будды на одну йокто-Карму. Есть и хорошая новость, карма обычно созревает в болезни, смерть и всякие гадости, так что незаметить если вы нашкодничали не получится. Например вы держите чашку в руках и хотите выпить чаю, вы подносите объект чашку к своему субъектному рту и совершаете действие выпивания чая, получая от этого удовольствие в виде тепла, растекающегося по вашему телу. Просветленный Майндстрим не разделяет свое мышление на себя, чашку, подношения и удовольствие, другими словами Просветленный Майндстрим не накапливает кармы.

