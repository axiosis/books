\section{Служба Бузпеки}

Начну из далека, с одного из классических описаний Пути. В Буддизме часто Путь сравнивают с горой, на которую восходит идущий. Есть много троп, некоторые из них похожи на забетонированный серпантин с безопасным ограждением, так, что даже если вы превысите скорость, вы не сорветесь в обрыв. Есть тропы для нищебродов без автомобиля, есть трекинговые маршруты 1—4 категорий. Есть отвесные скалы для фронтальных и экстремальных восхождений, есть вертолеты, есть ракеты на базе двигателй Раптор, есть телепорт (мгновенная транспортировка на вершину). Каждый может выбрать себе уровень сложности и идти на вершину. Часто на пути возникают всякие прикючения. Например, вот группа скалолазов не послушала проводника и не перерезала веревку, поломался мотор у вертолета или телепорт телепортнул не на вершину, а в ядро земли. Разные случаи бывают. Например вот две бригады пересеклись на тропе и усторили друг другу Джихад. В таких случая работает Служба Безопасности или другими словами, защитники-спасатели на Пути Будды, охранители Учения.

Все программисты любят наркотики: кофе, травы покурить, вмазаться коксом, эндорфин поднять на тренировке со штангой, сахарку и глюкозы добавить (для интеллектуальной деятельности) — все это запрещенная или разрешенная наркота. Лидеры индустрии сидят на ЛСД и открыто это пишут в книгах и фильмах: Стив Джобс и другие. Если вы едете по автостраде на Вольво с одним лейн на скорости 30 км/ч с безопасным ограждением, то, да вы можете бухнуть для Острия Восприятия и прочищения каналов, но если вы на вертикальном маршруте, то даже повышения сахара может быть фатальным. Таких Путешественников снимает и задерживает Служба Безопасности. Если вы повышаете категорию восхождения, вы должны предупредить Службу Безопасности, что вы меняете маршрут, для вашей же безопасности! Максимально эффективным и максимально опасным является Телепорт (Дзогчен), тут для вас выдвигаются нечеловеческие требования, где восьмерка Оно Озаки является детской игрой.

Службу Безопасности нельзя тревожить и флудить заявками об изменении маршрута. Например, смена Учителя, смена линии прибежища, смена школы, смена практик, смена Сертификата. В Службу Безопасности нужно обращаться когда для этого назрела действительная необходимость. Представьте, что вы находитесь в супермаркете, вы же не бежите в Службу Безопасности Супермаркета сразу знакомить их со своей семьей и показывать какие у вас клевые четки и какой раритетный ваджр из Непала. Ведите себя вежливо и культурно, не вызывайте Службу Безопасности по пустякам, и не создавайте причин, чтобы Служба Безопасности пришла за вами.

У Безумного Монаха было два друга (Тарас и Игорь), которые буквально сошли с ума (ментальные и поведенческие расстройства группы F00-F99) из-за несоблюдения протокола взаимодействия со Службой Безопасности. Если вы мешаете практики Бон и Буддизма или Норбупа и Буддизма или Норбупа и Бон  или Буддизм и Шиваизм — это еще хуже чем практиковать под ЛСД. Уже не говорю об эклектике Христианства и Буддизма или остальной эзотерике. Я слышал один практик в Кунпенлинге заявлял, что Ригпа невозможен без Грибов, за что сгорел заживо в ретритном домике. Так работает Служба Безопасности. Представьте что на стрелу приезжают Две Службы Безопасности и все достают пистолеты, гранаты и начинается выяснение чья Служба Безопасности более кошерная и крутая. Не делайте так. Выберите линию и Учителя и практикуйте, установите маршрут, предупредите один раз Службу Безопасности зарегистрировав свое средство передвижения или купив официальный билет подкрепленный Сертификатом и отправляйтесь в Путь. Служба Безопасности всегда работает безупречно и будет рядом, чтобы вы не натворили делов.

