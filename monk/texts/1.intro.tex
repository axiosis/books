\section{Вступ}

Цель и этимология названия.
Основная цель этого стрима обеспечить поддержку
в вопросах Дхармы для Тиртиков. Что такое Дхарма?
Дхарма имеет множество значений на санскрите, от
общего абстрактного понятия “Феномен”, заканчивая
классическим “Учение Будды” и более фундаментальным
“Благое Мышление”. Кто такие Тиртики? Тиртики — это
все, кто придерживаются философских взглядов и систем
противоречащих учению Будды.

Целевая аудитория. Предполагается, что основная аудитория этого блога — это молодые люди находящиеся в зрелом возрасте, они обладают высшим образованием и успели разочароваться в Самсаре. Новое слово? Самсара — это то, что приносит нам страдания. Вам не хватило последней модели айфона в ближайшем Apple Store? Вам изменила девушка? У вас физическая боль? Вам задерживают зарплату? Поломалась зеркалка? Украли велосипед? Все это страдания. Дхарма — это лекарство от Страдания.

Снятие всякой ответственности и ограничения по использованию этой информации. Этот блог не является аффилированным ни к одной религиозной организации, учителю, монастырю или сектарному движению. Безумный Монах снимает с себя всякую ответственность за возможный ментальный, физический, моральный или любой другой вред нанесенный прямо или косвенно через чтение или само существование этого стрима.  Это не Дзен Буддизм, не Чань Буддизм, не Буддизм Тхеревады. Можно лишь сказать, что этот стрим связан с тибетской веткой буддизма. Но здесь не будет Дзогчена, Тантры, Мадхьямики, личных наставлений или ответов на вопросы. Это не руководство к действию — это всего лишь публичный блог Безумного Монаха. Вы используете эту информацию с львиной долей критического мышления и вы зрелый самостоятельный человек. Другими словами — это полнейший AS IS и PUBLIC DOMAIN.
