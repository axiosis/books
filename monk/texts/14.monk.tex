\section{Чим займається безумний монах?}

Расскажу вам немного о себе и своих проектах.
Почему Безумный Монах ничего не пишет вот уже
несколько месяцев, когда будут новые посты про
тибетский буддизм, и зачем вот это вот все. С
буддизмом я познакомился в 2010 году на ретрите
Намкая Норбу, с тех пор решил очень глубоко влезть в тему.
С самого начала мне были интересны первоисточники,
поэтому я быстро (за 2 ночи выучил алфавит, взял
переводчик \footnote{\url{http://www.thlib.org/reference/dictionaries/tibetan-dictionary/translate.php}}
и начал пытаться переводить тибетские тексты.
Потом взял пару уроков у Олега Филлипова, потом
вдохновляющие наставления у Сергея Дудко, и еще
некоторых мелких наставлений у менее известных лам.
Занятие тибетскими переводами --- вещь неблагодарная.
Во-первых все тибетские переводчики сруться между
собой --- это очень ужасно. Вставлять букву Х или
не вставлять, писать Г или К и тому подобное.
В прежние времена это вполне могло стать причиной
войны или какого-то отравления или просто пером
порезать могли. Ну или как случилось во второй
период становления буддизма в Тибете --- раскол
на школы старых переводов (ньингма) и новых
переводов (сарма). Понимая и ощущая глубокую
карму прошлых веков я вознамерился запустить
следующий проект, который был запланирован еще в 2010 году.

Во-первых — опен соурс и абсолютная бесплатность.
Бесплатность для читателей, лам, схоластов, академиков,
и обычных людей. Каждый человек должен иметь возможность
получить текст в UTF-8. Возможны платные PDF, в виде
бумажных платных книг, но цифра должна быть бесплатной.
Переводчики будут получать деньги, дизайнеры будут
получать деньги, но люди не должны платить ничего.
Пока единственный самовластный спонсор --- Безумный Монах.
Если кто-то хочет вкинуться баблом --- милости просим,
мне многие учителя говорили что нету ничего святее
святых текстов и нет ничего более полезного для личной
кармы, чем заниматься святыми текстами.

Во-вторых — это открытая платформа для коллаборации разных школ, линий передачи, их переводчиков и других случайных людей: корректоров, дизайнеров. У меня есть опыт в управлении ИТ командами и я успешно его применяю для овеществления этого проекта. Мы покупаем переводы у переводчиков с открытой лицензией и публикуем их у себя на сайте.

В-третих — это касается не только текстов, но и любой цифровой дхарма информации — изображений, символов, тханок мастеров, дидактических материалов, аудио и видео материалов. Я так часто я вижу книги практик напечатанные криворукими дебилами ваджрными братьями, что хочется незамедлительно вкинуть еще пару тысяч долларов в систему верстки и систему управления публикацией, потому что кровь просто сочится из глаз настолько все плохо. Поэтому все материалы должны быть бесплатные для всех — я хочу предоставить максимальные условия для воровства и бесконтрольного распространения тибетских текстов. Все картинки как все тексты – абсолютно бесплатны и легальны для копирования, распространения, зарабатывания на этом денег (ISC, BSD, MIT).

Как же так спросите вы? А как же самаи и тантрические обязательства? Во первых все возможно если есть хорошая крыша у хорошего Ринпоче. Например TBRC – это пример такого глобального проекта, который ведут люди у которых я вдохновился ихнему намерению по сохранению и упорядочиванию тибетского текстового наследия. Для тех кто не знает TBRC — это офис в NYC который занимается каталогизацией ВСЕХ тибетских текстов всех школ, направлений и линий. Тупо все. Все бесплатно для всех, но с ограничениями по распространению, потому как много терма настолько свежие, что еще действуют современные законы по авторскому праву на них. Что сказать TBRC крышуют крутые ринпоче, я пробовал просить крышу у некоторых Ринпоче, но мало кто хочет инвестировать свое время и деньги в такой проект, люди даже на практику не могут прийти в дхарма-центр, куда им заниматься переводами и упорядочиванием тантр, ах оставьте.

Есть некоторые похожие проекты, тибетский OCR,
его натравить на базу данных всех отсканированных
тибетских текстов на TBRC, многие тексты я ворую
у них, но у них есть ошибки, поэтому все равно
надо сверять каждый текст с оригинальными источниками.
Я решил что я сам буду в ответе за то, что делаю,
поэтому если вы хотите пожаловаться на меня Экаджати
или Циу Марпо --- пожалуйста, дело ваше.

Что сделано на сегодняшний момент\footnote{\url{https://longchenpa.guru}}.
Этот сайт посвящен в первую очередь линии Лонгчен Нингтик,
но также ведется работа по упорядочиванию и оцифровке других
нингмапинских линий \footnote{\url{https://longchenpa.guru/gter.ma/index.htm}}.
Тханки учителей \footnote{\url{https://longchenpa.guru/dpe.mdzod/thang.ga/dpe.cha/index.htm}},
8 символов: \footnote{\url{https://longchenpa.guru/dpe.mdzod/bkra.shis.rtags.bryad/index.htm}}.
Нингмапинское зеркало свободно распространяемое по
запросу: \footnote{\url{https://longchenpa.guru/dpe.mdzod/dpe.tshogs/index.htm}},
утилиты для работы с Wylie и каталогами \footnote{\url{https://github.com/longchenpa/wylie}}.
Открытые позиции (вакансии) в Лонгчен Нингтик Украина
можно посмотреть здесь \footnote{\url{https://github.com/longchenpa/longchenpa.guru}}.

Знаю, меня многие читают на русском, поэтому есть
предыдущая версия на русском: \footnote{\url{http://nyingma.longchenpa.guru}},
сейчас я сосредоточился исключительно на украинском, так
как до сих пор нет ни одного моего перевода на родном языке.
Позор мне и бесславие!

Наслаждайтесь цифровой дхармой от Безумного Монаха!
