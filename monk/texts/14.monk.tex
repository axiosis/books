\section{Чим займається Божевільний монах?}

Розповім вам трохи про себе та свої проекти.
Чому Божевільний Монах нічого не пише ось уже
кілька місяців, коли будуть нові пости про
тибетський буддизм, і навіщо це все. З
буддизмом я познайомився у 2010 році на ретріті
Натякаючи Норбу, з того часу вирішив дуже глибоко влізти в тему.
З самого початку мені були цікаві першоджерела,
тому я швидко (за 2 ночі вивчив алфавіт, взяв
перекладач \footnote{\url{http://www.thlib.org/reference/dictionaries/tibetan-dictionary/translate.php}}
і почав намагатися перекладати тексти Тибету.
Потім узяв пару уроків у Олега Філіпова, потім
надихаючі настанови у Сергія Дудка, та ще
деяких дрібних настанов у менш відомих лам.
Заняття перекладами Тибету --- річ невдячна.
По-перше, всі тибетські перекладачі зрушаться між.
собою --- це дуже жахливо. Вставляти літеру Х або
не вставляти, писати Г або К тощо.
За старих часів це цілком могло стати причиною
війни чи якогось отруєння чи просто пером
порізати могли. Ну чи як трапилося в другій
період становлення буддизму в Тибеті --- розкол
на школи старих перекладів (ньінгма) та нових
перекладів (сарма). Розуміючи та відчуваючи глибоку
карму минулих століть я мав намір запустити
наступний проект, який було заплановано ще 2010 року.

По-перше — опен соурс та абсолютна безкоштовність.
Безкоштовність для читачів, лам, схоластів, академіків,
та звичайних людей. Кожна людина повинна мати можливість
отримати текст у UTF-8. Можливі платні PDF у вигляді
паперових платних книг, але цифра має бути безкоштовною.
Перекладачі отримуватимуть гроші, дизайнери будуть
отримувати гроші, але люди не повинні платити нічого.
Поки що єдиний самовладний спонсор --- Божевільний Монах.
Якщо хтось хоче вкинутися баблом --- ласкаво просимо,
мені багато вчителів говорили що немає нічого святішого
святих текстів і немає нічого кориснішого для особистої
карми, чим займатися святими текстами.

По-друге, це відкрита платформа для колаборації різних шкіл,
ліній передачі, їх перекладачів та інших випадкових людей:
коректорів, дизайнерів. Я маю досвід в управлінні ІТ командами
і я успішно його застосовую для уречевлення цього проекту.
Ми купуємо переклади у перекладачів з відкритою ліцензією
та публікуємо їх у себе на сайті.

По-третє, це стосується не лише текстів, а й будь-якої цифрової
дхарми інформації --- зображень, символів, тханок майстрів,
дидактичних матеріалів, аудіо та відео матеріалів. Я так часто
бачу книги практик надруковані криворукими дебілами ваджрними
братами, що хочеться негайно вкинути ще кілька тисяч доларів
у систему верстки і систему управління публікацією, тому що
кров просто сочиться з очей настільки все погано. Тому всі
матеріали мають бути безкоштовними для всіх --- я хочу надати
максимальні умови для крадіжки та безконтрольного розповсюдження
текстів Тибету. Всі картинки як усі тексти --- абсолютно
безкоштовні та легальні для копіювання, розповсюдження,
заробляння на цьому грошей (ISC, BSD, MIT).

Як же ви запитаєте? А як же самі та тантричні зобов'язання?
По-перше, все можливо, якщо є хороший дах у хорошого Рінпоче.
Наприклад TBRC --- це приклад такого глобального проекту, який
ведуть люди, у яких я надихнувся їхньому наміру щодо збереження
та впорядкування тибетської текстової спадщини. Для тих хто не
знає TBRC --- це офіс в NYC який займається каталогізацією ВСІХ
текстів Тибету всіх шкіл, напрямків і ліній. Тупо все. Все
безкоштовно для всіх, але з обмеженнями поширення, тому що
багато терма настільки свіжі, що ще діють сучасні закони з
авторського права на них. Що сказати TBRC кришують круті рінпоче,
я пробував просити дах у деяких Рінпоче, але мало хто хоче інвестувати
свій час і гроші в такий проект, люди навіть на практику не можуть
прийти в дхарма-центр, куди їм займатися перекладами та впорядкуванням тантр, ах залиште.

Є деякі схожі проекти, тибетський OCR,
його нацькувати на базу даних усіх відсканованих
тибетських текстів на TBRC, багато текстів я краду
у них, але у них є помилки, тому все одно
треба звіряти кожен текст із оригінальними джерелами.
Я вирішив що я сам буду відповідати за те, що роблю,
тому якщо ви хочете поскаржитися на мене Екаджаті
або Ціу Марпо --- будь ласка, справа ваша.

Що зроблено на сьогоднішній момент \footnote{\url{https://longchenpa.guru}}.
Цей сайт присвячений в першу чергу лінії Лонгчен Нінгтік,
але також ведеться робота з упорядкування та оцифрування інших
нінгмапінських ліній \footnote{\url{https://longchenpa.guru/gter.ma/index.htm}}.
Тханки вчителів \footnote{\url{https://longchenpa.guru/dpe.mdzod/thang.ga/dpe.cha/index.htm}},
8 символів: \footnote{\url{https://longchenpa.guru/dpe.mdzod/bkra.shis.rtags.bryad/index.htm}}.
Нінгмапінське дзеркало вільно розповсюджується по
запиту: \footnote{\url{https://longchenpa.guru/dpe.mdzod/dpe.tshogs/index.htm}},
утиліти для роботи з Wylie та каталогами \footnote{\url{https://github.com/longchenpa/wylie}}.
Відкриті позиції (вакансії) в Лонгчен Нінгтік Україна
можна подивитися тут \footnote{\url{https://github.com/longchenpa/longchenpa.guru}}.

Насолоджуйтесь цифровою дхармою від Божевільного Монаха!