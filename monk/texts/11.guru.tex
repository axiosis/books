\section{Гуру та ламаїзм в тибетському буддизмі}

Что такое ламаизм и кто такой Гуру в тибетском буддизме. В нашем безумном и жестоком мире, где Кока-Кола является кровью Исуса Христа, а Сникерс — его телом, где отношения к Дхарме потребительское и вы можете выбрать дхарма-бутик на любой вкус и желание, хочу рассказать как вам нужно относится к ламам и ламаизму, понятию, полностью непонимаемому западными послушниками.

Многие думают, что выбор учителя — это как выбор одежды, нравится или не нравится. Оно и логично, как можно выбрать то, что тебе не нравится. В буддизме говорят, что не ученик выбирает Учителя, а Учитель эманирует учениками и дает им Учение. По такому принципу построены все тантры (сказки), основной лейд-мотив которых это диалог Естественного Состояния (Учителя) со своими проекциями (Бодхисаттвами, Просветленными Майндстрмами).

Немного истории про Тибет. Задайте себе вопрос откуда в Тибете золото и драгоценные камни, из которых созданы статуи Будды и священные реликвии. В Тибете же кроме железной руды, камней и яков ничего нет. Тибетцы издревле были кочевниками, которым повезло, что рядом с ними проходил Шелковый Путь, и поэтому они, да, Грабили Корованы. Таким образом большинство тибетцев прошлого — это бандиты, которые несли награбленное своим святым, которые были уважаемыми людьми. Первый Додрубчен — ученик Джигме Лингпы, например был учителем одной такой тибетской бригады и часто пересекался с другими кочевниками, конвертируя их в последователей Учения. Бандитам не нужно объяснять о бренности жизни и давать лоджонг, они естественным образом понимают, что их могут замочить из лука или от точности их мыслей и базара на разборках буквально зависит их жизнь. Поэтому их сознание всегда находится на острие восприятия. Много святых, и даже тулку в Тибете являются хорошими охотниками. Это лирическое отступление должно вам дать некоторый намек на то, как выглядит настоящий лама. Европейцы обычно сами просят учения на свой выбор: хочу Дзогчен, хочу учения про Бардо, хочу Отходную Молитву, и.д., если бы вы в Тибете прошлого выразили такое желание могли бы и получить посохом по голове. Многие и выхвачивали, отобрать жизнь у безнадежного послушника тоже может быть кик-оффом на пути.

Наше тело и сознание в известном вам мире не имеет границ развития и основаная задача Учителя — это расширить горизонты вашего восприятия. Если вы задрот, Учитель вам должен показать как быть типочком; если вы Казанова, учитель должен вам показать как быть Монахом; если вы пандит, учитель более усердно дает вам йогу. Понятно, что в такой концепции на комфорт рассчитывать не приходится. Более того, настоящий Учитель должен быть Виртуозным Провокатором с Испепеляющим Взглядом, который нажимает и бьет в ваши самые душевные болевые точки, которые для вашего сознания выглядят как табу. А теперь задайте себе вопрос: как тибетский лама, приглашенный вами на очередной европейский ретрит сможет это сделать не понимая вашей культуры и вашего контекста? У него просто нет инструментов для провокаций. Все ламы которые приезжают и дают вам Дзогчен и другие учения, произнесение которых вслух уже считается обесцениванием, выбираются обычно по тому, насколько они хорошо владеют колокольчиком, насколько они красиво крутят мудры и насколько звонкий их голос. Все эти гастроли направлены только на сбор денег на поддержку монастырей и развод европейских лохов. Лучший ваш учитель — это тот, кто смог разорвать вашу душу в клочья, и этот человек очевидно должен принадлежать вашей культуре. Ищите ламу-соплеменника, который знает ваш вокабулар и ваши мемасики, иначе вы просто заплатите за очередное представление с колокольчиком и ваджрой (есть еще ламские танцы).

Много информации можно найти в интернете про женский харасмент от каких-то там лам. Надо понимать, что в Тибете отказ ламам — это дикость, сравнимая с тем, как если бы вы запарковались на газон Верховного Совета или Президента. Это все последствия потребительской культуры. Не знаю, читают ли меня женщины, но те немногие из них, которые дочитали до этого абзаца, должны убегать от настоящих тибетских лам подбирая юбки и отламывая каблуки.

Конечно, пока вы находитесь на стадии обучения, с этим всем вы столкнетесь не сразу, но моя задача предупредить вас, во что вы ввязываетесь. Путь Будды — это не курс Компьютер Сайнс в МИТ, это не христианское затворничество, это очень экстремальный путь, где риск выше чем у белок-бейсджамперов. На сегодя страшилок хватит, попугаю вас еще в следующем посте!

