\noindent УДК 13\\
\noindent УДК 821.161.2

\section*{Божевільний монах \dharmachakra}

\begin{tabular}{ll}
& Автор: Чо Дха (19801126-02596)\\
& Присвячується: Марії Бєліковій (19940927-11601)
\end{tabular}

\section*{Про Автора}
Чо Дха (Максим Сохацький) --- буддистський піп-капелан лінії передачі
Лонгчен Нінгтік тибетського буддизму школи Нінгма (38778275).
\\
\\
\\
\\
\\
\\
Постійне посилання твору: https://axiosis.top/monk/ \\
Видавець: Лонгчен Нінгтік Україна, ЄДРПОУ: 38778275 \\
Юридична адреса: 04071, м. Київ, вул. Костянтинівська 20 \\
Сайт релігійної організації: https://longchenpa.guru/ \\
Лінія передачі: Школа Нінгма тибетського буддизму \\
\\
\\
{\bf ISBN: 978-617-8027-08-7 \hspace{2em}} \\
\\
\\
\indent В цьому тексті я зібрав все, що колись мене
питали мої друзі тиртики про буддизм, а я встидався відповісти.
Неканонічне і неформальне пояснення канонічності та формальності Старої Школи.
\\
\\
\begin{tabular}{ll}
\textcopyright{} 2017---2024 & Максим Сохацький
\end{tabular}
