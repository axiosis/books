\section{Чотири Істини Благородних}

Якщо вам дають буддійський прихисток без Чотирьох Істин Благородних,
як це роблять у Боні та Норбупа — це швидше за все фейковий або
небуддійський прихисток. Тому щоб зняти з себе відповідальність
та можливі нападки за здійснення неканонічної діяльності,
я публікую текст сутри відповідно до Канону.

Будда Шак'ямуні довго не хотів вчити Просвітленому Майндстріму,
але погодився лише в Варанасі і тільки коли йому піднесли Раковину,
Колесо і т.д., взагалі завалили безглуздими подарунками,
але від щирого серця (це головне). Без підношення та подарунків,
Договір Прийняття Притулку вважається недійсним. Це на
першому рівні. На третьому рівні ви взагалі повинні розлучитися
буквально з усім своїм майном піднісши його повністю ламі,
а також (адже майно --- безглуздий подарунок) піднести
головне --- свої Тіло, Мову та Мислення. Тобто піднести всього
себе без винятку та застережень. На цьому кроці я попереджу, що
ті небагато з вас хто ще хотів стати буддистом, нарешті відмовляться
від цієї ідеї і цей стрім транслюватиметься в порожнечу, як і
належить справжнім нєтлєночкам.

До цього абзацу дійдуть справді сміливі одиниці. Розкажу вам
коротко про Чотири Істини. Вони описують ситуацію порівняну
з прийомом у лікаря: 1) Ви --- істота, яка захворіла Самсарою (тобто.
страждаєте зі змінним успіхом); 2) Існує причина цієї
хвороби --- це ваше неблаге мислення; 3) Існують ліки
від цієї хвороби --- Дхарма; 4) Будда --- лікар, який виписує
вам ліки і курс лікування, що перший прийняв його і вилікувавшись
випробувавши на собі. Важливо зауважити, що на відміну від лікарів,
які зазвичай не хворіють на всі хвороби, які лікують,
Будда --- сам був хворий і перший прийняв ліки випробувавши
його на собі. Ці ліки — вчення про Благе Мислення --- Дхарму.

Кожен лама вчить Чотирьом Істинам по-своєму, тому не лякайтеся
коли зустрінете дико різні коментарі на канон. Саму
красиву бінарну модель я чув одного разу таку. Є дихотомія
по двох осях: Благе або Неблаге мислення і Ваше мислення або
мислення інших. 1) Моє страждання --- результат мого недоброго
мислення; 2) Моє недобре мислення --- єдина причина
мого страждання та умова страждання інших істот; 3)
Моє Щастя --- результат мого Благого мислення; 4) Моє
Благе мислення --- це єдина причина мого Щастя
та умова щастя інших істот. Таким чином тут
Чотири Істини Благородних записані у вигляді теореми-визначення
(тоді і тільки тоді), фактично встановлюючи синонімічний
зв'язок між стражданнями та їх причинами, а також показуючи
відносність вашого буття по відношенню до інших істот.
Якщо ви страждаєте чи щасливі --- ви транзитивно дієте
на всі Майндстріми у всіх Всесвітах, тому що всі істоти
перебувають у Єдиному Квантовому Вакуумі та їх поля знаходяться
неперервно у суперпозиції. Важливо навчитися керувати
своїм мисленням, адже всі ми знаходимося в одному місці та в один
час, завжди, не покидаючи цей простір ні на мить.

Вчення про Чотири Істини Благородних є Першим Поворотом
Колеса Дхарми (всього чотири) і зазвичай дається перед формальним
Прийняття Притулку. При цьому ритуал ухвалення притулку
супроводжується начиткою мантр. Зазвичай усі халявлять, але читачі
цього стріму повинні начитати щонайменше 100 000 разів. На вибір даю
відразу дві мантри притулку та тибетський мантро-вірш.
\newpage
Перша для лінивих:
\\
\\
НАМО БУДДЯ НАМО ДХАРМА НАМО САНГХАЯ.
\\
\\
Друга для релігійних фанатиків середньої упоротості:
\\
\\
БУДДХАМ САРАНАМ ГАЧЧАМИ ДХАРМАМ САРАНАМ ГАЧЧАМИ САНГХАМ САРАНАМ ГАЧЧАМИ
\\
\\
І для релігійних фанатиків, які хочуть знати все,
тибетською мовою позасектарний літургійний притулок:
\\
\\
САНГ'Є ЧОДАНГ ЦОКІ ЧОКНАМ ЛА ЧАНГЧУБ БАРДО ДАКНІ
К'ЯБ СУ ЧИ ДАГИ ДЖИНСОК ГІ ПЕСО НАМ КИ ДРОЛА ПАНЧИР
САНГ'Є ДРУПАР ШОК.