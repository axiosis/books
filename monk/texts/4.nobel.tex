\section{Чотири Істини Благородних}

Если вам дают буддийское прибежище без Четырех Истин Благородных,
как это делают в Боне и Норбупа — это скорее всего фейковое или
небуддийское прибежище. Поэтому, чтобы снять с себя ответственность
и возможные нападки за совершение неканонической деятельности,
я публикую текст сутры в соответствии с Каноном: http://5ht.co/dharma/chokor.duchen.htm

Будда Шакьямуни долго не хотел учить Просветленному Майндстриму,
но согласился лишь в Варанаси и толко когда ему поднесли Раковину,
Колесо и т.д., вообщем завалили бессмысленными подарками,
но от чистого сердца (это главное). Без подношения и подарков,
контракт Принятия Прибежища считается недействительным. Это на
первом уровне. На третьем уровне вы вообще должны расстаться
буквально со всем своим имуществом поднеся его полностью ламе,
а также (ведь имущество — бессмысленный подарок) поднести
главное — свои Тело, Речь и Мышление. Т.е. поднести всего
себя без исключения и оговорок. На этом шаге я предикчу, что
те немногие из вас кто еще хотел стать буддистом, наконец откажутся
от этой идеи и этот стрим будет транслироваться в пустоту, как и
полагается для настоящих нетленочек.

До этого абзаца дойдут воистину смелые единицы. Расскажу вам
вкратце о Четырех Истинах. Они описывают ситуацию сравнимую
с приемом у врача: 1) Вы — существо которое заболело Самсарой (т.е.
страдаете с переменным успехом); 2) Существует причина этой
болезни — это ваше неблагое мышление; 3) Существует лекарство
от этой болезни — Дхарма; 4) Будда — врач, который выписывает
вам лекарство и курс лечения, первый принявший его и излечившись
испробовав на себе. Важно заметить, что в отличие от врачей,
которые обычно не болеют всеми болезнями, которые лечат,
Будда — сам был больной и первый принял лекарство испробовав
его на себе. Это лекарство — учение о Благом Мышлении — Дхарме. 

Каждый лама учит Четырем Истинам по-своему, поэтому не пугайтесь
когда встретите дико отличающиеся комментарии на канон. Самую
красивую бинарную модель я слышал однажды такую. Есть дихотомия
по двум осям: Благое или Неблагое мышление и Ваше мышление или
мышление Других. 1) Мое страдание — результат моего неблагого
мышления; 2) Мое неблагое мышление — единственная причина
моего страдания и условие страдания других существ; 3)
Мое Счастье — результат моего Благого мышления; 4) Мое
Благое мышление — это единственная причина моего Счастья
и условие счастья других существ. Таким образом здесь
Четыре Истины Благородных записаны в виде теоремы-определения
(Тогда и Только Тогда), фактически устанавливая синонимическую
связь между страданиями и их причинами, а также показывая
относительность вашего бытия по отношению к другим существам.
Если вы страдаете или счастливы — вы транзитивно действуете
на Все Майндстримы во всех Вселенных, так как все существа
пребывают в Едином Квантовом Вакууме и их поля находятся
непрерывно в суперпозиции. Поэтому так важно научится управлять
своим мышлением, ведь все мы находимся в одном месте и в одно
время, всегда, не покидая это пространство ни на мгновение.

Учение о Четырех Истинах Благородных является Первым Поворотом
Колеса Дхармы (всего четыре) и обычно дается перед формальным
Принятием Прибежища. При этом ритуал принятия прибежища
сопровождается начиткой мантр. Обычно все халявят, но читатели
этого стрима должны начитать минимум 1000 раз. На выбор даю
сразу две мантры прибежища и тибетский мантро-стих.

Первая для ленивых:
\\
\\
НАМО БУДДАЯ НАМО ДХАРМАЯ НАМО САНГХАЯ.
\\
\\
Вторая для религиозных фанатиков средней уопорости:
\\
\\
БУДДХАМ САРАНАМ ГАЧЧАМИ ДХАРМАМ САРАНАМ ГАЧЧАМИ САНГХАМ САРАНАМ ГАЧЧАМИ
\\
\\
И для религиозных фанатиков, которые хотят знать всё,
на тибетском языке внесектарное литургическое прибежище:
\\
\\
САНГЬЕ ЧОДАНГ ЦОКИ ЧОКНАМ ЛА ЧАНГЧУБ БАРДО ДАКНИ
КЬЯБ СУ ЧИ ДАГИ ДЖИНСОК ГИ ПЕСО НАМ КИ ДРОЛА ПАНЧИР
САНГЬЕ ДРУПАР ШОК.
