\section{Свідомість, Усвідомлення, Самоусвідомлення}

Трошки метафізики для любителів езотерики та Адептів Таємного Знання.
Сьогодні ми розберемо детально п'яту скандху, двигун нашого мислення.
Почну з далекого, з Дзогчена. Існує чотири види Основи:
\\
\\
1. Основа первісного. Поза Самсарами та Нірванами. Дзогчен.\\
2. Основа Визволення. Основа Нірвани. Махамудра.\\
3. Основа Спільного. Основа Самсари та Нірвани. Махамудра.\\
4. Основа Ілюзій. Пунсова-віджняна. Кунжі Намше. Основа Самсари.\\
\\
\\
В Абхідхармі Пітак, каноні Палі, текстах Абхідхарма Коша та працях Асанги,
коментарів на це Шантракшити та Джамгон Міпама Рінпоче йдеться
про Алая-віджняну. Тобто найбільш забрудненій основі нашого мислення.
У цій Основі Ілюзії або Свідомості Всеоснови зберігається вся карма
і всі феномени, накопичені вами за мільярди життів. Вона забезпечує
потенційне дозрівання насіння Оточення, Об'єктів та Тіла. Її також називають
свідомістю дозрівачем чи свідомістю одержувачем. Це контейнер для
першої скандхи --- Форми. Тут знаходяться всі ваші форми: Причинні,
Зовнішні (Об'єкти сприйняття) та Внутрішні (Здібності Тіла, Суб'єкти, Нейронна мережа).
Те, що ми сприймаємо прямо зараз називається об'єкти, те, що ми не сприймаємо,
але воно є --- це оточення і наше Тіло --- це наше єство. Реінкарнує саме цей контейнер,
отримуючи в кожному бардо нове тіло. Тіло, Оточення та Об'єкти ---
це все феномени, дозріла Карма яка знаходиться в 8-му свідомості Основі Ілюзій, всі вони знаходяться тут.

Далі ми маємо 5 неконцептуальних свідомостей, які виникають як тільки
феномени зустрічаються зі здібностями тіла, які є системами рецепторів
нейронної мережі. Породжувач форми (потік фотонів) зустрічається зі здатністю
ока (Оком та ділянкою мережі розпізнавання об'єктів) і народжується неконцептуальна
свідомість ока. Породжувачі можуть бути коливання пронстрансва (хвилі) або
молекули (запах) або структура/фактура матерії (смак) --- всі вони,
зустрічаючись з частинами тіла, запускають неконцептуальні п'ять свідомостей.

Є ще особливе 6-те неконцептуальне мислення --- це свідомість, що виникає під
час передачі ментальних образів і смислів, тобто свідомість, що виникає при руху
прани по нейронній мережі або як вона називається в буддизмі Тибету ---
здатність розуму. Тобто наш мозок --- це навіть не свідомість ---
це форма, здатність розуму, яка є самоумовою створення свідомості.

І є особлива концептуальна 7 свідомість. Це свідомість що вибирає думає і
накопичує карму, тобто переносить досвід сприйняття у 8-му свідомість контейнер карми.
Коли називають істоту «той хто вибирає» мають на увазі цю свідомість.
Ця свідомість структурує інформацію та записує її в голографічний носій 8-ї
Свідомості Основи. Це концептуальне свідомість поділу сприйняття три сфери ---
об'єкт суб'єкт і дію. Воно також є джерелом створення міфічного "Я" як ототожнення
себе з тілом та відкидання феноменів та оточення кажучи постійно "це не Я". Саме
цим поділом ми зайняті левову частку свого часу. На першому Бхумі 7-ма свідомість
вже має бути повністю відсутньою чи бути мінімальною. З 7-го бхумі ця свідомість
геть-чисто відсутня і далі шлях бодхісатв полягає в спустошенні 8-ї свідомості
та спалюванні карми та феноменів, щоб дістатися до нижчих прошарків Основи,
аж до Дхармакаї --- Абсолютної Примордіальної Основи Початку.

Всі ці порівняння з формами наших систем сприйняття, рецепторів, звичайно ж,
діють тільки в цьому бардо постійного мислення. У сновидіннях п'ять свідомостей
не працюють але там породжувачами є інші форми точно так саме схожі на такі як тут,
які мають такі ж обриси, а орган сприйняття як і раніше пов'язаний з тілом сновидіння.
У кожному бардо є свідомість і кожному бардо є форма і орган зустріч яких
народжує це свідомість. Вся карма всіх бардо перебуває у свідомості всеоснови.

Та ще й вам бонус, який у лам вартує як квартира. Кожна свідомість має самоусвідомлення.
Якщо істота має п'яту скандху свідомість, то вона може звільнитися і стати Буддою.
Таке самоусвідомлення кожної свідомості в Тибеті називається Рігпа. Це не
Дзогчен Рігпа, просто будь-яке самоусвідомлення називається рігпа.
Є 5 рівнів рігпа від самоусвідомлення (awareness) і до Дзогчен Рігпа ---
Самоусвідомлення Основи Початкового за межами Самсари та Нірвани.
