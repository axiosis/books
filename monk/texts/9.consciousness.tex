\section{Свідомість, Усвідомлення, Самоусвідомлення}

Немного метафизики для любителей эзотерики и Адептов Тайного Знания. Сегодня мы разберем детально пятую скандху, мотор нашего мышления. Начну из далека, из Дзогчена. Существует Четыре вида Основы:

1. Основа Изначального. Вне Самсары и Нирваны. Дзогчен.
2. Основа Освобождения. Основа Нирваны. Махамудра. 
3. Основа Общего. Основа Самсары и Нирваны. Махамудра.
4. Основа Иллюзий. Алая-виджняна. Кунжи Намше. Основа Самсары. 

В Абхидхарме Питак, каноне Пали, текстах Абхидхарма Коша и трудах Асанги, комментариев на это Шантракшиты и Джамгон Мипама Ринпоче идет речь об Алая-виджняне. Т.е. наиболее загрязненной основе нашего мышления. В этой Основе Иллюзии или Сознании Всеосновы хранится вся карма и все феномены накопленые вами за миллиарды жизней. Она обеспечивает потенциальное созревание семян Окружения, Объектов и Тела. Ее также называют сознанием созревателем или сознанием получателем. Это контейнер для первой скандхи — Формы. Здесь пребывают все ваши формы: Причинные, Внешние (Объекты восприятия) и Внутренние (Способности Тела, Субъекты, Нейронная сеть). То что мы воспринимаем прямо сейчас называется объекты, то, что мы не воспринимает но оно есть — это окружение и наше Тело — это наше тело. Реинкарнирует именно этот контейнер приобретая в каждом бардо новое тело. Тело, Окружение и Объекты — это все феномены, созревшая Карма которая находится в 8-м сознании Основе Иллюзий, все они находятся здесь.

Дальше мы имеем 5 неконцептуальных сознаний, которые возникают как только феномены встречаются со способностями тела, котороые являются системами рецепторов нейронной сети. Порождатель формы (поток фотонов) встречается со способностью глаза (Глазом и участком сети распознавания объектов) и рождается неконцептуальное сознание глаза. Порождатели могут быть колебания пронстрансва (волны) или молекулы (запах) или структура/фактура материи (вкус) — все они встречаясь с частями тела запускают неконцептуальные пять сознаний. 

Есть еще особое 6-е неконцептуальное сознание — это сознание которое возникает при передаче ментальных образов и смыслов, т.е. сознание которое возникает при движение праны по нейронной сети или как она называется в тибетском буддизме — способность ума. Т.е. наш мозг — это даже не сознание — это форма, способность ума которая является самоусловием создания сознания.

И есть особое концептуальное 7-е сознание. Это сознание которое выбирает думает и накапливает карму, т.е. переносит опыт восприятия в 8-е сознание контейнер кармы. Когда называют существо "тот кто выбирает" имеют ввиду это сознание. Это сознание структурирует информацию и записывает его в голографический носитель 8-го Сознания Основы. Это концептуальное сознание разделения воспрриятия на три сферы — объект субъект и действие. Оно также является источником создания мифического "Я" как отождествления себя с телом и отбрасывании феноменов и окружения говоря постоянно "это не Я". Именно этим разделением мы заняты львиную долю своего времени. На первом Бхуми 7-е сознание уже должно польность отсувсвовать или быть минимальным. К 7-му бхуми это сознание напрочь отсувствует и дальше путь бодхисатв заключается в опустощении 8-го сознания и расворении кармы и феноменов, что бы добраться до более низких слоев Основы, вплоть до Дхармакаи — Абсолютной Основы Изначального.

Все эти сравнения с формами наших систем восприятия, рецепторов конечно же действуют только в этом бардо постоянного мышления. В сновидениях пять сознаний не работают но там порождателями являются другие формы в точности похожие на такие как здесь, которые имеют такие же очертания, а орган восприятия по прежнему связан с телом сновидения. В каждом бардо есть сознание и в каждом бардо есть форма и орган встреча которых рождает это сознание. Вся карма всех бардо находится в сознании всеосновы.

Да еще вот вам бонус, который у лам стоит как квартира. Каждое сознание обладает самоосознанием. Если у существа есть пятая скандха сознание, то оно может освободиться и стать Буддой. Такое самоосознание каждого сознания в тибете называется ригпа. Это не Дзогчен Ригпа, просто любое самоосознание называется ригпа. Есть 5 уровней ригпа от самоосознания (awareness) и до Дзогчен Ригпа — Самоозознания Основы Изначального за пределами Самсары и Нирваны.

