\section{Медитація}

В предыдущем выпуске мы немного удовлетворили нашего Внутреннего Пандита (ученого), а сегодня мы попытаемся удовлетворить нашего Внутреннего Йогина (спортсмена). Теперь немного будет практики. Поскольку целевая аудитория этого стрима — это программисты, будем объяснять так, чтобы было понятно в первую очередь им.

Намерение. Прежде всего не важно что человек делает, так не действие накапливает карму, а намерение. Если у вас на 100% благое намерение — это во первых значит, что вы не совершите поступков приводящих к падению, а во вторых вы гарантировано накопите благой кармы. В зависимости от посвящений которые вы применяете ваша благая карма может созреть в явления (отпуск на море, деньги, успех, удача) единожды, или как сезонные плоды плодоносного дерева. Консервация кармы для созревания в будущем сново и сново — это одна из тайных практик тибетского буддизма, без которой невозможно достичь состояния Будды, это называется посвящение Манджушри. Впервые такое посвящение применил исторический ученик Шакьямуни, бодхисаттва Манджушри, отсюда и название. Правильное намерение — это наше всё. Абсолютное намерение это Просветленный Майндстрим — стремление, что бы все существа стали Буддами. Делать для этого все возможное, в том числе и привод себя к состоянию Будды это один из способов достичь этого. Без такого намерения жизнь на Земле — это просто прозябание в желаниях.

Медитация. Медитация в буддизме это очень часто совсем не то, что понимают под этим словом обыватели. Естественное Спокойствие ума — это движение, движение мыслей в бардо бодрствования, поэтому единственный способ это раствориться в потоке своего сознания, один из важных и ключевых моментов — это принятие всех явлений. Все что совершается в этом мире — это естественное проявление законов причины и следствия по отношению к мышлению всех живых существ. Все получаю ровно столько и то на что заслуживают. Всему есть объяснение: войнам, кажущейся несправедливости и т.д.

Медитация состоит из двух элементов Транквилити (Шине или Шаматха) и Инсайт (Лхатонг Випашьяна). Основной элемент Шине — это концентрация. Основной элемент Випашьяны — это эмоциональный пик. Когда я работаю (и даже когда не работаю) я концентрируюсь на том, что я делаю. Помимо этого нет других вещей. Я полностью в работе. Моя чистая земля это мой код. Я гуру восседающий во дворце диктата своего благого мышления. Мои работники и коллеги — это собрание бодхисаттв. Мой заказчик — это линия моей передачи и Самантабхадра в одном лице. Я возвожу новые миры с помощью протипирования за считаные дни и довожу до совершенства узоры своего кода благодаря инкременткальному рефакторингу. Я и тестировщик, и программист и администор и менеджер и дизайнер и рекламщик и верстальщик и технический директор — Я Будда. Я это мой проект и проект это Я. Это мой путь и от этого зависит счастье всех существ. Я так хочу, я так думаю — поэтому так и есть. Я даю обет привести все живых существ к просветлению благодаря своей самоотверженной работе в чистой земле моего проекта. Мой код прекрасен, лаконичен и совершенен. Мой проект работает без cбоев. Я полностью контролирую все силой своей концентреции. Это Шине для программистов, подкрепленное намеренеим Просветленного Майндстрима. Надеюсь идею вы уловили, идите пишите код.
