% Copyright (c) 2022 Synrc Research Center

\usepackage[english,russian]{babel}
\usepackage{xeCJK}
\usepackage{fontspec}
\usepackage{graphicx,changepage,txfonts}
\usepackage{newunicodechar}
\usepackage{amsmath}
\usepackage{hyphenat}
\usepackage[top=18mm, bottom=25.4mm,
            inner=15mm,outer=18mm,
            paperwidth=142mm, paperheight=200mm]{geometry}

\hyphenation{}
\fontencoding{T1}
\addto\captionsrussian{\renewcommand{\contentsname}{Зміст}}

\newunicodechar{ꑭ}{ꑭ}
\newunicodechar{☸}{☸}
\newunicodechar{❤}{❤}
\newunicodechar{☸}{☸}
\newunicodechar{∀}{∀}

% \newcommand{\dharmachakra}{\setmainfont{Segoe UI Emoji}☸\setmainfont{Geometria}}

\newcommand{\includeimage}[2]
{\begin{figure}[h!]
\centering
\includegraphics[width=\textwidth]{#1}
\caption{#2}
\end{figure}
}

\newcommand*{\titleMONK}
{
\newfontfamily{\cyrillicfont}{Geometria}
\setmainfont{Geometria}
    \begingroup
        \thispagestyle{empty}
        \hspace*{0.15\textwidth}
        \rule{1pt}{\textheight}
        \hspace*{0.05\textwidth}
        {
        \parbox[c][][s]{0.75\textwidth}
        {
             \vspace{-15cm}
            \setmainfont{DDC Uchen}
             \textsc{}
            \\
             \textsc{\noindent
             \setmainfont{Geometria}
             Чо Дха \\ [0.3\baselineskip]
             \\
             \Large
             Божевільний монах \dharmachakra  \\[0.5\baselineskip]
             \\
             \small
             Вступ \\
             Для чого ви читаєте це? \\
             Прихисток \\
             Істини Благородних \\
             Взаємозалежність \\
             Мислення \\
             Філософія \\
             Медитація \\
             Усвідомлення \\
             Тантри Древніх \\
             Безпека \\
             Наставник \\
             Примордіальний звук \\
             Чим займається Божевільний монах? \\
             }
        }}
    \endgroup
    \setmainfont{Geometria}
}

\lefthyphenmin=1
\hyphenpenalty=100
\tolerance=3000

\newcommand{\ru}{\setmainfont{Geometria}}
\newcommand{\ti}{\setmainfont{DDC Uchen}}

\newlength\tindent
\setlength{\tindent}{\parindent}
%\setlength{\parindent}{0pt}
\renewcommand{\indent}{\hspace*{\tindent}}
