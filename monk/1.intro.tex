\section{Вступ}

Мета та етимологія назви.
Основна мета цього стріму забезпечити підтримку
у питаннях Дхарми для Тіртіків. Що таке Дхарма?
Дхарма має безліч значень на санскриті, від
загального абстрактного поняття «Феномен», закінчуючи
класичним «Вчення Будди» та більш фундаментальним
«Добре Мислення». Хто такі Тіртіки? Тиртіки --- це
всі, хто дотримуються філософських поглядів та систем, що
суперечать вченню Будди.

Цільова аудиторія. Передбачається, що основна аудиторія
цього блогу --- це молоді люди, які перебувають у
зрілому віці, вони мають вищу освіту і встигли розчаруватися
в Самсарі. Нове слово? Самсара це те, що приносить
нам страждання. Вам не вистачило останньої моделі
айфону у найближчому Apple Store? Вам зрадила дівчина?
У вас фізичний біль? Вам затримують зарплатню? Поломалася
дзеркалка? Вкрали велосипед? Все це страждання.
Дхарма --- це ліки від Страждання.

Зняття будь-якої відповідальності та обмеження щодо
використання цієї інформації. Цей блог не є афілійованим
до жодної релігійної організації, вчителя, монастиря чи
сектарного руху. Божевільний Монах знімає з себе будь-яку
відповідальність за можливу ментальну, фізичну, моральну
чи будь-яку іншу шкоду, завдану прямо чи опосередковано
через читання, чи саме існування цього стріму. Це не Дзен
Буддизм, Чань Буддизм, не Буддизм Тхеревади. Можна лише сказати,
що цей стрім пов'язаний із тибетською гілкою буддизму.
Але тут не буде Дзогчена, Тантри, Мадхьяміки, особистих
настанов чи відповідей на запитання. Це не керівництво до дії ---
це лише публічний блог Божевільного Монаха. Ви використовуєте
цю інформацію з левовою часткою критичного мислення і ви зріла
самостійна людина. Іншими словами --- це найповніший AS IS і PUBLIC DOMAIN.
