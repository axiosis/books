\frontmatter
\thispagestyle{empty}
\mbox{}\vspace{1in}
\noindent
\begin{flushright}
\vspace{0.5cm}
\textbf{\Huge ПЕРША ДЕРЖАВНА СИСТЕМА} \\
\vspace{0.5cm}
\textbf{\large Формальна модель та програмна архітектура \\
        \large функціонального верифікованого мовного забезпечення \\
        \large для побудови інфраструктурних процесінгових систем  \\
        \large орієнтованих на державну модель управління:  \\
        \large процесами для проведення зовнішнього аудиту, \\
        \large різними видами розподілених сховищ, \\
        \large телекомунікаційними та реєстровими фреймворками, \\
        \large інтернет-утворюючими сервісами зокрема та для \\
        \large автоматизації захищених автономних офісів і \\
        \large державних підприємств України у цілому. \\
}
\vspace{4em}
        \Large Навчальний посібник курсу «Інформаційні системи»
\vspace{1em}
\vspace{4cm}
\hfill{\Large Максим Сохацький \\
              18 лютого 2024, Київ, Україна \\
}
\vspace{0.3cm}
\hfill{}
\end{flushright}
\newpage
\noindent УДК 002\\
\noindent УДК 004.4, 004.6, 004.9
\epigraph{Присвячується всім державним службовцям України}

Система управління державними підприємствами ERP/1 визначає формальну
специфікацію та її імплементацію для сучасних оптимізованих підприємств
які вимагають сучасних засобів контролю операцій та цілістності даних.
\\
\\
Телекомунікаційна платформа Erlang/OTP від Ericsson успішно застосовується
в індустрії мобільними операторами понад 30 років, а її віртуальна машина
досі вважається однією з найкращих в галузі. Системи ERP на її базі також
уже не один рік використовується у банківській сфері, процесінгу транзакцій,
розподілених системах повідомлень, в IoT секторі. Ви можете переглянути демо
модулі системи ERP/1 в нашому захищеному середовищі зі своїм центром
випуску ECC X.509 сертифікатів. У цій книзі ви знайдете класичну авторську
монографію на тему архітектури та імплементації такої системи, побудованої
на міжнародних та державних стандартах України:
\\
\\
RFC: 7363, 6350, 4180, 5126, 5652,
     8567, 9006, 9011, 9019, 9159, 9100, 8323, 7815, 7228, 6455,
     8927, 8259, 4627, 7493, 7159, 4227, 3288, 6025, 5911, 4120, 4122, 7363, 6537, 6940, 7890,
     2251-2256, 6960, 5280, 1034-1035, 4033-4035.
\\
\\
ISO: 19510, 19514, 42010, 18033, 14888, 10118, 10116, 15946, 29146,
     9075, 27001,
     19464, 20922, 21823, 27402, 30161, 30165,
     20452, 42010, 19501, 19505,
     8824-8825.
\\
\\
NIST: 800-162.
\\
\\
ДСТУ: 28147, 15946, 9798, 4145, 319-422, 319-122.
\\
\\
Постійне посилання твору: https://axiosis.top/sep/ \\
Видавець: Державний науково-дослідний інститут МВС України
\\
\\
{\bf ISBN --- 978-617-8027-23-0 \hspace{2em}}
\\
\\
Підготовлено до друку на Подолі, м. Київ.
\\
\\
\begin{tabular}{ll}
\textcopyright{} 2024 & Максим Сохацький
\end{tabular}

\newpage
\cleartorecto
\tableofcontents*
\mainmatter
