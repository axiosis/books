\chapter{Обліково-реєстраційний рівень}

\section{Вступ}

Друге видання КНИГИ ERP (англ. ERP BOOK VOL.3 Blue Book) визначає формальну
бізнес специфікацію та її імплементацію для сучасних оптимізованих підприємств.
Системи ERP на її базі також уже не один рік використовується у банківській сфері,
процесінгу транзакцій, розподілених системах повідомлень, в IoT секторі.

\subsection{Види реєстрів}

1) Реєстри орієнтовані на суб'єктів організаційних систем; \\
2) Реєстри орієнтовані на облік матеріальних ресурсів; \\
3) Реєстри орієнтовані на георафічні об'єкти; \\
4) Реєстри орієнтовані на події (накази, укази, нормативно-правові акти); \\
5) Реєстри медичних систем (FHIR); \\
5) Реєстри предметно-орієнтованих словників для функціональних підсистем. \\

\subsection{Функціональні можливості}

\newpage
\section{Модулі підприємства}

ERP/1 є комплексом бібліотек (N2O.DEV) та підсистем додатків (ERP/1),
який використовує загальну шину і загальну розподілену базу даних для швидкіснх операційних вітрин.
\\
\\
\textbf{FIN} — Фінансовий модуль підприємства для бухгалтерії, зберігає бізнес процеси,
        які представляють собою рахунки учасників системи: персонал (для нарахування зарплат),
        рахунки та субрахунки підприємства (для здійснення економічної діяльності) і
        зовнішні рахунки в платіжних системах.
\\
\\
\textbf{ACC} — Система управління персоналом: зарплатні відомості,
        календар підприємства, відпустки, декретні відпустки, інші календарі.
\\
\\
\textbf{SCM} — Система управління ланцюжком поставок: головний БП системи —
           експедиційний процес доставки товарів ланцюжку одержувачів
           за допомогою транспортних компаній.
\\
\\
\textbf{PLM} — Система управління життєвим циклом проектів і продуктів.
           Також містить CashFlow та P\&L звіти.
\\
\\
\textbf{PM} — Система управління проектами підприємства з деталізацією
           часу і протоколів прийому-передачі (прийняті коміти в гитхабі).
\\
\\
\textbf{WMS} — Система управління складом та деталями.
\\
\\
\textbf{TMS} — Система управління транспортом підприємства.

\section{Архітектура CART системи}

