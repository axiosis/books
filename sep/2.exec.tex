\chapter{Органи виконавчої влади}

\section{Міністерство науки і освіти}

Дана наукова робота є статтею-дослідженням таксономії структури Міністерства освіти і науки з точки зору як інформаційної автоматизованої системи так і соціальної структури з точки зору державного управління. Як приклад, в статті наводиться конкретна структура, яка є незначною модифікацією існуючої ієрархічної системи Міністерства освіти і науки.

\subsection{Структурні підрозділи}

1. Патронатна служба \\
2. Управління початкової школи \\
3. Управління середньої школи \\
4. Управління вищої школи \\
5. Управління юстиції \\
\hspace{2cm}     — Департамент експертизи і сертифікації \\
\hspace{2cm}     — Інститут інтелектуальної власності \\
\hspace{2cm}     — Департамент кадрового забезпечення \\
\hspace{2cm}     — Департамент аудиту і внутрішніх розслідувань (CRM/DRF trace) \\
\hspace{2cm}     — Департамент архівної справи (SCAN) \\
\hspace{2cm}     — Технічний департамент (CA) \\
\hspace{2cm}     — Департамент соціального і гуманітарного забезпечення (ACC) \\
\hspace{2cm}     — Відділ кадрів (ACC) \\
\hspace{2cm}     — Юридичний департамент \\
\hspace{2cm}     — Департамент міжнародного співробітництва \\
6. Управління науково-дослідними інститутами (агенція) \\
7. Національна академія наук (агенція) \\
\hspace{2cm}     — Інститут формальної математики \\
\hspace{4cm}         — Ректорат формальної філософії \\
\hspace{4cm}         — Ректорат чистої математики \\
\hspace{4cm}         — Ректорат прикладної математики \\
\hspace{4cm}         — Ректорат мовного забезпечення \\
\hspace{4cm}         — Ректорат теоретичної інформатики \\
\hspace{2cm}     — Інститут формальної літератури \\
\hspace{2cm}     — Інститут музики, кіно і образотворчого мистецтва \\
\hspace{4cm}         — Національна консерваторія \\
\hspace{4cm}         — Національна академія мистецтв \\
\hspace{4cm}         — Національна кінематика \\
\hspace{2cm}     — Інститут фізики і матеріалів \\
\hspace{2cm}     — Інститут геології і геохімії \\
\hspace{2cm}     — Інститут хімії і біології \\
\hspace{2cm}     — Інститут соціальних і гуманітарних наук \\
\hspace{4cm}         — Ректорат філософії \\
\hspace{4cm}         — Ректорат археології \\
\hspace{4cm}         — Ректорат національної історії \\
\hspace{4cm}         — Ректорат права \\
\hspace{4cm}         — Національна бібліотека \\
8. Управління політиками і структурними підрозділами (агенція) \\
\hspace{2cm}     — Департамент комунікації (відділ кадрів) \\
\hspace{2cm}     — Департамент контролю виконання показників (CRM) \\
\hspace{2cm}     — Департамент планування переходу (аналіз процесів BPMN) \\
\hspace{2cm}     — Департамент трансформації (широкий спектр спеціалізацій) \\

\section{Міністерство охорони здоров'я}

\section{Міністерство внутрішніх справ}

\section{Міністерство закордонних справ}

\section{Міністерство оборони}

Ця секція є дослідженням таксономії структури Міністерства
оборони з точки зору як інформаційної автоматизованої системи
так і соціальної структури з точки зору державного управління.
Як приклад, в статті наводиться конкретна структура, яка є незначною
модифікацією існуючої ієрархічної системи Міністерства оборони України,
підсилена повним спектром інституцій для організації неперервного
науково-освітнього і технологічно-виробничого процесу існування
державного органу виконавчої влади — Міністерства оборони України.

У якості моделі гранулярності використана українська державна модель
(міністерство, управління, департамент, відділ, сектор). Оскільки дана робота
зосерджена в першу чергу на логіці існування процесу, тут значною мірою
надається перевага формальним моделям, які потребують мінімальних зусиль
для верифікації, моделювання і прогнозування. Оскільки формалізація процесу
безпосередньо торкається інформаційного програмного забезпечення на користь
приходять міжнародні стардарти телекомунікаційних протоколів, сертифікація
яких торкається (в свою чергу) університетів, науково-дослідних інститутів,
науково-виробничих інститутів. Науково-виробничі інститути використовується
в широкому смислі як ті, що можуть бути комерціними угруповуваннями,
міжнародними фундаціями, тощо. Для підтримки автономної діяльності ці
всі інституції повинні входити в арсенал функціональних можливостей
міністерства, включаючи головним чином універсистет четветого рівня
акредитації, навчальні програми якого представлені частково обраними
курсами пʼяти кафедр інституту математики НАН (див. Додатку 1).
Розміщення інформаційної структури розгалуженої національної
структури міністерства і його частин передбачає автономне
забезпечення класу EDGE офіс з власним ресурсами охолодження,
водо-електро-постачання, силами та засобами оборони.

Інформаційна політика формального моделювання передбачає довільне
використання мовних сучаних засобів здатних до формальної верифікації (наявність
промислових верифікаторів) ТЗІ рівня Г7 (повна математична верифікація)
покладаючись основним чином на телекомунікаційні протоколи і міжнародні
ISO стандарти (див. Додаток 5).

\subsection{Мотивація}

Основна мотивація даної роботи полягає у висвітленні таксономії
Міністерства оборони України з точки зору оптимізації, автономності
існування (sustainability), само-відтворюваності, підтримки життєвого
циклу існування Міністерства.

\subsection{Модель}

1. Патронатна служба \\
2. Управління освіти і науки (агенція) \\
\hspace{2cm}     — Університет четвертого рівня акредитації (див. Додаток 1) \\
\hspace{2cm}     — Науково-дослідні інститути \\
\hspace{2cm}     — Науково-виробничі інститути \\
3. Управління медицини (агенція) \\
\hspace{2cm}     — Клінічні наукові дослідження та лабораторії, НДІ ПВМ (MED) \\
\hspace{2cm}     — Програми реабілітації («Пуща-Водиця», «Трускавецький», «Хмельник») \\
\hspace{2cm}     — Клінічні, мобільні (4) лікарні, госпіталі (14) \\
\hspace{2cm}     — Медичні служба (5 родів), тактична медицина, медичні сили \\
\hspace{2cm}     — Інститут медицини (ЗДМУ, ХНМУ, ЛНМУ, ТНМУ) \\
4. Виробничо-промислове управління (агенція) \\
\hspace{2cm}     — Конструкторські бюро (ДАТ «Укроборонпром», ДАХК «Артем») \\
\hspace{2cm}     — Департамент економічного моделювання і планування (PLM, PM) \\
\hspace{2cm}     — Департамент ресурсного забезпечення (SCM, TMS, WMS) \\
\hspace{2cm}     — Департамент бюджетування і закупівель (FIN) \\
\hspace{2cm}     — Департамент будівництва і архітектури, ліній виробництва (ArchiCAD) \\
\hspace{2cm}     — Телекомунікаційний департамент інформаційних систем (див. Додаток 2) \\
\hspace{4cm}         — Відділ систем врядування \\
\hspace{4cm}         — Відділ облікових систем \\
\hspace{4cm}         — Відділ телекомунікаційних систем \\
\hspace{4cm}         — Відділ безпекових протоколів Інтернет \\
\hspace{2cm}     — Департамент авіації, авіоніки, аеронавтики і безпілотних систем (AutoCAD) \\
\hspace{4cm}         — Відділ авіації \\
\hspace{4cm}         — Відділ авіоніки \\
\hspace{4cm}         — Відділ аеронавтики \\
\hspace{4cm}         — Відділ безпілотних систем \\
\hspace{2cm}     — Департамент машинобудування (terrain) (SolidWorks) \\
\hspace{2cm}     — Департамент кораблебудування \\
5. Управління юстиції \\
\hspace{2cm}     — Відділ експертиз і сертифікації (ISO/IETF) \\
\hspace{2cm}     — Департамент архівної справи (SCAN) \\
\hspace{2cm}     — Департамент аудиту і внутрішніх розслідувань (CRM/DRF trace) \\
\hspace{2cm}     — Технічний департамент (CA) \\
\hspace{2cm}     — Департамент соціального і гуманітарного забезпечення (ACC) \\
\hspace{2cm}     — Відділ кадрів (ACC) \\
\hspace{2cm}     — Юридичний департамент \\
\hspace{2cm}     — Департамент міжнародного співробітництва \\
6. Управління розвідки \\
7. Головне управління позиційною політикою \\
\hspace{2cm}     — Мобілізаційний департамент \\
\hspace{2cm}     — Департамент навчальних програм \\
\hspace{2cm}     — Департамент сил і засобів оборони CRM (див. Додаток 3) \\
\hspace{4cm}         — Родина сухопутних військ \\
\hspace{4cm}         — Родина повітряних сил \\
\hspace{4cm}         — Родина воєнно-морських сил \\
\hspace{4cm}         — Родина спеціальних сил (ССО, МС, РЕБ, РХБЗ, ТРО) \\
\hspace{4cm}         — Родина кібербезпеки і ДШВ \\
\hspace{2cm}     — Департамент операційного контролю і управління DFR (див. Додаток 4) \\
\hspace{2cm}     — Економічний департамент \\
\hspace{4cm}         — Відділ ресурсного забезпечення (WMS) \\
\hspace{4cm}         — Відділ бюджетування і закупівель \\
8. Управління політиками і структурними підрозділами (агенція) \\
\hspace{2cm}     — Департамент комунікації (відділ кадрів) \\
\hspace{2cm}     — Департамент контролю виконання показників (CRM) \\
\hspace{2cm}     — Департамент планування переходу (аналіз процесів BPMN) \\
\hspace{2cm}     — Департамент трансформації (широкий спектр спеціалізацій) \\

\subsection{Принципи}

Одним з головних принципів закледаних у фундамент МО є принцип вищої освіти, яка здобувається згідно до вимог міжнародних стандартів. Для забезпечення потреб існування працівників всіх сфер цієї таксономії в основу її неперервної маніфестації покладено існування університету четвертого рівня акредитації з усіма спеціальностями необхідними для покриття потреб самого МО.

Одним з універсальних принципів закладених в фундамент МО є принцип розподілу влади, з якого випливає три корпуси МО: 1) Політичний корпус (головне управління, перехідне управління, патронатна служба), 2) Виконавчий корпус (Управління освіти і науки, Медичне управління, Виробничо-промислове управління), 3) Судовий корпус (Управління юстиції, трибунал). Політичний корпус передбачає виділення окремої агенції яка здійснює запуск процесів створення і апробація структурних підрозділів міністерства. Сюди також входить структурний підрозділ — головне управління яке управляє силами (ЗСУ) та засобами оброни. Виконавчий корпус убезпечує інтелектуально-ємні структурні підрозділи і виокремлює їх під автономний контроль агенції з більшою дотичністю до зовнішніх структур. Судовий корпус — окремий стурктурний підрозділ з процесами аудиту і внутрішніх розслідувань в існуючих соціально-інформаційних системах в структурі МО.

\subsection{Структурні підрозділи}

1. Розподіл влади і мінімізація таксономії \\
2. Нормалізація формальних процесів \\
3. Аудит міністерства і процес трансформації \\
4. Архітектура структурних підрозділів \\
5. Апробація результатів і циклічність мета-процесу \\

Розподіл влади є головним принципом ефективного управління. Контролюючі
органи повинні спеціалізуватися на розслідуваннях і бути виокремлені.
Політики процесів і реквізітної інформації повинні здійснюватися політичним
органом. Політика повинна здійснюватися окремими виконавчими
органами (освіта, наука, виробництво).

Вимоги до документування процесів повинні бути на найвищому рівні (еквіваріантні
семантики, спрощення процесів до нормальних форм, мінімазія горизонтальних
звʼязків), відповідати вимогам постанови №55 КМУ і наказу №124 МО. Повинна
бути організований процес історіографії і аудиту діло-процесів згідно
стандарту ISO-19510 для аудиту NATO. Запуск діло-виробництва передбачається
поступово і гранулярно, спочатку від головних і малоресурних проєктів
управління політик і далі згідно стратегії управління по іншим структурним підрозділам.

\subsubsection{Патронатна служба}

Сприяння реалізації політичних цілей Міністра, консультування Міністра, організаційне, інформаційне, експертно-аналітичне забезпечення діяльності Міністра.

\subsubsection{Управління освіти і науки (агенція)}

Науковий процес крім освітнього видіяє науково-дослідний і наково-виробничий у тих сферах, яких потребують департаменти організаційної структури МО. Ці компанії повинні знаходитися як і університет у сфері впливу МО.

\subsubsection{Виробничо-промислове управління (агенція)}

Пропонується повне дублювання сфер виробництва засобів оборони у відповідні департаменти, так як це є основними ресурсами головного управління тому доцільно зберігати бачення повної картини під ієрархією МО. Існуючі виробничі процеси зосереджені в «Укроборонпром» і «Артем», пропонується розглядати як зовнішні конструкторські бюро, в одному переліку з комерційними підприємствами які працюють можливо навіть проектно.

\subsubsection{Управління юстиції}

Управління юстиції бере на себе функції обліку сил (відділ кадрів) та протокольних дій в середині системи, які аналізуються антикорупційним відділом департаменту аудиту і внутрішніх розслідувань. Тут також зосереджені юридичні функції які обслуговують всю ієрархію МО, а також зовнішніх партнерів (NATO). Також тут зосереджений центр видачі криптографічних ключів всієї ієрархії.

\subsubsection{Головне управління}

Спрощені функції головного штабу з локальним резервом сил та засобів оборони. Головна консоль (Godot) управління підпорядковуєть логіці ведення позиційної політики силами та засобами оборони з прогнозування подій та ціною їх усунення. Позиційна стратегія передбачає локальний облік сил та засобів оборони.

\subsubsection{Управління політиками і структурними підрозділами (агенція)}

Формальна модель адміністративного управління переходу (або врядування) від існуючої структури до будь-якої наперед заданої (як приклад наведеної). Перехідне управління займається аудитом існуючих процесів (юридичне забезпечення) та їх трансформації в урядові та облікові системи міністерства у взаємодії з Телекомунікаційним департаментом, контролем їх виконання: впровадження, апробації, калібрування, проектний менеджмент.

Процес трансформації розділений на наступні категорії (обернено до швидкоплинності): 1) трансформація (діджіталізація) існуючих процесів без модифікації логіки, такі як діловодство; 2) створення нових процесів і технологічного забезпечення для вакантних департаментів і відділів, такі як виробничо-промислове управління; 3) планування і розвиток вакантних департаментів і управлінь (довготривалий процес, розвиток навчальних програм, побудова та підтримка життєвого циклу продуктів).

Кожна фаза процесу трансормації передбачає побудову життєвого циклу проєкту, а також продукту, який покладений в його основу, сюди входить повний набір документації: Технічне Завдання, Ескізний проєкт, Технічний проєкт, Технічна документація. Управління політиками поділяє сфери компетенції з Телекомунікаційним департаментом виробничо-промислового відділу і загальні правила документування.

Важливі, головні процеси управління політиками, як основні його функції можна класифікувати так: 1) архівна справа процесного виробництва; 2) розробка процесів запуску структурних підрозділів; 3) розробка процесів документообігу; 4) розробка процесів управління і координації; 5) розробка технологічних процесів ISO-9001.

\subsection{Результати}

В статті представлена запропонована таксономія нормалізованого міністерства з урахування міжнародного досвіду організації державних структур з чітким і прозорим принципом розподілу влади і організації глобального (стратегічного) і локального (операційного) контекстів для виконання своїх функції у найпростіший і безпосередній спосіб мінімізуючи кількість протоколів взаємодії всередині системи.

В додатках статті представлена таксономія освітніх програм університету четвертого рівня акредатації і таксономія телекомунікаційного депратемнту, у функції якого входять розробка і впровадження системи. В процесі виконання завдання первинного моделювання було здійснено намагання охопити ключові органи, аж до рівня гранулярності відділів і секторів (побудова первинного індексу) і їх горизонтальних протоколів взаємодії.

\subsection{Бібліографія}

ДСТУ 2732-94 Діловодство і архівна справа. \\
26.05.2014 №333 Інструкція з організації обліку особового складу. \\
21.11.2017 №608 Порядок проведення службового розслідування. \\
26.07.2018 №370 Інструкція з діловодства. \\
07.04.2017 №124 Інструкція з діловодства. \\
07.10.2015 №393 Положення Про Юридичну Службу. \\
29.11.2018 №604 Інструкція з надання доповідей і донесень про події,
                кримінальні правопорушення, військові адміністративні
                правопорушення та адміністративні правопорушення,
                пов'язані з корупцією, порушення військової дисципліни та їх облік.

\section{Міністерство фінансів}

\section{Міністерство економіки}


