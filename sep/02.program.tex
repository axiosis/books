\chapter{Національна програма інформатизації}

Мета Національної програми інформатизації (НПІ) --- створення цифрового простору як
проміжний птап розвитку інформаційного суспільства, прозорого правового середовища,
яке захищене і базується на міжнародних стандртах та забезпечує інформаційні потреб та
реалізацує права і свободи громадян на основі своєчасної, достовірної та повної інформації,
підвищення ефективності державного управління.

Національна програма інформатизації (НПІ) веде свою історію з 74/98-ВР документа 1998 року,
якій містив 28 статей, до поточного документа 2024 року 2807-IX, який містить вже 15 статей.
Головним чином НПІ визначає протоколи запуску та термінації програм які містять наступні функції:
експертизи, аналізу, формування, контролю, виконання, звітування програм інформатизації
на державному (галузеві програми) ті місцевому (самоврядування) рівні, а також визначає
субʼєктів інформатизації: генеральний замовник --- Міністерство цифрової трансформації,
Керівник --- посадова особа, виконавці та підрядники. НПІ є власником і розробником
системи обліку таких програм. НПІ визначає процеси розробки згідно ISO/IEC 12207 та ISO 9001,
а супроводу та підтримки згідно ISO/IEC/IEEE 14764:2022.

\section{Загальні принципи}

На мета рівні неперервний процес реформування ОВВ зараз уявляється мені,
як такий що керується наступними правилами: 1) Інʼєктивність управління
юстиції (як необхідний атрибут кожного міністерства, аудит, розслідування);
2) ІТ-департамент або управління (у якості зовнішнього ЄДРПО, холдер продуктів
Міністерства); 3) Управління трансформації (яка механізує процес ІТ-продуктами
ІТ-департаменту). Далі управління Міністерства додаються в залежності від
функцій Міністерства, але ці три плюс патронатна служба — обов’язкові.
Міністерство Юстиції, Судова система, генеральна прокуратура, Поліція
та інші агенції, як НАЗК, НАБУ, мають безпосередній або опосередкований
доступ до (1). Це має регулюється відповідними ABAC правилами. Мінцифра
має доступ до (3), як координатор мета-процесу трансформації. Також
Мінцифра координує роботу і взаємодіє з (2) кожного міністерства для
забезпечення каналу до реєстрів відповідних міністерств, які підтримуются
відповідними ІТ-управліннями кожного мінстерства. Шини всіх документобігів
і реєстрів координуються ІТ-управлінням Мінцифри, «Дія».

Завдання тансформації передбачає виконання (у тому числі) наступних цілей:
1) Кожне міністерство буде мати свій автономний і потужний ІТ-департамент,
який обслуговує реєстри міністерства; 2) Міністерства будуть мати свої
управління трансформації, працюватимуть на одному продукті, в якому
будуть моделювати свою роботу; 3) Мінцифра як координатор буде затверджувати
регламенти роботи ІТ-управлінь і управлінь трансформації кожного міністерства.
«Дія» буде мати не тільки шину документообігу, продукт «Дія: Документообіг»
(база), але і видавати ліцензії для учасників ринку (зараз 30 ліцензій).

\section{Модель}

1. Кабінет міністрів України \\
2. Міністерство освіти і науки України (МОН) \\
3. Міністерство охорони здоров'я України (МОЗ) \\
4. Міністерство внутрішніх справ України (МВС) \\
5. Міністерство закордонних справ України (МЗС) \\
5. Міністерство оборони України (МО) \\
6. Міністерство юстиції України (Мінюст) \\
7. Міністерство фінансів України (Мінфін) \\
8. Міністерство економіки України (Мінекономіки) \\
9. Міністерство енергетики України (Міненерго) \\
10. Міністерство соціальної політики України (Мінсоцполітики) \\
11. Міністерство регіональної політики України (Мінрегіон) \\
12. Міністерство цифрової трансформації України (Мінцифра) \\
13. Міністерство молоді та спорту України (Мінспорту) \\
14. Міністерство природних ресурсів України (Мінекології) \\
15. Міністерство аграрної політики України (Мінагрополітики) \\

\section{Структурне ядро}

1) Управління юстиції;
2) Департамент інформаційних систем виробничо-промислового управління;
3) Управління трансформації

Оскільки соціо-інформаційні системи повинні бути автономними та мати
керований життєвий цикл, технічне відображення організаційної струткури
повинно бути під контролем міністерства, можливо у вигляді окремого
підприємства (для убезпечення інтелктуальних ресурсів підприємства
від ручного керування, а також оскільки ІТ-депратаменти міністерст
відповідають за реєстри і персональні дані громадян).

Для керування структурою в реальному часі пропонується окремий вид
протолу ОВВ 2.0 як наступне розширення після НПА до вже існуючого
базового протоколу Документообігу згідно постанови №55 керуючого
органу Кабінету Міністрів України. Наприклад, ДІТ (Державна
Інституційна Трансформація).

В процесі як операційного документообігу (породжуваного структурними
підрозділами міністерств), так і інституційного (породжуваного управліннями
або агенціями державної цифрової трансформації) породжуються зліпки кваліфікованих
електронних підписів посадових осіб які розслідуються як внутрішніми
органами (відділи аудиту і відділи внутрішніх розслідувань), так і
координуючим органом — Міністерством юстиції України.

\section{Протокол державної інституційної транфсормації}

Цей протокол визначає наступні операції над організаційною структурою і її політикою:

1) Створення і ліквідація структурних підрозділів;
2) Погодження та модифікація установчих конституційних документів;
3) Розробка технопроектів структурних підрозділів та їх систем;
3) Вибір і впровадження інформаційних систем для структурних підрозділів;
4) Запуск та операційна діяльність структурних підрозділів (виробництво);

