\addtocontents{toc}{\protect\newpage}
\chapter{Державна система}

По аналогії зі стандартом ISO 42010 «Фреймворку Закмана»,
фреймворк Максима Сохацького визначає та уточнює архітектурні рівні
з яких складаються сучасні корпоративні інформаційні системи:
\\
\\
--- Юридично-документальний рівень\\
--- Обліково-реєстровий рівень\\
--- Зв'язність людей та пристроїв\\
--- Телекомунікаційна платформа\\
--- Схема та метаінформація\\
--- Безпека інтернету\\

\section{Юридично-документальний рівень}

Згідно фреймворку верхній шостий рівень визначає BPMN процеси згідно яких здійнюється
відзеркалення юридично-правивих відносин електронного документообігу. Кожен крок такого
процесу, та усі його документи підписуються особистим ключем КЕП посадової особи, що дає
змогу проведення диспутів та розслідувань Міністерством юстиції України. Окрім того цей
рівень системи орієнтований на аналітику у взаємодії з громадянами через СЕВ ОВВ.
\\
\\
У 2022 році юридично-документальні системи ERP/1 будуються на сховищі з єдиним
простором ключів Facebook RocksDB, що здатне працювати через Intel SPDK на NVMe
дисках, наприклад у складі таких сховищ як CEPH. Обсяг обігу документів на великих
підприємствах сягає 1ТБ на рік.

\newpage
\section{Обліково-реєстровий рівень}

Обліково-реєстровий рівень пропонує низькорівневе масштабоване розподілене
журнальне сховище даних та метаданих, яке може бути побудоване на реляційних
базах даних, базах даних з єдиним простором ключів з гарантіями
консистентності (chain-hash) або їх комбінаціях.
\\
\\
Класичні представники цього рівня в системах управління підприємствами: система
управління людськими та матеріальними ресурсами, банківські системи PCI DSS,
складські системи, системи управління поставками та виробництвом, системи сервісних
послуг, системи управління проектами, тощо.

\section{Зв'язність людей та пристроїв}

Рівень зв'язності людей та пристроїв визначає комунікаційні протоколи та технології,
які об'єднують головні ресурси підприємства (пристрої та людей) у одну
телекомунікаційну мережу. Як правило виробництво складається з багатьох пристроїв
що підключаються до промислових шин як MQTT, та робочих місць користувачів.

З точки зору продуктів цей рівень представляється зазвичай корпоративними
комунікаторами та дашбордами де здійнюється моніторинг роботизованого обладнання:
пристрої, датчики, тощо. Ресурси підприємства — люди та пристрої як правило
зберігаються в LDAP директорії підприємства.

\newpage
\section{Телекомунікаційна платформа}

Рівень платформи визначає засоби мастшабування пам'яті (персистентної та волатильної) та
обчислювальних ресурсів (за допомогою процесінгових брокерів доставки повідомлень).
Це рівень визначає реляційні бази даних та бази даних з єдиним простором ключів,
а також стандарти та протоколи передачі інформації у промислових ERP системах,
такі як CSV, JSON, SOAP, BERT, ASN.1, тощо.

\section{Схема та метаінформація}

Рівень схеми даних визначає модель зберігання даних як з точки зору об'єктів-сутностей
та і з точки зору технологій та протоколів, які необхідні для їх опису.
Головним чином це Фреймворк Закмана та сімейство стандартів які описують UML.

\section{Безпека інтернету та інфраструктури}

Рівень безпеки визначає схему функціонування основного центрального засвідчувального орнагу,
акредитованих центрів сертифікації ключів, протоколи шифрування та підпису, директорію
підприємства, інтернет протоколи найменування ресурсів. Усе визначено згідно ASN.1 специфікації.
Компанія ІНФОТЕХ є утримувачем та автором усіх імплементації.

