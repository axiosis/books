\chapter*{Передмова автора}

\section*{Присвята}

В колофоні написано шо посібник присвячено всім держслужбовцям України,
це означає шо вони, як частина соціо-інформаційної системи державного устрою
є основними кінцевими користувачами системи і для них, майбутніх розробників,
операторів, архітекторів, керівників, аналітиків написаний цей твір.

\section*{Ідея виникнення і історія розвитку системи}

Ідея написати ERP/1 виникла в мене одразу після початку шляху
підприємництва, в 2005 році, коли я вибирав між OCaml, Haskell та Erlang платформу
на якій будувати модель, яку я хочу впроваджувати. Так сталося, що під мої
критерії мотивації підійшла тільки Erlang платформа і я почав шлях
спочатку з соціальної турецької мережі, потім ПриватБанк, потім месенджер NYNJA,
потім аспірантура по AXIO/1, потім міністерські системи.

Частини системи, такі як клієнтські компоненти для TWAIN сканування та роботи з Word, Excel
документами написані на C\#/F\# так як перший досвід роботи був пов'язаний з .NET.
Вже 18 років ERP/1 обслуговує потреби замовників, а її ідеї розповсюдилась за цей час
на виробничий простір багатьох технологій, таких як LiveView, HTMX, Turbo, і більш старших:
Ocsigen, WebSharper, Lift, Nitrogen.

Зараз ERP/1 включає повністю всі рівні державних систем, а коштовність
володіння як основний показник якості зменшений до мінімуму, так як вся система
може бути запущена на одній ноді в одній віртуальній машині і написана на одній мові.
Кожен з модулів ERP/1 вміщаєьтся на одну дискету 3.5" 2.88 МБ.

\newpage
\section*{Розкриття компонент основної мотиваці}

При розробці системи застосовувалася філософія мінімалізму в галузі
програмного забезпечення\footnote{\url{https://slides.com/maximsokhatsky/minimal/}},
яку я розказував на одному з київських форумів. Якшо стисло, то вона
зводиться до оптимізаційних критеріїв по часу, якості, людям і ресурсам:
1) зменшення часу виконання і розробки та збільшення часу життя системи і її ефективності;
Основні показники: Computation/Time, Weeks/Release, Feature/Hour, Hour/Lifetime;
2) зменшення складності і ціни та збільшення надійності та простоти;
Основні показники: Bugs/Code, Failures/Period, Cost/Support, Time/Fix;
3) зменшення невизначеності та збільшення зрозумілості і очікуваності;
Основні показники: Commiters/App, Issues/Feature, Msgs/Issue, Commits/User;
4) зменшення обслуговування і додаткових ресурсів та збільшення ємності даних і обчислень;
Основні показники: Cost/Information, Machines/User, Data/User,
Size/Features, Requests/Time, Information/User.

\section*{Коментарі та настанови для слухачів курсу}

Цей курс є особливим в тому сенсі, що предмет дослідження,
інфрормаційна система ERP/1 одночасно добре формалізована
і підходить у якості педагогічного дидактичного матеріалу,
а також має стабільну історію впроваджень і підтримку сумісності.

Курс розділений на 12 частин, кожна з яких відповідає за певний
етап розробки програмного забезпечення згідно ISO-9001, сюди входять:
аналіз предметної області, робота з юридичними документами, формування
архітектури і ескізного проекту, модель OSI і Закмана, аж до перевірки
кваліфікаційного електронного підпису і шифрування по ДСТУ-4145 без
допоміжних бібліотек. Цей посібник міг би бути також корисний
для курсу по державному управлінню.

Цей посібник написаний з урахуванням вимог які висувається для EDGE офісів,
з підвищенним порогом входу по розміру коду, латенсі, відмовостійкості,
розподіленості та вартості підтримки.

\section*{Подяки}

Я би хотів подякувати всім вчителям, так як планетарно інститут
педагогіки переживає кризу поваги до вичтелів.

\chapter{Вступ}

Цей посібник описує формальну модель та програмну архітектуру
функціонального верифікованого мовного забезпечення
для побудови інфраструктурних процесінгових систем
орієнтованих на державну модель управління:
процесами для проведення зовнішнього аудиту,
різними видами розподілених сховищ,
телекомунікаційними та реєстровими фреймворками,
інтернет-утворюючими сервісами зокрема та для
автоматизації захищених автономних офісів і
державних підприємств України у цілому.

Система управління державними підприємствами ERP/1 представлена у
посібнику є не тільки ідіоматичним прикладом побудови інформаційних державних та комерційних систем у цілому,
але і визначає формальну специфікацію та її імплементацію (з багатьма національними впровадженнями)
для сучасних оптимізованих підприємств які вимагають сучасних засобів контролю операцій та цілістності даних.

Телекомунікаційна платформа Erlang/OTP від Ericsson успішно застосовується
в індустрії мобільними операторами понад 30 років, а її віртуальна машина
досі вважається однією з найкращих в галузі. Системи управління підприємствами та інші інформаційні системи
на її базі також уже не один рік використовується у банківській сфері, процесінгу транзакцій,
розподілених системах повідомлень, в IoT секторі. Ви можете переглянути демо
модулі системи ERP/1 в нашому захищеному середовищі зі своїм центром
випуску ECC X.509 сертифікатів. У цій книзі ви знайдете перелік модулів
системи та основні сутності схеми.

Універсальна платформа для створення та забезпечення функціонування
інформаційних реєстрів баз (банків) даних різних масштабів: від базових
міжсистемних довідників та класифікаторів, до високонавантажених корпоративних,
місцевих та державних ресурсів. Цей посібник буде корисний всім,
хто хоче зрозуміти які інформаційні системи застосовуються і
державному і комерційному секторах.

\section*{Скорочення}

ЄРЗ (Єдиний реєстр зброї НПУ),
СУСЗЦЗ (Система управління силами та засобами цивільного захисту ДСНС),
ЄІС (Єдина інформаційна система МВС),
ЕСОЗ (Електронна система охорони здоров'я МОЗ),
ФП МТРЗ (Функціональна підсистема матеріально-технічного та ресурсного забезпечення МВС),
НГУ (Національна гвардія України),
ДСНС (Державна служба з надзвичайних ситуацій МВС),
ГСЦ (Головний сервісний центр МВС).
