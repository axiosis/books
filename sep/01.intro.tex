\chapter*{Передмова автора}

— Присвята \\
— Ідея виникнення і історія розвитку системи \\
— Розкриття компонент основної мотиваці \\
— Коментарі та настанови для слухачів курсу \\
— Подяки \\

\chapter{Вступ}

Цей посібник описує формальну модель та програмну архітектуру
функціонального верифікованого мовного забезпечення
для побудови інфраструктурних процесінгових систем
орієнтованих на державну модель управління:
процесами для проведення зовнішнього аудиту,
різними видами розподілених сховищ,
телекомунікаційними та реєстровими фреймворками,
інтернет-утворюючими сервісами зокрема та для
автоматизації захищених автономних офісів і
державних підприємств України у цілому.

Система управління державними підприємствами ERP/1 представлена у
посібнику є не тільки ідіоматичним прикладом побудови інформаційних державних та комерційних систем у цілому,
але і визначає формальну специфікацію та її імплементацію (з багатьма національними впровадженнями)
для сучасних оптимізованих підприємств які вимагають сучасних засобів контролю операцій та цілістності даних.

Телекомунікаційна платформа Erlang/OTP від Ericsson успішно застосовується
в індустрії мобільними операторами понад 30 років, а її віртуальна машина
досі вважається однією з найкращих в галузі. Системи управління підприємствами та інші інформаційні системи
на її базі також уже не один рік використовується у банківській сфері, процесінгу транзакцій,
розподілених системах повідомлень, в IoT секторі. Ви можете переглянути демо
модулі системи ERP/1 в нашому захищеному середовищі зі своїм центром
випуску ECC X.509 сертифікатів. У цій книзі ви знайдете перелік модулів
системи та основні сутності схеми.

Універсальна платформа для створення та забезпечення функціонування
інформаційних реєстрів баз (банків) даних різних масштабів: від базових
міжсистемних довідників та класифікаторів, до високонавантажених корпоративних,
місцевих та державних ресурсів. Цей посібник буде корисний всім,
хто хоче зрозуміти які інформаційні системи застосовуються і
державному і комерційному секторах.

\section*{Скорочення}

ЄРЗ (Єдиний реєстр зброї НПУ),
СУСЗЦЗ (Система управління силами та засобами цивільного захисту ДСНС),
ЄІС (Єдина інформаційна система МВС),
ЕСОЗ (Електронна система охорони здоров'я МОЗ),
ФП МТРЗ (Функціональна підсистема матеріально-технічного та ресурсного забезпечення МВС),
НГУ (Національна гвардія України),
ДСНС (Державна служба з надзвичайних ситуацій МВС),
ГСЦ (Головний сервісний центр МВС).
