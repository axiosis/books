\addtocontents{toc}{\protect\newpage}
\chapter{Інфраструктурний рівень безпеки інтернету}

\section{Центри сертифікації CA, АЦСК, ЦЗО та ОЗО}

\subsection{Електронний підпис і цифрова печатка}

Кваліфікований Електронний Підпис, або Кваліфіклована Електронна Печатка ---
це набір стандартів криптографічного захисту ДСТУ 4145,
та міжнародних стандартів які визначають його конверт: X.501, X.509, X.511, X.520.

\renewcommand{\footnotesize}{\tiny}

Серія міжнародних стандартів X.500, групується по категоріям, кожна
з яких має свій перелік ASN.1 файлів. Аби підключити усі визначення необхідні
для КЕП використані наступі компоненти стандартів (виділені болдом):
X.501 --- BasicAccessControl \footnote{\url{https://www.itu.int/ITU-T/formal-language/itu-t/x/x501/2019/BasicAccessControl.html}},
InformationFramework \footnote{\url{https://www.itu.int/ITU-T/formal-language/itu-t/x/x501/2019/InformationFramework.html}},
UsefulDefinitions \footnote{\url{https://www.itu.int/ITU-T/formal-language/itu-t/x/x501/2019/UsefulDefinitions.html}};
X.509 --- SpkmGssTokens \footnote{\url{https://www.itu.int/ITU-T/formal-language/itu-t/x/x509/2019/ExtensionAttributes.html}},
PkiPmiExternalDataTypes \footnote{\url{https://www.itu.int/ITU-T/formal-language/itu-t/x/x509/2019/PkiPmiExternalDataTypes.html}},
AttributeCertificateDefinitions \footnote{\url{https://www.itu.int/ITU-T/formal-language/itu-t/x/x509/2019/AttributeCertificateDefinitions.html}}, \\
AlgorithmObjectIdentifiers \footnote{\url{https://www.itu.int/ITU-T/formal-language/itu-t/x/x509/2019/AlgorithmObjectIdentifiers.html}}, \\
AuthenticationFramework \footnote{\url{https://www.itu.int/ITU-T/formal-language/itu-t/x/x509/2019/AuthenticationFramework.html}},
CertificateExtensions \footnote{\url{https://www.itu.int/ITU-T/formal-language/itu-t/x/x509/2019/CertificateExtensions.html}}; \\
X.511 --- SpkmGssTokens \footnote{\url{https://www.itu.int/ITU-T/formal-language/itu-t/x/x511/2019/SpkmGssTokens.html}},
DirectoryAbstractService \footnote{\url{https://www.itu.int/ITU-T/formal-language/itu-t/x/x511/2019/DirectoryAbstractService.html}};
X.520 --- PasswordPolicy \footnote{\url{https://www.itu.int/ITU-T/formal-language/itu-t/x/x520/2019/PasswordPolicy.html}},
UpperBounds \footnote{\url{https://www.itu.int/ITU-T/formal-language/itu-t/x/x520/2019/UpperBounds.html}},
SelectedAttributeTypes \footnote{\url{https://www.itu.int/ITU-T/formal-language/itu-t/x/x520/2019/SelectedAttributeTypes.html}}.

Можно було би винести необхідні визначення одразу в KEP.asn1,
однак цим хотілося підкреслити сумісність з міжнародними стандартами. Окрім серії
протоколів X.500, КЕП ще визначає також запити та відповіді OCSP, також у ASN.1 форматі.

На відміну від самого алгоритму КЕП, який визначено ДСТУ 4145,
конверти визначаються не стандартами, а наказами міністерства юстиції:
Проект наказу Адміністрації Держспецзв'язку та Держкомінфоматизації (2009) \footnote{\url{http://www.dsszzi.gov.ua/dsszzi/control/uk/publish/article?art\_id=77726}},
Наказ Міністерства юстиції України 1236/5/453 \footnote{\url{https://zakon.rada.gov.ua/laws/show/z1401-12}}.
Керуючись цими нормативними документами було створено файл KEP.asn1 \footnote{\url{https://github.com/synrc/ca/blob/master/priv/kep/KEP.asn1}},
який є одним з трьох top-level файлів необхідниї для компіляції ASN.1 компілятором\footnote{\url{https://asn1.erp.uno}}.

Існує небагато безкоштовних та повних компіляторів (генераторів парсерів)
ASN.1 специфікацій. Erlang є прикладом системи, до складу якої входить
першокласний безкоштовний з відкритою ліценцією ASN.1 компілятор, де
файли в ASN.1 нотації можуть бути зкомпільовані безпосередньо Erlang компілятором:

\renewcommand{\footnotesize}{\normal}

\begin{lstlisting}
> erlc AuthenticationFramework.asn1
> erlc InformationFramework.asn1
> erlc KEP.asn1
\end{lstlisting}

Створити файл підпису PKCS-7 можна за допомогою будь якої програми сертифікованої в Україні.
Найпростіше отримати свою КЕП печатку будучи клієнтом ПриватБанку. За допомогою
"Користувача ЦСК" компанії ІІТ ви можете підписувати файли використовуючи безкоштовну
форму приватного ключа у вигляді звичайного файлу.

\newpage
\subsubsection{Приклад використання}

Щоб показати як користуватися КЕП, та прочитати атрибутивну інформацію з сертифікату,
який вшитий в PCKS-7 повідомлення з криптографічним підписом, покажемо 5 функцій:

\renewcommand{\footnotesize}{\tiny}

\begin{lstlisting}[utf8]
> CA.CAdES.readSignature
[
  {:certinfo, ~c"TINUA-2955020254",
   "СОХАЦЬКИЙ МАКСИМ ЕРОТЕЙОВИЧ",
   "МАКСИМ ЕРОТЕЙОВИЧ", "СОХАЦЬКИЙ",
   "СОХАЦЬКИЙ МАКСИМ ЕРОТЕЙОВИЧ",
   [
     subjectKeyIdentifier: "VNXfTvJQccGtPgNhUftIQZV+mUROTgroLotsbtYZsFE=",
     authorityKeyIdentifier: "XphNUm+C84/0vi5ABGgN/rOvysLkBHVNB9CuTISwfB0=",
     keyUsage: [<<6, 192>>],
     certificatePolicies: {"https://acsk.privatbank.ua/acskdoc",
      ["1.2.804.2.1.1.1.2.2", "1.3.6.1.5.5.7.2.1"]},
     basicConstraints: [],
     qcStatements: {"https://acsk.privatbank.ua",
      ["0.4.0.1862.1.1", "0.4.0.1862.1.5", "1.3.6.1.5.5.7.11.2",
       "0.4.0.194121.1.1", "1.2.804.2.1.1.1.2.1"]},
     cRLDistributionPoints: ["http://acsk.privatbank.ua/crl/PB-2023-S6.crl"],
     freshestCRL: ["http://acsk.privatbank.ua/crldelta/PB-Delta-2023-S6.crl"],
     authorityInfoAccess: [
       {"1.3.6.1.5.5.7.48.2",
        "http://acsk.privatbank.ua/arch/download/PB-2023.p7b"},
       {"1.3.6.1.5.5.7.48.1", "http://acsk.privatbank.ua/services/ocsp/"}
     ],
     subjectInfoAccess: [
       {"1.3.6.1.5.5.7.48.3", "http://acsk.privatbank.ua/services/tsp/"}
     ],
     subjectDirectoryAttributes: [
       {"1.2.804.2.1.1.1.11.1.4.7.1", "0"},
       {"1.2.804.2.1.1.1.11.1.4.1.1", "2955020254"}
     ]
   ], "ФІЗИЧНА ОСОБА", "", "", ~c"UA", "КИЇВ"},
  {:certinfo, ~c"UA-14360570-2310",
   "КНЕДП АЦСК АТ КБ \"ПРИВАТБАНК\"", "", "",
   "КНЕДП АЦСК АТ КБ \"ПРИВАТБАНК\"",
   [
     contentType: "0.6.9.42.840.113549.1.7.1",
     signingTime: "240221110356Z",
     messageDigest: "MfvlhoDVCPkptQRN+S2zNGp0nrOsS93mLdbcz/kZ9GI=",
     signingCertificateV2: 540041581425012649131508804155871837613877419268,
     contentTimestamp: {"1.2.840.113549.1.7.2",
      36995253346304402407284752111874897026, "20240221110626Z",
      "MfvlhoDVCPkptQRN+S2zNGp0nrOsS93mLdbcz/kZ9GI="}
   ], "АТ КБ \"ПРИВАТБАНК\"", "", "", ~c"UA", "Київ"}
]
\end{lstlisting}
\renewcommand{\footnotesize}{\normal}

\subsection{Криптографічні інформаційні повідомлення}

\section{Безпечна система доменних імен DNSSEC}

\section{Система директорії підприємства LDAP}

\section{Протокол розмежування доступу ABAC}

