\section{Формальна Йогачара}

\subsection*{Вступна частина}

У той час як буддизм Yogācāra досить добре відомий академічним дослідникам та спеціалістам
буддистських студій, він все ще в основному невідомий звичайним буддистам
азіатських країни, а також практикуючим буддистам та іншим неспеціалістам
студентам на Заході. Чому так? Перш за все, незважаючи на величезний вплив
Йогачари в період становлення буддизму махаяни в Індії школа вимерла разом
із буддизмом загалом, до кінця першого тисячоліття. в Тибеті, незважаючи
на свій вплив, в Школа ніколи не існувала як окрема традиція, а лише у
складі Абхідхарми (Трипітака) Пімапа Рінпоче (Gateway To Knowledge).
У Східній же Азії, Йогачара справді існувала як окрема традиція, але
для практичних цілей майже перестала мати будь-який великий вплив після
першого тисячоліття нашої ери.

Незважаючи на її остаточне зникнення як незалежної школи, Yogācāra як
вчення про карму, медитацію, пізнання та теорію шляху мали силу ---
повний вплив на інші школи махаяни, що розвинулися в той час імпорту
Йогачари до Тибету та Східної Азії, так що більшість технічної термінології,
на якій засновані інші школи махаянського дискурса, була поглинена з різних
напрямів Yogācāra.

Відсутність розвитку школи йогачари в Тибеті головним чином була
внаслідок того, що вона була поглинена новоствореним Тибетськими
доктринальними школами. у Східній Азії, з іншого боку, Йогачара
певний час існували як незалежна секта, відома китайською як Weishi (лише
свідомість) або Faxiang (характеристика дхарми). Але школа зрештою
вимерли перед різними формами конкуренції з (1) доктринальними школами,
чиї вчення вважалися більш резонансними зі східноазіатським світоглядом
і (2) більш масово орієнтовані школи, такі як Школи чистої землі та
медитації (чан/сон/дзен), які пропонували форму навчання та практики
набагато легше сприйняті звичайним людям мирянин віруючий.

Найбільша перешкода Yogācāra для набуття широкої популярності полягала
у складності її громіздкої системи точок зору, шляхів і категорії,
пояснені складною технічною термінологією. це, дійсно, вимагають досить
значного ступеня відданості з боку студента досягти рівня базового
розуміння, достатнього для читання та розуміння священних писаннь Йогачари.

Однак є деякі, хто стверджує, що ця передбачувана проблема здатності
до розуміння Йогачари також може значною мірою полягати в манері презентації,
що і стало основною мотивацією Тагава Шун'єі. Тобто, незважаючи на
здавалося б громіздку складність Система Yogācāra, про що говорять
майстри Yogācāra у багатьох кейсах --- це легко впізнавані повсякденні
переживання, якими ми всі ділимося. Багато моментів, на яких
зосереджувалися майстри Йогачари, були речами, які ми всі сприймаємо
як належне, але для якого, якщо розглянути більш детально, ми
насправді немає пояснень. і в більшості випадків --- я вважаю,
що ми можемо додати --- багато з цих питань, для яких дослідники
в таких галузях, як сучасна психологія, фізіологія, хімія і фізика
ще не мають відповіді.

Нас, звичайно, з дитинства вчили тому, що пам'ять зберігається десь
у мозку. Якщо це правда, то з мозок складається з фізичної матерії,
хіба це не так, як ми продовжуйте додавати інформацію, активність
мозку має збільшуватися, щоб тренувати це? Звичайно, ні. Але тоді
де вся ця концептуальна інформація, що зберігається, навіть не кажучи
про інформацію, що стосується тілесних активностей.

Очевидною відповіддю на це запитання є те, що ця інформація
зберігається десь у «свідомості». але якщо це так, то де це в
розумі великий обсяг інформації, що зберігається? І як ми знаємо,
що ми ні постійно втрачає інформацію? і якщо ми його зберігаємо,
як саме ми його отримуємо, коли нам це потрібно? Для більшості
відповідей, відповідь: «ну, ми точно не знаємо».

Для авторів школи Йогачара така відповідь не була прийнятною, і
тому вони прагнули через свої дослідження, дослідження та коншаблонні
техніки для надання деяких відповідей, а також широкий спектр суміжних,
і навіть більш фундаментальних питання.

На цьому етапі слід зазначити, що мотивація для Дослідники Yogācāra
були не просто творінням ранніх індійців буддійських еквівалент
сучасної когнітивної або поведінкової психології.

Асанга, Васубанду (Автори Йогачари) та їхні колеги були релігійними
мислителями, вимушено, через очевидні протиріччя та доктринальні
складності, властиві буддійським поясненням природи людського розуму,
зіставлення з процесами, які ведуть або до просвітлення, або до
глибшого захоплення через незнання та страждання --- пробували
знайти якісь рішення, які були би раціонально сприйняті. У процесі
розробки таких рішень (успадковуючи давню традицію філософії
перетворення розуму, надані попередніми вченими) їм довелося
зробити дуже серйозне дослідження, як саме ми знаємо речі, і як,
точно, наші тіла та розум змінюються та розвиваються. Маючи справу
з подібними проблемами, вони не могли не зіткнутися з тими ж проблемами,
з якими зустрічаються сучасні філософи, психологи і навіть еволюційні біологи.

І саме з цієї причини Йогачара прийшла у наш час аби привернули
інтереси різноманітних інтелектуалів, людей, чия робота лежить
поза сферою релігійної віри, які вивчають проблеми у пізнанні,
поведінці людини, розвитку особистості, тощо.

Врешті-решт, проблеми, з якими займаються йогачара --- це буддійські
проблеми, наскрізь, і, таким чином, зрозуміти мотивації, що стоять
за працями цих мислителів, мабуть, буде корисно надати короткий
огляд розвитку цих проблем.

\newpage
\subsection*{Насіння}

Я хотів би попередити, що будь-яка відповідність між п'ятьма ефектами/зернами
філософії Йогачари та п'ятьма теоріями суперструн є спекулятивною і не
базується на жодній загальноприйнятій чи усталеній теорії. Однак, якби
ми вели спекулятивне листування на основі роботи Девіда Фу, то отримали
би такі результати.

Формальна Йогачара є мненомічним альтернативним езотеричним
індексом-показчиком сучасної теоретичної математичної фізики
яка базується на еквіваріантній теорії гомотопій у відповідності
до тибетських термін школи Йогачари, яка збереглася в університеті
Наланди в Тибеті. Терміни йогачари базуються на ранніх перекладах
Сутр трипітаки, основні контрибютори Йогачари були брати Асанга та
Васубандху в Абхідхарма-коші та коментареві. Обширні коментарі на
Йогачару та Абхідхарму були залишені Міпамом Рінпоче в традиції Кама,
який вважається еманацією Манджушрі.

У цьому випуску даються мнемонічні недоказові відповідності між
поняттями йогачари та сучасної М-теорії (класифікація п'яти суперструнних теорій).

В коментарях Абхідхарма-коша-бхася про «саманартху» та «паратантру»
розказується в главі 5, Термін «васана» використовується в главі 6,
і термін «ніродха» в главі 7.

Терміни «саманартха» і «паратантра» вживаються у Махаяна-сутраламкара-каріці
у віршах 2.9-2.11, а термін «парінішпанна» вживається у вірші 2.14.
Термін «васана» не використовується в Mahāyāna-sūtrālamkāra-kārikā,
але це загальноприйнятий термін у філософії Йогачари. Термін «ніродха»
вживається в контексті припинення страждань у вірші 2.43.

\newpage
\subsubsection*{Samanartha}

\ti རང་བཞིན་གནས་དང་སྐོར་བར་དགོས་པ་ 
\\
\ua (rang bzhin gnas dang skor bar dgos pa)\\
\\
Саманартха відноситься до «взаємно підтримуючих» або «взаємопов'язаних»
аспектів існування, підкреслюючи взаємозалежність усіх явищ.

Той самий порядок (санскр. samanartha) --- відповідає теорії суперструн типу IIB,
яка є теорією суперсиметричних струн, які мають властивість самодуальності.
Властивість самодуальності означає, що той самий тип струни може трансформуватися
в інший тип струни за певних умов, вказуючи на те, що всі частинки та сили у
Всесвіті виникають із тих самих базових мод коливань струн.

\subsubsection*{Paratantra}

\ti གནས་སུ་སྡུད་པའི་གྲོས་པའི་སེམས་དང་བྱིན་རླབས་པ་  
\\
\ua (gnas su sdud pa'i gros pa'i sems dang byin rlabs pa)\\
\\
Паратантра термін відноситься до «залежної» або «відносної» природи всіх явищ,
які розглядаються як такі, що виникають в залежності від різних причин і умов.

Створено людиною (санскр. paratantra) --- відповідає теорії суперструн типу IIA,
яка також є теорією суперсиметричних струн. У цій теорії властивості та поведінка
частинок і сил залежать від контексту, в якому вони виникають, подібно до ідеї
паратантри у філософії Йогачари.

\newpage
\subsubsection*{Parinispanna}

\ti མངོན་པར་བྱེད་པ་  \ua (mngon par byed pa)\\
\\
Парініспанна термін стосується «повністю усвідомленої» або «просвітленої» природи
явищ, які розглядаються як за своєю суттю чисті та вільні від оман его та
дуалістичного мислення.

Неперевершеність (санскр. parinispanna) --- відповідає гетеротичній теорії струн $E_8 x E_8$,
яка описує остаточну реальність Всесвіту за допомогою математичних і груп симетрії, які
лежать в основі поведінки струн. Група симетрії $E_8 x E_8$ є найбільшою можливою групою симетрії,
яка може описати поведінку струн.

\subsubsection*{Vasana}

\ti བར་དོ་དགའ་བ་  \ua (bar do dga' ba)\\
\\
Васана термін стосується «звичних тенденцій» або «кармічних відбитків»,
які формують наше сприйняття та досвід, що призводить до прихильності та страждань.

Дозрівання (санскр. васана) --- відповідає теорії гетеротичних струн SO(32),
яка є теорією суперсиметричних струн, що враховує історію взаємодій і попередні
стани частинок і сил, подібно до ідеї васани у філософії Йогачара.

\newpage
\subsubsection*{Nirodha}

\ti རྣམ་པར་འདས་པ་  \ua (rnam par 'das pa)\\
\\
Цей термін стосується «припинення» або «знищення» страждання та причин страждання,
що є кінцевою метою практики Йогачари.

Нірвана (санскр. nirodha) --- відповідає теорії суперструн типу I,
яка містить відкриті і закриті струни, що описує остаточну природу реальності
через розуміння та споглядання математичних принципів, які керують поведінкою
струн. Цю теорію можна розглядати як більш просту та фундаментальну версію
теорій суперсиметричних струн, подібну до ідеї ніродхи у філософії Йогачари.

\subsection*{Формальна частина}

Думав про локальну гомотопічну теорію статтю написати ще і про
хроматичну теорію гомотопій та індексовані p-адичними числами К-теорії Морави.

Ну і ще про модальні категорії. Так само, як похідні категорії
є категорною семантикою когомологічних типів, так само локальна
теорія Жардіна і теорії Морави є категорною семантикою гомотопічних
типів. Вони є основними інструментами, що пропонують градуальний
спуск і локалізацію (ітеративний процес), у випадку когомології,
похідні категорії --- інформацію про виколоті многовиди, векторні
розшарування та інтегрування, а у випадку гомотопій, модальні
категорії та локальна гомотопічна теорія --- інформацію про топологію
простору та його гомотопічний тип, що теж обчислюється ітеративно. 

\newpage
\subsection*{Ādarśa-jñāna}

\ti མདང་སུ་གསོ་བ་ཤེས་རབ་གཞན་པོ་  \ua (mdangs su gso ba shes rab gzhan po)\\
\\
Найабстрактніша категорія, яка охоплює будь-які категорії логіки,
у тому числі топосо-теоретичні з локалізацією відкритих підпросторів
в підкласифікаторі об‘єктів є категорія комутативних діаграм --- воістину
найчистіший та найбстрактніший простір мислення, у якому написані (або закодовані)
усі математичні формули у сучасній (абстрактній) математиці. Це ---
найабстрактніший чистий листок, табула раса, відкритий гіперкуб, де
вершини --- це топоси, а стрілки --- геометричні морфізми (пара спряжених
функторів між топосами), простір для математичної творчості, необмежений
конкретними числами чи груповими представленнями. Саме на цьому «листку»
працює мислення математика. Це --- перший рівень абстрактності.

Інфініті категорії, групоїди, топоси, стеки. Поняття інфініті
категорій та інфініті групоїдів є основним та найбільшохоплюючим
в теорії категорій. Це такі багаторівневі категорії категорії, де
всі композиції у всіх всесвітах є когерентно-узгодженими. Категоріальний
аналіз включає у себе насупні стадії-конструктції від поняття самої
категорії до понятя спряженої трійки:
1) Категорії (С);
2) Функтори (F);
3) Природні перетворення (N);
4) Спряжені пари (P);
5) Спряжені еквівалентності (E);
6) Спряжені трійки (T); Загальна вежа:

$$
C \rightarrow F \rightarrow N \rightarrow P \rightarrow E \rightarrow T.
$$

Категорії діаграм. Нехай $C$ --- мала категорія, а $D$ --- будь-яка категорія.
Функтор $F: C \rightarrow D$ можна розглядати як діаграму в $D$ з індексом $C$.
Категорія діаграм в $D$ з індексом $C$, позначена $D^C$, має як об'єкти всі
функтори від $C$ до $D$, а як морфізми всі природні перетворення між цими
функторами. Тобто, для двох діаграм $F,G: C \rightarrow D$ морфізм від $F$
до $G$ є природним перетворенням $\alpha : F \rightarrow G$, яке пов'язує
кожному об'єкту $c$ у $C$ морфізм $\alpha_c : F(c) \rightarrow G(c)$ у $D$
таким чином, що для кожного морфізму $f: c \rightarrow c'$ у $C$ діаграма комутує:

\begin{center}
\begin{tikzpicture}
  \matrix (m) [matrix of math nodes,row sep=3em,column sep=3em,minimum width=3em]
  {
     F(C) & G(C) \\ % (1,1) (1,2)
     F(C') & G(C') \\ % (2,1) (2,2)
  };
  \path[-stealth]
    (m-1-1) edge node [above] {$\alpha_c$} (m-1-2)
    (m-1-1) edge node [left]  {$F(f)$} (m-2-1)
    (m-1-2) edge node [right] {$G(f)$} (m-2-2)
    (m-2-1) edge node [above] {$\alpha_{c'}$} (m-2-2);
\end{tikzpicture}
\end{center}

Локальні декартово-замкнені категорії. Категорії зі скінченними лімітами,
де слайс-категорії по будь-якому об'єкту є декартово-замкненими називаються LCCC,
та моделюють фібраційне лямбда-числення як внутрішню мову таких категорій.

Симетричні моноїдальні категорії. Моноїдальні категорії мають додаткову
структуру яка складається з: 1) тензорного добутку $\otimes: M \times M \rightarrow M$;
2) об'єкту одиниці; 3) природнього-перетворення «асоціатора»
$a_{x,y,z}: (x\otimes y)\otimes z \rightarrow x\ptimes(y \otimes z)$;
4) правила трикутника та п'ятикутника.

Сплетена моноїдальна
категорія --- це категорія з додатковим природнім перетворенням
яке називається «сплетіння» $B_{x,y} : x\otimes y \rightarrow y\otimes x$,
таким, що комутують дві діаграми:

Слетена моноїдальна каегорія, сплетення якої задовільняє
рівняння $B_{x,y} = B^{-1}_{y,x}$ для всіх об'єктів $x,y$ називається
симетричною моноїдальною категорією.

\subsection*{Samatā-jñāna}

\ti འདུས་པ་ཤེས་རབ་གཞན་པོ་ \ua (dus pa shes rab gzhan po)\\
\\
Потім з цього листка ми ітеративно інстанціюємо спочатку абстрактні
алгебраїчні інструменти, такі як прикладні теорії категорій орисані
вище, або шість йог Гротендіка, симетричні моноїдальні категорії,
модальні спряження (трійки), тощо. Це --- другий рівень «абстрактності».

Абелеві категорії --- це збагачене поняття категорії Сандерса-Маклейна
поняттями нульового обʼєкту, що одночасно ініціальний та термінальний,
властивостями існування всіх добутків та кодобутків, ядер та коядер, а
також, що всі мономорфізми і епіморфізми є ядрами і коядрами
відповідно (тобто норомальними).

Похідні категорії. Похідна категорія --- це конструкція в гомологічній
алгебрі, яка пов'язує з даною категорією комплексів нову категорію,
в якій ідентифікуються квазіізоморфні комплекси. Зокрема, задана
категорія C комплексів над комутативним кільцем R, будується нова
категорія, позначена D(C), об'єкти якої такі ж, як об'єкти C, але
в якій морфізми визначені більш гнучким способом. Зокрема, морфізми
в D(C) --- це класи еквівалентності діаграм морфізмів у C, які
називаються квазіізоморфізмами, які задовольняють певним аксіомам.
Ці аксіоми включають існування певних виділених трикутників і
сумісність із прямими сумами.

Модельні категорії. Категорія моделі Квіллена --- це категорія,
оснащена певною структурою, яка дозволяє створити гомотопічну
теорію об'єктів у цій категорії. Більш конкретно, це категорія
з класом слабких еквівалентностей, класом розшарувань і класом
корозшарувань, які задовольняють набір аксіом, відомих як аксіоми
категорії моделі Квіллена. Слабкі еквівалентності в категорії моделі
Квіллена є класом відображень, які індукують ізоморфізми на певних
гомотопічних групах. Розшарування є класом відображень, які
задовольняють певну властивість підняття щодо корозшарувань,
тоді як корозшарування є класом відображень, які задовольняють
певну властивість підняття щодо розшарувань.

Категорії спектрів. Спектр --- це послідовність точкових
просторів $X_0, X_1, X_2$, ... разом із відображеннями $\sigma_n : X_n \rightarrow \Sigma X_{n+1}$,
де $\Sigma$ позначає суспензію простору, що задовольняє деяким аксіомам,
які гарантують, що $\sigma_n$ узагальнюють звичайні карти
суспензій. Морфізм спектрів $f: X \rightarrow Y$ є послідовністю
відображень $f_n : X_n \rightarrow Y_n$, які комутують із
відображеннями суспензій в тому сенсі, що комутує діаграма:

\newpage
\begin{center}
\begin{tikzpicture}
  \matrix (m) [matrix of math nodes,row sep=3em,column sep=3em,minimum width=3em]
  {
     X_n & \Sigma X_{n+1} \\ % (1,1) (1,2)
     Y_n & \Sigma Y_{n+1} \\ % (2,1) (2,2)
  };
  \path[-stealth]
    (m-1-1) edge node [above] {$\sigma_n$} (m-1-2)
    (m-1-1) edge node [left]  {$f_n$} (m-2-1)
    (m-1-2) edge node [right] {$\Sigma f_{n+1}$} (m-2-2)
    (m-2-1) edge node [above] {$\sigma_{n}$} (m-2-2);
\end{tikzpicture}
\end{center}

T-Спектри та досніпи спектрів. Для потреб стабільного теорії
гомотопії та для схемного узагальнення різних видів спектрів:
Адамса, Ейленберга-Мак Лейна, К-теорій, в тому числі для потреб
$\mathbb{A}^1$-теорії гомотопій, локальної гомотопічної теорії, тощо.

Функторіальні йоги. Перші шість йог Гроендіка визначають наступні
спряжені функтори: геометричні морфізми, похідного функтора образу
прямого пучка та оберненого образу, дуальності Верд'є, та дуальне
спряження тензорного добутку в симетричних моноїдальних категоріях
та внутрішнього функтора Hom. Такі йогічні вправи дали змогу
побудувати похідні категорії, та виділити системи коефіцієнтів
і їх узагальнення: $D(X,\mathbb{Q})$ --- $l$-адичні пучки і $l$-адична
когомологія; $D(X(\mathbb{C}),\mathbb{Z})$ --- аналітичні
пучки і когомології Бетті; $D(\mathit{D}_X)$ --- голономічні $D\mathit{D}$-модулі
і когомології де Рама; $D(Сoh(X))$ --- когерентні пучки
та когерентні когомології; $DM(X)$ --- мотивні пучки і
0-зважені мотивні когомології; $SH(X)$ --- стабільні мотивні
гомотопічні пучки і стабільні мотивні 0-зважені групи когомотопій.

Функторіальні йоги і пошуки спряжень Ловіра відкривають наступні
спряженя в інфініті зв'язаних топосах, або спряжені трійки $(F \dashv G \dashv H)$:
1) $\exists \dashv f^{-1} \dashv \Pi$;
2) $\Sigma \dashv f_\star \dashv \exists$;
3) $\Sigma \dashv f_\star \dashv \Pi$;
4) монадичні і комонадичні дуальності;
5) спряження Квілена;
6) спряження Фробеніуса;
7) $\sh \dashv \fl \dashv \sharp$;
8) $\Pi \dashv Disc \dashv \Gamma \dashv coDisc$;
9) $\Re \dashv \Im \dashv \&$;
10) $\rightrightarrows\ \dashv\ \rightsquigarrow\ \dashv Rh$. 

\newpage
\subsection*{Pratya-veksanā-jñāna}

\ti རང་གི་ཤེས་རབ་གཞན་པོ་ 
\\
\ua (rang gi shes rab gzhan po)\\
\\
Потім самі теорії, які виражується вже як інстанції формул попереднього,
другого рівня абстрактності, наприклад теорії (ко-)гомологій та (ко-)гомотопій.
Це --- третій рівень «абстрактності».

Теорія гомотопій. Категорія моделей для теорії гомотопій Ho(С) ---
це категорія, де об'єкти це гомотопічні типи, а морфізми це гомотопії
між цими типами. Категорія моделей має зв'язок з теорією гомотопій за
допомогою забуваючого функтора, який переводить кожен гомотопічний тип
на його модель. Теорія гомотопій на C --- це категорія моделей для теорії
гомотопій Ho(C), де Ho(C) є гомотопічною категорією, отриманою шляхом
локалізації категорії C з морфізмами, які стають ізоморфізмами в категорії
гомотопічних функторів.

Гомологічна алгебра. Алгебраїчна структура, яка взаємозв'язана з
топологічною теорією і використовується для вивчення властивостей
топологічних просторів. У категорійному підході, гомологічна алгебра
визначається як об'єкт у категорії кольцевих об'єктів в категорії модулів
над деякою категорією або топологічним простором. Зазвичай, це є категорія
модулів над комутативним кільцем.

Алгебраїчна геометрія. Формально, алгебраїчна геометрія може бути
визначена як вивчення категорії афінних алгебраїчних множин та алгебраїчних
відображень між ними. Ця категорія побудована на основі категорії
комутативних алгебр над заданим полем. Алгебраїчна геометрія вивчає
структуру цієї категорії, таку як топологію, властивості ідеалів та
їх геометричні властивості, а також вивчає властивості алгебраїчних
відображень, таких як внутрішні гомоморфізми та проективні відображення.

Диференціальна геометрія. Функтор який приписує кожному гладкому
многовиду M комутативну алгебру H(M), яка залежить від топологічних
та диференціальних властивостей многовиду і зберігає зв'язок між гладкими
відображеннями многовидів та алгебрними гомоморфізмами між відповідними
комутативними алгебрами називєься кобордизмом. Цей функтор відображає
категорію гладких многовидів у категорію комутативних алгебр,
які звуться алгебрами Хопфа.

Категоріальна супергеометрія. Або теорія супер формальних гладких інфініті групоїдів на
супер формальних Картанових просторах:

\subsection*{Krtyānusthāna-jñāna}

\ti བྱེད་པར་ཕྱིན་པའི་ཤེས་རབ་གཞན་པོ་ 
\\
\ua (byed par phyin pa'i shes rab gzhan po)\\
\\
Потім йде четвертий рівень «абстрактності» --- прикладні теорії з
просторами та алгебрами, або бозонно-ферміонними суперпросторами
та супераогебрами Лі. Теорія суперструн, квантова топологічна теорія поля.

Топологічна квантова теорія поля. Топологічна квантова теорія поля,
яка є розділом математичної фізики, має справу з математичною структурою
топологічних просторів.

Супергеометрія М-теорії: не-пертурбативне доповнення всіх струнних теорій.

\newpage
\subsection*{Йогачара і Математика}

В тибетському буддизмі основними текстами по Йогачарі вважуються тескти Міпама Рінпоче.
Згідно канону, свідомості \ti རྣམ་ཤེས  \ua (rnam shes) Йогачари, яких є 8, визначаються так:
\\
\\
\footnotesize
\noindent 1) \ti མིག་ \ua (mig, око);\\
2) \ti རྣ་ \ua (rna, вухо);\\
3) \ti སྣ་ \ua (sna, ніс);\\
4) \ti ལྕེ་ \ua (lce, смак);\\
5) \ti ལུས་ \ua (lus, тіло);\\
6) \ti ཡིད་ \ua (yid, логіка);\\
7) \ti ཉོན་ཡིད་  \ua (nyon yid, алогіка);\\
8) \ti ཀུན་གཞི་  \ua (kun gzhi, основа).\\
\normalsize

8-й вид мислення. Усі конструкції наведені в цій статті розповсюджуються на 4 рівня
очищення само-усвідомлення для 8-ї свідомості алая-віджняни. Якщо мнемонічно кодувати,
або езотеризувати цю модель, то вийде: 1. Основа примордіального (табула раса);
2. Основа звільнення (другий рівень абстрактності); 3. Основа загального (геометричні теорії);
4. Основа ілюзій (теорія суперструн, або нісіння причини та наслідку).
\\
\\
\footnotesize
\noindent 1) Гедоністичний (pramuditābhūmi; \ti རབ་ཏུ་དགའ་བ་,  \ua rab tu dga' ba).\\
2) Міцний (vimalābhūmi; \ti དྲི་མ་མེད་པ་, \ua dri ma med pa).\\
3) Ілюмінуючий (prabhākarībhūmi; \ti འོད་བྱེད་པ་, \ua 'od byed pa).\\
4) Радіантний (arcismatībhūmi; \ti འོད་འཕྲོ་ཅན་, \ua 'od 'phro can).\\
5) Незворотній (sudurjayābhūmi; \ti ཤིན་ཏུ་སྦྱང་དཀའ་བ་, \ua shin tu sbyang dka' ba).\\
6) Ясний (abhimukhībhūmi; \ti མངོན་དུ་གྱུར་བ་, \ua mngon du gyur ba).\\
7) Прогресуючий (durangamabhūmi; \ti རིང་དུ་སོང་བ་, \ua ring du song ba).\\
8) Непорушний (acālabhūmi; \ti མི་གཡོ་བ་, \ua mi g.yo ba).\\
9) Неперевершений (sādhumatībhūmi; \ti ལེགས་པའི་བློ་གྲོས་, \ua legs pa'i blo gros).\\
10) Хмари Дхарми (dharmameghaābhūmi; \ti ཆོས་ཀྱི་སྤྲིན་, \ua chos kyi sprin).\\
\normalsize

В тибетській традиції Йогачари, чотири з останніх 10 бхумі шляху Бодхісатви відповідають
чотирьом рівнями само-усвідомлення Алая-віджняни: 8-й вид мислення це онтолічний мисленний
процес який покриває усі можливі абстрактні і прикладні математики, тому його модель ---
це одна з можливих індексацій бібліотеки формальної математики.

7-й вид мислення це мислення з bottom, яким може скічитися довільне обчислення довільного морфізму,
або мислення з парадоксами, або мислення з Fixpoint, самовимотуючий процес мислення. Формальні
теорії з парадоксами. Розглядається вбудовування в категорях повних частково-впорядкованих множин.
В сутності --- це всі мови з неконтрольованую рекурсію, теорії рекурсивних функцій, тощо.
Більшість неформальних (просто формалізуємих) інтерпритаторів з неконтрольованую рекурсією
які не-реалізують Тюрінг повноту, через обмеженість ресурсів, також класифікують як вимотуюче
мислення. Більшість помилок в індустрії, а також в теорії паралельних обчислень на наявність
неконтрольованих циклів, які в граничних помилкових ситуаціях та споживають значну кількість
ресурсів. 7-й вид мислення моделюється коіндукивними процесами, паралельними процесами, тощо.

6-й вид мислення або sems sdе, звичайний гільбертовий обчислювач, або топос Гротендіка.
CoC, MLTT --- теж відносяться до 6-го типу мисленння. Чисті формальні теорії без парадоксів.
Хоча декартово-замкнені категорії та симетричні моноїдальні категорії представлені як чотири
ступені 8-го виду мислення, можна згадати їх тут, так як вони є моделями самоусвідомлення
базового мислення. Яке з приходом наступного рівня зустрічається за парадоксами, які можуть
стати на заваді до 8-го виду мислення. 6-й вид мислення --- це фібраційна ясність просвітлення.

5 когнітивних мисленнь які маніфестуються в Йогачарі як (оціночні) дуальності та мандала
п'яти сімейств. Активно проаналізовані можливості текстуальної генерації, відео та аудіо
генерації та впроваджені вигляді публічних натренованих когнітивних сервісів. В основному
використовується лінійна алгебра, теорія груп, топологічний аналіз даних, теорія нейромереж.

\subsection*{Кон'юнктура}

Немає математики за межами цієї вежі, немає мислення за межами цієї вежі, немає просторів
за межами цієї вежі. Можливо, навіть, це спряжена трійка:
\begin{center}Простір $\dashv$ Мислення $\dashv$ Математика\end{center}
