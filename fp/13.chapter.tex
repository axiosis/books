\section{Суперпростір}

В цій статті у формі посилань на статті дається визначення суперточки,
суперлінії, суперсфер та супермноговидів, їх розшаруваня та когомології.
Слово «супер» означає $\mathbb{Z}_2$ градуйовані супералгебри Лі, які містять два
вектори координат, з парними та непарними індексами та використовуються в теорії
суперструн. Показується еволюція емерджентного суперпростору та всіх його струнних
теорій (Type I, IIA, IIB, SO(32), $E_8\times E_8$).

\subsection*{Супералгебри Лі}

Супер-алгебри Лі як необхідний пререквізит суперсиметричних
бозонно-ферміонних геометрій. Категорно, супералгебра Лі --- це
внутрішній об'єкт в симетричній моноїдальній категорії (SMC) $\mathbb{Z}_2$
градуйованих суперпросторів $sVect = (Vect_{\mathbb{Z}_2},\otimes_k, \tau, \textgoth{g})$.
Морфізми в цих категоріях --- дужки Лі, такі що:
1) $[a,b] = -(-1)\alpha\beta[b,a]$;
2) $[a,[b,c]]=[[a,b],c]+(-1)\alpha\beta[b,[a,c]]$.
$\alpha\beta \in \mathbb{Z}_2, a \in V_𝟷, b \in V_2, Vect_{\mathbb{Z}_2}=V_𝟷\otimes V_2$.

\subsection*{Суперсфери та розшарування Хопфа}

Крім суперточки в $\mathbb{Z}_2$ градуйованих, [точніше над градуйованими
просторами які визначаються прямим декартовим добутком цілих чисел]
суперсиметричних алгебрах (супералгебрах) Лі нас будуть цікавити
класичні розшарування Хопфа, та суперсфери з їх використанням.
При цьому форма точних послідовностей які визначають розшарування
не змінюються, але змінюються їх когомології.

Так, в нас є проблема з відсутністю $SL(2)$ на октаніонах,
тому в супергеометрії ми використовуємо накриваючі групи
спінів лоренцевих груп в сигнатурі Мінковського (9,1)
для останнього розшарування Хопфа.

\subsection*{Супермноговиди}

Окрім суперточки, суперлінії, суперсфер, нас будуть цікавити
також супермноговиди довільної форми.

\subsection*{Когомології де Рама}

За теоремою де Рама існує ізоморфізм між групами когомологій
де Рама $H_{dR}^k(M)$ та групами когомологій $H^k(M;\mathbb{R})$
для будь-якого гладкого многовида. З комутативної діаграми
ізоморфізмів випливає, що спектральні послідовності, що
ґрунтуються на гомологічних групах сфер, можуть бути
адаптовані до суперсфер після спеціалізації на конкретному полі коефіцієнтів.

\subsection*{Когомології Шевал'є-Ейленберга}

Традиційно перед визначенням когомологій типу Шевал'є-Ейленберга,
спочатку дають визначення алгебрам Шевал'є-Ейленберга $CE(\textgoth{g})$
як супералгебрам Грасмана в дуальному суперпросторі $\wedge\ \partialvartoint\ \textgoth{g}*$,
що має диференціал $d\textgoth{g} := [\_,\_]* : \textgoth{g}* \rightarrow \textgoth{g}* \wedge\ \textgoth{g}*$,
що розширюється на весь дуальний простір за допомогою
градуйованості Лейбніца.

\subsection*{Когомології BRST}

У фізиці, супералгебри Шевал'є-Ейленберга $CE(\textgoth{g}, N)$ дії алгебри
Лі або L-$\infty$ алгебри групи калібрування G на простір полів N називається
BRST комплексом на честь Беккі, Руе, Стора, Тютіна.

\newpage
\subsection*{Емерджентний суперпростір}

Суперточка $\mathbb{R}^{0|N}$ визначається для всіх $N$.
Суперточка $\mathbb{R}^{0|1}$ має природнє розширення до
суперлінії $\mathbb{R}^{1|1}=\mathbb{R}^{1,0|1}$.
Максимальний інваріант центрального розширення суперточки
$\mathbb{R}^{0|2}$ є трьохвимірна сумер-алгебра Мінковського $\mathbb{R}^{2,1|2}$.

Далі простір розвивається по сферам згідно конструкції Келі-Діксона
та розшарувань Хопфа, набуваючи свого повного змісту у обʼєднуючій М-теорії:
\\
\\
1). $\mathbb{R}^{2,1|2} \rightarrow \mathbb{R}^{2,1|2+2}$;\\
2). $\mathbb{R}^{3,1|4} \rightarrow \mathbb{R}^{3,1|4+4}$;\\
3). $\mathbb{R}^{5,1|8} \rightarrow \mathbb{R}^{5,1|8+8}$;\\
4). $\mathbb{R}^{9,1|16} \rightarrow \mathbb{R}^{9,1|16+16}$.\\
\\
$IIB \rightarrow \mathbb{R}^{9,1|16+16} \leftarrow \mathbb{R}^{9,1|16} \rightarrow \mathbb{R}^{9,1|16+16} \leftarrow IIA$.
Максимальний інваріант центрального розширення простору
Мінковського IIA типу $\mathbb{R}^{9,1|16+16}$ є $\mathbb{R}^{10,1|32}$ --- 11-вимірна
М-теорія з тридцятьма двома додатковими ферміонними параметрами.
