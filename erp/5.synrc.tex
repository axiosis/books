\chapter{Технологічний рівень зв'язності людей та пристроїв}

\section{Вступ}

Архітектурна компанія SYNRC розробляє та підтримує систему автоматизації підприємства N2O.DEV,
побудовану згідно формальної специфікації яка призначена для розробки багатофункціональних
гетерогенних платформ для додатків та сервісів на основі шини та розподіленої бази даних.
N2O.DEV уже використовується у банках, системах повідомлень, державних підприємствах та
інших менших чисельних організаціях в Америці, Європі та Азії.
\\
\\
Друге видання КНИГИ N2O (англ. N2O BOOK Vol. 2 Green Book) визначає форамальну специфікацію
на програмне забезпечення усіх рівнів моделі Закмана для підприємств ISO-42010, містить
широкий спектр прикладів, розказує про складові компоненти та є вичерпним авторським
стартовим посібником для курсу навчання розробки технологічних програм для платформи Erlang.

\section{Виробничий процес}

\subsection{Середовище}

Для забезпечення повного замкненого середовища пропонують наступні заміни бібліотек kernel та stdlib:

\setmainfont{Segoe UI Emoji}🚀\setmainfont{Geometria}\ --- віртуальна машина середовища виконання

\setmainfont{Segoe UI Emoji}⭐\setmainfont{Geometria}\ --- базова системна бібліотека як заміна stdlib

\setmainfont{Segoe UI Emoji}⭐\setmainfont{Geometria}\ --- бібліотека середовища виконання як заміна kernel

\setmainfont{Segoe UI Emoji}📨\setmainfont{Geometria}\ --- бібліотека PubSub для розподілених систем


\newpage
\subsection{Бібліотеки}

Для запезпечення повноцінної промислової специфікації ERP.UNO, ми розширили
набір інструментальних засобів наступними бібліотеками: формальними представленнями
презентаційного рівня FORM та системою управління бізнес-процесів BPE. FORM представляє
собою декларативну бібліотеку побудови графічних інтерфейсів, а бібліотека BPE
підтримує XML файли стандарту BPMN 2.0 та реалізує безпосередню інтерналізацію
BPMN семантики у семантику віртуальної машини Erlang.

Ядро бібліотек які реалізують фундацію N2O.DEV (організація SYNRC) для системи
управління підприємствами ERP.UNO (організація ERPUNO) у версії 8.0 виглядає
наступним чином:

\setmainfont{Segoe UI Emoji}⭕\setmainfont{Geometria}\ --- сервер протоколів для стандартів MQTT, WebSocket, QUIC

\setmainfont{Segoe UI Emoji}🔥\setmainfont{Geometria}\ --- веб-фреймворк Nitrogen та його контрольні елементи UI

\setmainfont{Segoe UI Emoji}💿\setmainfont{Geometria}\ --- бібліотека доступу до KV сховищ RocksDB та SpanDB/NVMe

\setmainfont{Segoe UI Emoji}🧾\setmainfont{Geometria}\ --- бібліотека декларативного конструювання іформ

\setmainfont{Segoe UI Emoji}💠\setmainfont{Geometria}\ --- сисема управління процесами стандарту BPMN 2.0

\setmainfont{Segoe UI Emoji}☎\setmainfont{Geometria}\ --- бібліотека генерації SDK для мов JS, protobuf, Swift

\subsection{Приклади}

Головні приклади фундації N2O.DEV присвячені наступним темам: MQTT та WebSocket
чати для демонстрації веб-фреймворку NITRO, який працює як модуль N2O, приклад
REST адаптер до бази даних KVS, та повністю чистий N2O додаток CHAT на основі
бібліотеки SYN без використання NITRO:

\setmainfont{Segoe UI Emoji}💧\setmainfont{Geometria}\ --- ідіоматичний приклад Nitrogen поверх WebSocket

\setmainfont{Segoe UI Emoji}💧\setmainfont{Geometria}\ --- ідіоматичний приклад Nitrogen поверх MQTT

\setmainfont{Segoe UI Emoji}☕\setmainfont{Geometria}\ --- бібліотека для побудови HTTP API

\setmainfont{Segoe UI Emoji}💬\setmainfont{Geometria}\ --- ідіоматичний приклад системи доставки повідомлень

\subsection{Інструменти}

Цей посібник більше присвячений бібліотекам N2O та NITRO, та лише незначним
чином торкається сховища даних KVS. Пізніше нами було проінвестовано ще у
наступні бібліотеки, які в основному стосуються створення API та керування
пакетами та інстансами:

\setmainfont{Segoe UI Emoji}⚡\setmainfont{Geometria}\ --- бібліотека управління пакетами та інстансами

\setmainfont{Segoe UI Emoji}📁\setmainfont{Geometria}\ --- бібліотека управління файловою системою

\setmainfont{Segoe UI Emoji}🐋\setmainfont{Geometria}\ --- бібліотека перекомпіляції

\newpage
\subsection{Ресурси}

Концептуальна модель системи в рамках якої функціонує N2O визначаена як обчислювальне середовище,
яке складається з процесору подій (N2O), операційного (ETS) та персистентного сховища (KVS).
З точки зору обчислювального середовища, ресурси підприємства складаються з глобального
 сховища та обчислень, які розділяють глобальну адресацію та представляють собою
Erlang-процеси (N2O протоколи). Кожен процес PI, може містити певний набір протоколів,
будь-який з яких відповідає на певний набір повідомлень. Протоколи N2O визначені на
точці підключення повинні не перетинатиcя, в іншому випадку протокольні модулі можуть
перехоплювати та впливати на інші протокольні модулі, які повинні реагувати на той
самий тип повідомлень.

Усі асинхронні процеси PI запускаються під головним супервізором n2o та індексуються
URI ключем разом з типом реактивного каналу реального часу: ws або mqtt. N2O протоколи
підключені безпосередньо до веб-сокет точок підключення виконуються в контексті TCP
процесів, у даному випадку TCP-сервера бібліотеки RANCH, супервізор ranch_sup.

\begin{lstlisting}
> :supervisor.which_children :n2o
[
  { {:ws, '/chat/ws/4'}, <0.985.0>, :worker, [:n2o_ws] },
  { {:ws, '/chat/ws/3'}, <0.984.0>, :worker, [:n2o_ws] },
  { {:ws, '/chat/ws/2'}, <0.983.0>, :worker, [:n2o_ws] },
  { {:ws, '/chat/ws/1'}, <0.982.0>, :worker, [:n2o_ws] },
  { {:mqtt, '/bpe/mqtt/4'}, <0.977.0>, :worker, [:n2o_mqtt] },
  { {:mqtt, '/bpe/mqtt/3'}, <0.976.0>, :worker, [:n2o_mqtt] },
  { {:mqtt, '/bpe/mqtt/2'}, <0.975.0>, :worker, [:n2o_mqtt] },
  { {:mqtt, '/bpe/mqtt/1'}, <0.974.0>, :worker, [:n2o_mqtt] },
  { {:caching, 'timer'}, <0.969.0>, :worker, [:n2o] }
]
\end{lstlisting}

Для відображення усіх таблиць (префіксів) які існують в глобальному
просторі ключів, скористайтеся системним фідом writer.

\begin{lstlisting}

> :kvs.all :writer
[
  {:writer, '/bpe/proc', 2, [], [], []},
  {:writer, '/erp/group', 1, [], [], []},
  {:writer, '/erp/partners', 7, [], [], []},
  {:writer, '/acc/synrc/Kyiv', 3, [], [], []},
  {:writer, '/chat/5HT', 1, [], [], []},
  {:writer, '/bpe/hist/1562187187807717000', 16, [], [], []},
  {:writer, '/bpe/hist/1562192587632329000', 1, [], [], []}
]
\end{lstlisting}

\subsection{Erlang та сучасний веб}

Erlang реалізує недосяжну мрію кожного обчислювального середовища для
паралельної та узгодженої конкуретної обробки повідоблень. Так найбільш
відомі бібліотеки акторів (Akka, Orleans), які реалізують основні примітиви:
процесори та черги, копіюють модель акторів Erlang, зазвичай намагаються
також реалізують додатково механізми перезапуску та супервізії процесів
подібно до Erlang, проте тільки Erlang забезпечує soft real-time характеристики,
завдяки керуванню латенсі з точністю до таймінгу команд віртуальної машини.
А з виходом 24 версії в 2020 році, яка почала підтримувати JIT-компіляцію
завдяки asmjit, продуктивність та чуттєвість віртуальної машини зрозла
ще більше.

З формальної точки зору достатньо добре ізольоване середовище віртуальної
машини Erlang не тільки забезпечує характерстики реального часу для
SMP-планувальника легких зелених процесів, але і обмежує область видимості
heap пам'яті виключно для процесів-власників, що унеможливлює вплив відмови
певних процесів на глобальний стан віртуальної машини.

Erlang ідеально підходить для побудови високо-навантажених,
просто-масштабованих, подійно-орієнтованих, неблокуючих, надійних,
постійно-доступних, високо-ефективних, швидких, безпечних та надійних
систем обробки повідомлень та розподілених у просторі та часі систем.

\subsection{DSL vs Шаблони}

З технічної точки зору N2O успішно показує неперевершену досі якість
DSL програмування, яку ви не зможете знайти в сучасних веб-фреймворках
для мов Erlang та Elixir. За 7 років неперервної еволюції N2O ми переписали
кожен з 700 рядків по 30 разів, якшо порахувати через коміти Github.
Веб-фреймворк NITRO, сховище KVS, та BERT.JS кодування може забезпечити
відображення в веб-браузері повноекранних вертикальних форм з усіма
обчислюваними полями зі швидкістю 60 форм в секунду по веб-сокет каналу.
А надзвичайно компакта JavaScript бібліотека-компаньйон вміщується
в 4 MSS/MTU вікна — саме такий розмір мінімального веб-клієнта з BERT
кодуванням, який повністю управляється зі сторони сервера.

N2O сервер та веб-фреймворк NITRO реалізують концепцію не тільки
управління сесіями та каналами, але і усім стеком побудови додатків
включаючи UI частину, як це відбувається у таких веб-фреймворках як
Erlang Nitrogen, OCaml Ocsigen, Scala Lift, F# WebSharper, а завдяки
таким розширенням як FORM та BPE ідеально підходять і для побудови
автоматизованих CRM систем.

Це не означає, що за допомогою N2O ви не можете створювати більш
класичні та архаїчні додатки у стилі DTL шаблонізаторів, або як це
відбувається у таких фреймворках як PHP, ASP, JSP, Rails, тощо.
Перші версії NITRO містили в прикладах використання Django Template
Library (DTL), проте задля чистоти стеку були прийнято не включати
в N2O додаткові шаблонізатори крім NITRO DSL.

\section{Інтерфейс NITRO}

\section{Сховище KVS}

\section{Логіка BPMN}



\section{Додатки MQTT та WebSocket}

