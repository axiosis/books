\chapter{Державна система}

По аналогії зі стандартом ISO 42010 «Фреймворку Закмана»,
фреймворк Максима Сохацького визначає та уточнює архітектурні рівні
з яких складаються сучасні корпоративні інформаційні системи:
\\
\\
--- Юридично-документальний рівень \\
--- Обліково-реєстровий рівень \\
--- Зв'язність людей та пристроїв \\
--- Генерація, валідація і верифікація \\
--- Телекомунікаційна платформа і безпека інтернету\\

\section{Юридично-документальний рівень}

Згідно фреймворку верхній шостий рівень визначає BPMN процеси згідно яких здійнюється
відзеркалення юридично-правивих відносин електронного документообігу. Кожен крок такого
процесу, та усі його документи підписуються особистим ключем КЕП посадової особи, що дає
змогу проведення диспутів та розслідувань Міністерством юстиції України. Окрім того цей
рівень системи орієнтований на аналітику у взаємодії з громадянами через СЕВ ОВВ.
\\
\\
Юридично-документальні системи ERP/1 будуються на сховищі з єдиним
простором ключів Facebook RocksDB, що здатне працювати через Intel SPDK на NVMe
дисках, наприклад у складі таких сховищ як CEPH. Обсяг обігу документів на великих
підприємствах сягає 1ТБ на рік.

\newpage
\section{Обліково-реєстровий рівень}

Обліково-реєстровий рівень пропонує низькорівневе масштабоване розподілене
журнальне сховище даних та метаданих, яке може бути побудоване на реляційних
базах даних, базах даних з єдиним простором ключів з гарантіями
консистентності (chain-hash) або їх комбінаціях.
\\
\\
Класичні представники цього рівня в системах управління підприємствами: система
управління людськими та матеріальними ресурсами, банківські системи PCI DSS,
складські системи, медичні інформаційні системи, системи управління поставками
та виробництвом, системи сервісних послуг, системи управління проектами, тощо.

\newpage
\section{Технологічний рівень зв'язності людей та пристроїв}

\subsection{Локальний}
Рівень зв'язності людей та пристроїв визначає комунікаційні протоколи та технології,
які об'єднують головні ресурси підприємства (пристрої та людей) у одну
телекомунікаційну мережу. Як правило виробництво складається з багатьох пристроїв
що підключаються до промислових шин як MQTT, та робочих місць користувачів,
каналів зв'язку з інформаційними системами, корпоративні та національні шини, тощо.

Цей рівень також визначає засоби мастшабування пам'яті (персистентної та волатильної) та
обчислювальних ресурсів (за допомогою процесінгових брокерів доставки повідомлень).
Це рівень визначає реляційні бази даних та бази даних з єдиним простором ключів,
а також стандарти та протоколи передачі інформації у промислових ERP системах,
такі як CSV, JSON, SOAP, BERT, ASN.1, тощо.

\subsection{Крос-системний}
Крім того Мінцифра підтримує два середовища інтеграційної взаємодії національного рівня,
учасниками яких є суб'єкти господарювання індексовані ЄДРПО підприємства:

1) національна система електронної взаємодії органів виконавчої влади (СЕВ ОВВ)
   з відкритим ринком клієнтів\footnote{\url{https://se.diia.gov.ua/sedlist} — Перелік сертифікованих систем документообігу}
   і широким охопленням органів виконавчої влади\footnote{\url{https://se.diia.gov.ua/uploads/documents/45.xlsx} — Перелік систем документообігу і їх ЄДРПО підключених до національної шини СЕВ ОВВ}.
   Ця шина діє на рівні юридично-документального рівня і безпосередньо пов'язана з серверами
   документообігу, які є учасниками цієї взаємоії;

2) національна система електронної взаємодії державних електронних інформаційних
   ресурсів\footnote{\url{https://catalog.trembita.gov.ua/?env=SEVDEIR} — Каталог сервісів <<Трембіта>>} (СЕВДЕІР <<Трембіта>>),
   яка представляє собою спеціалізовану версію Ubuntu з пакетами X-ROAD,
   власною інфраструктурою CA, TSP, OCSP серверами і шифрованими каналами
   передачі конфіденційної інформації.

\newpage
\section{Генерація, валідація і верифікація}

Рівень схеми даних визначає модель зберігання даних як з точки зору об'єктів-сутностей
так і з точки зору технологій та протоколів, які необхідні для їх опису.
Головним чином це Фреймворк Закмана та сімейство стандартів які описують UML, System F
та необхідні генератори SDK, верифікатори типів (валідатори), моделі процесів, тощо.

\section{Безпека інтернету та інфраструктури}

Рівень безпеки визначає схему функціонування основного центрального засвідчувального орнагу,
акредитованих центрів сертифікації ключів, протоколи шифрування та підпису, директорію
підприємства, інтернет протоколи найменування ресурсів, шифровані протоколи комунікації,
диспетчерські системи трафіку повітряних суден, тощо. Усе визначено згідно ASN.1
специфікації і стандартів протоколів серії X\footnote{\url{https://www.itu.int/itu-t/recommendations/index.aspx?ser=X}}.
