\chapter{Обліково-реєстраційний рівень}

\section{Вступ}

Обліково-реєстровий рівень пропонує низькорівневе масштабоване розподілене журнальне сховище даних та метаданих, яке може бути побудоване на реляційних базах даних, базах даних з єдиним простором ключів з гарантіями консистентності (chain-hash) або їх комбінаціях. Класичні представники цього рівня в системах управління підприємствами: система управління людськими та матеріальними ресурсами, банківські системи PCI DSS, складські системи, системи управління поставками та виробництвом, системи сервісних послуг, системи управління проектами, тощо.

\subsection{Види реєстрів}

1) Реєстри орієнтовані на суб'єктів організаційних систем; \\
2) Реєстри орієнтовані на облік матеріальних ресурсів; \\
3) Реєстри орієнтовані на георафічні об'єкти; \\
4) Реєстри орієнтовані на події; \\
5) Реєстри орієнтовані на документи, накази, НПА; \\
5) Реєстри медичних систем (FHIR); \\
5) Реєстри предметно-орієнтованих словників для функціональних підсистем. \\

\subsection{Функціональні можливості}

\newpage
\section{Модулі підприємства}

ERP/1 є комплексом бібліотек (N2O.DEV) та підсистем додатків (ERP.UNO),
який використовує загальну шину і загальну розподілену базу даних для швидкіснх операційних вітрин.
\\
\\
\textbf{FIN} — Фінансовий модуль підприємства для бухгалтерії, зберігає бізнес процеси,
        які представляють собою рахунки учасників системи: персонал (для нарахування зарплат),
        рахунки та субрахунки підприємства (для здійснення економічної діяльності) і
        зовнішні рахунки в платіжних системах.
\\
\\
\textbf{ACC} — Система управління персоналом: зарплатні відомості,
        календар підприємства, відпустки, декретні відпустки, інші календарі.
\\
\\
\textbf{SCM} — Система управління ланцюжком поставок: головний БП системи —
           експедиційний процес доставки товарів ланцюжку одержувачів
           за допомогою транспортних компаній.
\\
\\
\textbf{PLM} — Система управління життєвим циклом проектів і продуктів.
           Також містить CashFlow та P\&L звіти.
\\
\\
\textbf{PM} — Система управління проектами підприємства з деталізацією
           часу і протоколів прийому-передачі (прийняті коміти в гитхабі).
\\
\\
\textbf{WMS} — Система управління складом, устаткуванням, деталями.
\\
\\
\textbf{TMS} — Система управління транспортом підприємства.
\\
\\
\textbf{HL7} — Медична система, яка реалізує міжнародний FHIR стандарт.

\noindent МЕДИЧНА КАРТА стаціонарного хворого --- №003/о\footnote{\url{https://zakon.rada.gov.ua/laws/show/z0662-12#n2}}\\
КАРТА ПАЦІЄНТА, який вибув із стаціонару --- №066/о\footnote{\url{https://zakon.rada.gov.ua/laws/show/z0668-12#n2}}\\
МЕДИЧНА КАРТА амбулаторного хворого --- №025/о\footnote{\url{https://zakon.rada.gov.ua/laws/show/z0669-12#n2}}\\
КОНТРОЛЬНА КАРТА диспансерного нагляду --- №030/о\footnote{\url{https://zakon.rada.gov.ua/laws/show/z0671-12#n2}}\\
МЕДИЧНА КАРТА стоматологічного хворого --- №043/о\footnote{\url{https://zakon.rada.gov.ua/laws/show/z0678-12#n2}}\\
КАРТА хворого фізіотерапевтичного відділення --- №044/о\footnote{\url{https://zakon.rada.gov.ua/laws/show/z0689-12#n2}}\\
Медична карта новонародженого --- №097/о\footnote{\url{https://zakon.rada.gov.ua/laws/show/z0233-16#n7}}\\

\\
\\
\newpage
\section{Архітектура облікових CART систем}

\subsection{Облік метаінформації}

\subsection{Облік ABAC правил}

\subsection{Облік інфраструктури і само-моніторинг}

\subsection{Облік словників і класифікаторів}

\subsection{Адміністративний облік}

\subsection{Облік таксономії предметної облісті}

\subsection{Облік процесів предметної області}

