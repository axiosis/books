\addtocontents{toc}{\protect\newpage}
\chapter{Технологічний рівень зв'язності людей та пристроїв}

\section{Вступ}

Ця глава визначає форамальну специфікацію на програмне забезпечення усіх рівнів моделі Закмана
для підприємств ISO-42010, містить широкий спектр прикладів, розказує про складові компоненти
та є вичерпним авторським стартовим посібником для курсу навчання розробки
технологічних програм для платформи Erlang і публікації в системі електронної взаємодії
державних електронних інформаційних ресурсів "Трембіта".

Трембіта — це система побудована на пакетах Ubuntu 18 LTS і пропонує головним чином
інфраструктуру X.509, а також розгортання національного форку шини X-ROAD,
який використовується як сереовище для гетерогенних сервісів, в якому всі 15 міністерств
публікують свої сервіси і їх клієнтські адаптери. Каталог сервісів доступний
публічно\footnote{\url{https://catalog.trembita.gov.ua}}. Сам інтерфейс управління X-ROAD
написаний з використаннями SOAP/WSDL специфікацій. В цьому цифровому просторі відбувається
взаємодія (передача конфіденційних даних в основному) між обліково-реєстраційними системами
міністерств на основі безпосередніх захищеник каналів зв'язку між сервісами та їх споживачами,
яка містить просту і фіксовану логіку.

СЕВ ОВВ, на відміну від системи Трембіта — це публічна шина даних для
урядової кореспонденції і нормативно-правових актів, вона побудована
згідно ISO/IEC 11756:2010 (MUMPS) на базі продукту InterSystems Caché.
В цьому цифровому просторі відбувається робота систем врядування
юридично-документального рівня

\section{Виробничий процес}

\section{Системи сховищ даних}

\subsection{Реляційні бази даних}

\subsection{Бази даних з єдиним простором ключів}

\subsection{Шини комунікації та брокери повідомлень}

\subsection{Розміщені в пам'яті гарячі дані}

\newpage
\section{Обчислювальні ресурси}

Концептуальна модель системи в рамках якої функціонує N2O визначаена як обчислювальне середовище,
яке складається з процесору подій (N2O), операційного (ETS) та персистентного сховища (KVS).
З точки зору обчислювального середовища, ресурси підприємства складаються з глобального
 сховища та обчислень, які розділяють глобальну адресацію та представляють собою
Erlang-процеси (N2O протоколи). Кожен процес PI, може містити певний набір протоколів,
будь-який з яких відповідає на певний набір повідомлень. Протоколи N2O визначені на
точці підключення повинні не перетинатиcя, в іншому випадку протокольні модулі можуть
перехоплювати та впливати на інші протокольні модулі, які повинні реагувати на той
самий тип повідомлень.

Усі асинхронні процеси PI запускаються під головним супервізором n2o та індексуються
URI ключем разом з типом реактивного каналу реального часу: ws або mqtt. N2O протоколи
підключені безпосередньо до веб-сокет точок підключення виконуються в контексті TCP
процесів, у даному випадку TCP-сервера бібліотеки RANCH, супервізор ranch\_sup.

\begin{lstlisting}
> :supervisor.which_children :n2o
[
  { {:ws, '/chat/ws/4'}, <0.985.0>, :worker, [:n2o_ws] },
  { {:ws, '/chat/ws/3'}, <0.984.0>, :worker, [:n2o_ws] },
  { {:ws, '/chat/ws/2'}, <0.983.0>, :worker, [:n2o_ws] },
  { {:ws, '/chat/ws/1'}, <0.982.0>, :worker, [:n2o_ws] },
  { {:mqtt, '/bpe/mqtt/4'}, <0.977.0>, :worker, [:n2o_mqtt] },
  { {:mqtt, '/bpe/mqtt/3'}, <0.976.0>, :worker, [:n2o_mqtt] },
  { {:mqtt, '/bpe/mqtt/2'}, <0.975.0>, :worker, [:n2o_mqtt] },
  { {:mqtt, '/bpe/mqtt/1'}, <0.974.0>, :worker, [:n2o_mqtt] },
  { {:caching, 'timer'}, <0.969.0>, :worker, [:n2o] }
]
\end{lstlisting}

\newpage
\subsection{Накопичувальні ресурси}

Розподілені хеш-кільця використовуються не тільки для розподілених обчислень,
але і для зберігання даних.
Деякі бази даних, наприклад RocksDB та Cassandra, використовують глобальний простір
ключів для даних (на відміну від таблично-орієнтованих баз). Саме для таких
баз і створено библиотеку KVS, де в якості синхронного транзакційного
інтерфейсу — API ланцюжків з гарантією консистентності. Нижче
наведено приклад структури ланцюжків екземпляру системи PLM:

   \begin{lstlisting}
> :kvs.all :writer
[
  {:writer, '/bpe/proc', 2},
  {:writer, '/erp/group', 1},
  {:writer, '/erp/partners', 7},
  {:writer, '/acc/synrc/Kyiv', 3},
  {:writer, '/chat/5HT', 1},
  {:writer, '/bpe/hist/1562187187807717000', 8},
  {:writer, '/bpe/hist/1562192587632329000', 1}
]
\end{lstlisting}

В нашій моделі синхронні протоколи використовуються для управління
накопичувальними ресурсами підприємства і транзакційного процесингу.

\section{Типові специфікації}

Протоколи визначаються типовими специфікаціями і генеруються для наступних мов:
Java, Swift, JavaScript, Google Protobuf V3, ASN.1. Також ми генеруємо валідатори даних по цих
типових анотаціях і вбудовуємо ці валідатори в тракт наших розподілених протоколів,
тому ми ніколи не дозволимо клієнтам зіпсувати сторадж. Для веб додатків у нас развинута
система валідації — як для JavaScript, так і на стороні сервера. Бізнес логіка повністью ізольована в нашій
системі управління бізнес процесами, де кожен бізнес процесс
є процесом віртуальної машини. Всі ланцюжки модифікуються атомарним чином,
підтримують flake адресацію, і не вимагають додаткової ізоляції
у своєму примітивному використанні.
Тому ви можете трактувати базу як розподілений кеш
і використовувати її з фронт додатків для примітивних випадків.

\newpage
\section{Середовище}

Для забезпечення повного замкненого середовища пропонують наступні заміни бібліотек kernel та stdlib: \\
\\ \indent
\setmainfont{Segoe UI Emoji}🚀\setmainfont{Geometria}\ VM --- віртуальна машина середовища виконання\footnote{\url{vm.n2o.dev}}

\setmainfont{Segoe UI Emoji}⭐\setmainfont{Geometria}\ BASE --- базова системна бібліотека як заміна stdlib\footnote{\url{base.n2o.dev}}

\setmainfont{Segoe UI Emoji}⭐\setmainfont{Geometria}\ RT --- бібліотека середовища виконання як заміна kernel\footnote{\url{rt.n2o.dev}}

\setmainfont{Segoe UI Emoji}📨\setmainfont{Geometria}\ SYN --- бібліотека PubSub для розподілених систем\footnote{\url{syn.n2o.dev}}

\setmainfont{Segoe UI Emoji}⚡\setmainfont{Geometria}\ MAD --- бібліотека управління пакетами та інстансами\footnote{\url{mad.n2o.dev}}

\newpage
\subsection{Бібліотеки}

Для запезпечення повноцінної промислової специфікації ERP/1, ми розширили
набір інструментальних засобів наступними бібліотеками: формальними представленнями
презентаційного рівня FORM та системою управління бізнес-процесів BPE. FORM представляє
собою декларативну бібліотеку побудови графічних інтерфейсів, а бібліотека BPE
підтримує XML файли стандарту BPMN 2.0 та реалізує безпосередню інтерналізацію
BPMN семантики у семантику віртуальної машини Erlang. \\
\\ \indent
\setmainfont{Segoe UI Emoji}⭕\setmainfont{Geometria}\ N2O --- сервер протоколів для стандартів MQTT/WS/QUIC\footnote{\url{ws.n2o.dev}}

\setmainfont{Segoe UI Emoji}🔥\setmainfont{Geometria}\ NITRO --- UI веб-фреймворк Nitrogen\footnote{\url{nitro.n2o.dev}}

\setmainfont{Segoe UI Emoji}💿\setmainfont{Geometria}\ KVS --- бібліотека доступу до KV сховищ RocksDB\footnote{\url{kvs.n2o.dev}}

\setmainfont{Segoe UI Emoji}📜\setmainfont{Geometria}\ FORM --- бібліотека декларативного конструювання іформ\footnote{\url{form.n2o.dev}}

\setmainfont{Segoe UI Emoji}💠\setmainfont{Geometria}\ BPE --- сисема управління процесами стандарту BPMN 2.0\footnote{\url{bpe.n2o.dev}}

\setmainfont{Segoe UI Emoji}☎\setmainfont{Geometria}\ RPC --- бібліотека генерації SDK для мов JS, protobuf, Swift\footnote{\url{rpc.n2o.dev}}

\subsection{Приклади}

Головні приклади фундації N2O.DEV присвячені наступним темам: MQTT та WebSocket
чати для демонстрації веб-фреймворку NITRO, який працює як модуль N2O, приклад
REST адаптер до бази даних KVS, та повністю чистий N2O додаток CHAT на основі
бібліотеки SYN без використання NITRO:\\
\\ \indent
\setmainfont{Segoe UI Emoji}💧\setmainfont{Geometria}\ SAMPLE --- ідіоматичний приклад Nitrogen поверх WS\footnote{\url{sample.n2o.dev}}

\setmainfont{Segoe UI Emoji}💧\setmainfont{Geometria}\ REVIEW --- ідіоматичний приклад Nitrogen поверх MQTT\footnote{\url{review.n2o.dev}}

\setmainfont{Segoe UI Emoji}☕\setmainfont{Geometria}\ REST --- бібліотека для побудови HTTP API\footnote{\url{rest.n2o.dev}}

\setmainfont{Segoe UI Emoji}💬\setmainfont{Geometria}\ CHAT --- приклад системи доставки повідомлень\footnote{\url{chat.n2o.dev}}

\newpage
\section{Протоколи, схеми та мови їх опису}

\subsection{Мова опису протоколів ASN.1}

\subsection{Мова опису протоколів SOAP/XSD/XML}

\subsection{JSON валідатори draft-07 і JTD}

\section{Формати передачі даних}

\subsection{Бінарні формати ETF/BERT}

\subsection{Бінарні формати DER/BER/PER}

\subsection{Колоночний текстовий формат CSV/CSM}

\subsection{Текстові формати JSON і XML}

\newpage
\section{Розробка Інтернет додатків}

\subsection{Erlang та сучасний веб}

Erlang реалізує недосяжну мрію кожного обчислювального середовища для
паралельної та узгодженої конкуретної обробки повідоблень. Так найбільш
відомі бібліотеки акторів (Akka, Orleans), які реалізують основні примітиви:
процесори та черги, копіюють модель акторів Erlang, зазвичай намагаються
також реалізують додатково механізми перезапуску та супервізії процесів
подібно до Erlang, проте тільки Erlang забезпечує soft real-time характеристики,
завдяки керуванню латенсі з точністю до таймінгу команд віртуальної машини.
А з виходом 24 версії в 2020 році, яка почала підтримувати JIT-компіляцію
завдяки asmjit, продуктивність та чуттєвість віртуальної машини зрозла
ще більше.

З формальної точки зору достатньо добре ізольоване середовище віртуальної
машини Erlang не тільки забезпечує характерстики реального часу для
SMP-планувальника легких зелених процесів, але і обмежує область видимості
heap пам'яті виключно для процесів-власників, що унеможливлює вплив відмови
певних процесів на глобальний стан віртуальної машини.

Erlang ідеально підходить для побудови високо-навантажених,
просто-масштабованих, подійно-орієнтованих, неблокуючих, надійних,
постійно-доступних, високо-ефективних, швидких, безпечних та надійних
систем обробки повідомлень та розподілених у просторі та часі систем.

\subsection{DSL vs Шаблони}

З технічної точки зору N2O успішно показує неперевершену досі якість
DSL програмування, яку ви не зможете знайти в сучасних веб-фреймворках
для мов Erlang та Elixir. За 7 років неперервної еволюції N2O ми переписали
кожен з 700 рядків по 30 разів, якшо порахувати через коміти Github.
Веб-фреймворк NITRO, сховище KVS, та BERT.JS кодування може забезпечити
відображення в веб-браузері повноекранних вертикальних форм з усіма
обчислюваними полями зі швидкістю 60 форм в секунду по веб-сокет каналу.
А надзвичайно компакта JavaScript бібліотека-компаньйон вміщується
в 4 MSS/MTU вікна — саме такий розмір мінімального веб-клієнта з BERT
кодуванням, який повністю управляється зі сторони сервера.

N2O сервер та веб-фреймворк NITRO реалізують концепцію не тільки
управління сесіями та каналами, але і усім стеком побудови додатків
включаючи UI частину, як це відбувається у таких веб-фреймворках як
Erlang Nitrogen, OCaml Ocsigen, Scala Lift, F\# WebSharper, а завдяки
таким розширенням як FORM та BPE ідеально підходять і для побудови
автоматизованих CRM систем.

Це не означає, що за допомогою N2O ви не можете створювати більш
класичні та архаїчні додатки у стилі DTL шаблонізаторів, або як це
відбувається у таких фреймворках як PHP, ASP, JSP, Rails, тощо.
Перші версії NITRO містили в прикладах використання Django Template
Library (DTL), проте задля чистоти стеку були прийнято не включати
в N2O додаткові шаблонізатори крім NITRO DSL.

\subsection{Історія}

N2O сервер, а також NITRO веб-фреймворк були спроектовані як інструментільні засоби
для створення промислових ERP модулів підприємства у складі відкритої платформи ERP/1.
Напочатку, N2O був відгалужений, як оптимізована версія веб-фреймворку Nitrogen,
створеного Расті Клопгаузом. Хотілося оптимізувати та вдосконалити мінімізований
WebSocket-тракт, який не містить синхронного протоколу HTTP взагалі та дозволяє
створювати повноцінні асинхронні веб-додатки реального часу. На ньому була створена
система управління депозитами в національному банку ПриватБанк. Пізніше N2O був розділений
на бібліотеку-фреймворк процесів та протоколів (власне N2O) та бібліотеку-веб-фреймворк NITRO.
Бібліотеки N2O та NITRO також отримали можливість роботи не тільки через WebSocket але і
через MQTT та через чисті TCP або UDP. Така оновлена версія 5.10 була впроваджена як ядро
системи повідомлень для додатку NYNJA з відкритим open-source протоколом і саме ій
присвячений друга версія підручника.

\subsection{Інтерфейс NITRO}

\subsection{Сховище KVS}

\subsection{Логіка BPMN}

\subsection{Додатки MQTT та WebSocket}
