\section{Практика}

\subsection{Спочатку йога розуму, потім вже йога тіла}

Рано или поздно, все кто занимается йогой ума при должном успехе, так или иначе переходят к освоению йоги тела, которое является продолжением ума. Секрет успешной практики в том, что как и йога ума, йога тела требует еще более острой осознанности. Под йогой тела мы будем понимать здесь такого рода энергии, которые вы можете пережить только в режиме эксремального спорта, там где ревард очень высокий. В спорте   это X-спорт, в сексе   это BSDM, подходит все дисциплины, где есть стоп-слово, за границей которого сразу наступает терминация существа.
\\
\\
Почему йога тела должна идти обязательно после укрпления в практике йоги ума? На это есть несколько причин. Считается, что основа устойчивого и взрослого мышления   это правильное мировозрение, которое должно формироваться существом в период изучения философских дисциплин, нерешенных вопросов трансгуманизма и других базовых принципов. Придерживаясь внесектарного стиля, самые базовые принципы мировозрения топового программисты были зацементированы в первом выпуске.
\\
\\
Положив неправильный майндсет в основание Изучения, Размышления и Медитации вы создате брешь, через которую в критический момент вашей жизни при встрече с Буддой ваше видения мира разрушиться как детский замок из песка после прилива. Для тренировки ума алмазной крепкости и остроты и предназначены практики формирования правильно взгляда на объект ислледования   свой собственный ум, за предалами котого не существуют никакие феномены.
\\
\\
Как и йога ума, йога тела предусматривает два режима исследования: пандита стайл (изучение теории) и йога стайл (практика). Усвоившись и укрепившись в своем сознании, существо движущиеся главным принципом будда-таковости, стремлением к всезнанию и полной реализации, зерном которого является познание феноменов, начинает выходить за рамки ментальных феноменов и начинает осознавать себя и свое тело как часть мышления, и естественным образом начинает экспериментировать с телом, расширяя свой фронтир восприятия.
\\
\\
Главный критерий который показывает можно ли вам переходить к спорту   это полный контроль над дофаминовой и эпинифриновой системой. Некоторые слишком впечатлительные спортсмены используют THC для суппрессии дофамина и более мягкой йога-сессии. Обычно рекомендуется входить в спорт в 40 лет, потому что в 25 и несколько лет после нужно посвятить математике и философии, ведь лучшего времени уже не будет и вернуть его будет непросто! А в 40 уже дофаминовый фон сам по себе исчезнет и останется только чистый ум и террейн. Это вторая безжалостная причина по которой спорт лучше отложить до adult возраста. Есть и другая сторона медали: раны после 40 заживают хуже, поэтому и ставки и острота и ревард в таком случае выше. Идеально   это имея острый ум не совершать вообще серьезных ошибок на пути спорта. Неидеальные случаи решаются имплантацией титановых пластин.
\\
\\
Коcвенный критерий   это когда вы достигли уровня непосредственного переживания отождествления языка пространства (выпуск Х), карты местности (выпуск 4) и своего мышления (выпуск 2), в таком случае выход из локальной самсары в виде темницы ума будет знаменовать выход в реальный мир на планету Земля. Алегория которая мне видится здесь такая: мать отправляют своего сына в университет передав ему все необходимые знания, которые помогут ему жить дальше автономно.

\subsection{Программування та спорт}

Гуру в спорте найти так же тяжело как и гуру в программировании. В светской жизни гуру спорта работают олимпийскими чемпионами или чемпионами в дисциплинах которые к этому приравниваются. Само их существование уже является учением. Разбирая до мельчайших деталей покадрово трейсы топ-спорсменов на youtube вы получаете алмазные знания виртуоза-ньюскулера методично оттачивая технику имея образец для верификации. Гуру спорта поменьшего калибра будут работать инструкторами на ближайшем спортивном курорте, рекламировать газировку и эквипмент или предлагать вам туры по национальным заповедникам.
\\
\\
Не все это говорят прямо, но в спорте важными являются картинки которые проходят через вашу сеточку и все органы чувств, поэтому ценятся картинки естественного ландшафта Земли, чтобы освободить свое мышление уже за границами тела, охватив своим мышлением всю планету и ее феномены, главный из которых   гравитация, таким образом став воистину космическим ребёнком планетарного масштаба.
\\
\\
Однако в разговорах с локальными спортивными гуру вы получите максимум репрессивные монотонные лекции о вреде курения на Джомолунгме, ужасные команды инструкторов-обывателей, толпы туристов на своём пути. Взрослый человек овладевший йогой ума сам должен стать себе гуру и планировать каждую вылазку на встречу с гравитацией как проект с многими параметрами-переменными, от проработки которого зависит ваша жизнь.
\\
\\
Все эти гуру будуть говорить вам, что только они понимают суть вещей, познали природу в не-мышлении, а у вас большой груз йоги-ума, который мешает вам достигать результатов, заморочки, излишняя концептуализация, чрезмерная начитанность, и другие смертные грехи. Тут действует такое же правило как и в детском саде, школе или университете   "умных не любят", поэтому имейте это ввиду собравшись поговорить по душам за костром с очередным спортивным гуру.
\\
\\
Разумный и рациональный человек всегда выберет более редкую и филигранную йогу ума вместо йоги тела, которой владеет гораздое большее количество существ. Ведь получить алмаз ума гораздо сложнее чем алмаз тела, поэтому партнеры по спорту и даже локальные гуру, будут готовы в прямом смысле подсознательно вас убить на склоне, тут тоже нужен глаз да глаз.
\\
\\
В тибетском буддизме аналогом спорта является тайное посвящение каналов, ветров и сфер   структуры вашей алмазной сети. Все эти йоги выполняются в паре с опорой на партнера и добраться до этого уровня будет затруднительно обывателю. Тем более, что таких гуру еще меньше чем спортивных.
\\

\subsection{Баланс}

Главный наркотик для йоги-ума и йоги-тела   это сахар. Все спорсмены как минимум висят на кока-коле, ред-булле, гелях, стимуляторах. Баланс этих веществ и правильное питание   ключ к быстрому восстановлению после спортивных сессий, которые так или иначе нужно будет залечивать перед полноценными сессиями йоги ума. Игнорировать вспомонательные вещества на пути спорта глубо, но и злоупотреблять не стоит. Главный критерий, непрерывность, вы должны планировать ваше путешествие таким образом, чтобы ваше сознание не пошатнулось от внезапного изменения гормонального фона, дефицита того или иного топливного элемента.


\normalsize
