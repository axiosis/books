\addtocontents{toc}{\protect\newpage}
\section{Таємні настанови}

\subsection{Аспекти курсу}

Після того, як у перших двох розділах ми розповіли про
топову мотивацію та топове мислення, перейдемо до топового
курсу топового програміста. Структура цього курсу програмування
буде дуже відрізнятися від інших підходів до навчання, але цей
курс є курсом, який я намагався "побачити" крізь ретроспективу
свого професійного досвіду, що стосується безпосередньо
програмування, того, як я бачу топовість цього процесу.

Перед початком зі структури курсу я хочу показати основні
індикатори (аспекти курсу), на які варто орієнтуватися при
виборі топових напрямків у програмуванні. Тому що саме ці
індикатори визначають ті предмети та їх послідовність в
якій ви повинні поглащати інформацію, щоб наблизитися до
максимальної топовості.

\subsection{Гранична точність}

Гранична точність означає абсолютні обчислення. У цю вигадану
категорію я включаю всі галузі програмування та математичного
моделювання, які вимагають абсолютної формальної точності:
системи доказу теорем, спеціалізовані формальні верифікатори
моделей, системи символьної алгебри, системи обчислювальної
гомологічної алгебри, тобто. ті системи моделювання, які не
тільки розкладатимуть саму систему до атомів, але ще й найближчу
метамодель на рівень вище, яку теж потрібно формалізувати, щоб
верифікувати самі моделі. Сюди входять такі галузі як
процесоробудування (модель чекерів друкованих плат,
процесорів, спеціалізовані мови програмування типу VHDL),
мовобудування (системи типів, формальні мови програмування,
мови загального призначення). Сюди також входять: будь-які
формальні математичні теорії, формальні логіки, мова кванторів Пі та Сігма.

\subsection{Гранична оптимальність}

Гранична оптимальність означає мінімальну кількість зусиль,
доданих до досягнення мети. Сюди входять дискретні завдання
міні-максу, лінійне програмування, симплекс-методи, поліедральне
багатовимірне симплектичне програмування, методи оптимізації.
У фізиці це основний принцип варіаційного числення, мінімальні
геодезичні лінії. Ваша система не тільки повинна бути
максимально точною, в апогеї абсолютно точною, але і
повинна бути закодована оптимальним чином, не містити
частин, що повторюються, займати мінімальний футпринт
по обмеженій кількості пам'яті і обчислювальних потужностей.
Краще, щоб була теорема яка доводить цю мінімальність,
як наприклад, чорч-кодування індуктивних типів як природне
кодування будь-яких структур у лямбда численні, оптимальні
лямбда евалуатори і т.д. Гранична оптимальність означає
також вищий пілотаж у прототипуванні та створенні MVP ескізів.

\subsection{Гранична складність}

Гранична складність означає дослівно максимальну складність
системи, з якою потрібно працювати. Якщо раптом виявиться,
що вам складності декартово-замкнутих категорій мало, завжди
можна перейти до симетричних модальних категорій, в яких живуть
такі мови програмування: квантові мови програмування,
конкруретні паралельні системи типу Erlang, системи
лінійних типів, мови для обробки тензорів. Також гранична
складність має на увазі наявність вже в системі граничної
точності та граничної оптимальності, інакше без першого у
вас буде просто непрацююче гівно, у відсутності другого у
вас буде мільйони рядків дублюючого коду, які не впливають
на реальну складність проекту. Якщо ви хочете побачити реальну
складність подивіться кубічні докази К-теорії.

Це стосується більш-менш чистих тем з логіки, дискретної
математики та програмування. У міждисциплінарному підході,
якщо ви хочете стати не топовим програмістом, а, наприклад,
топовим біофізиком, то список тем буде іншим. Гранична точність
там замінюється на математичну статистику та стохастичну фізику,
гранична оптимальність пов'язана з топологією просторів,
а гранична складність виражається в об'єднанні полів і,
як приклад, стандартної моделі. Ну а в біофізиці складність
збільшується топологічними різноманіттями нейромереж як
ієрархічних процесів, що працюють з тензорними потоками.
Гомотопічна теорія типів як мова програмування інфітіні
топосу є у фізиці аналогом теорії струн. Біофізика ж є
більш дисипативними структурами, які втрачають межі
точності, таким чином завдання створення AI не повною
мірою відповідає абсолютній точності. Я би сказав що AI
це більше системна інженерія та прикладна область, в той
час, як абсолютна точність це більше сфера математичного
програмування. І навіть машинне навчання при детальному
розгляді зводиться до статистичної точності, прогнозованої
оптимальності та системної складності.

Я як прихильник абсолютної складності вважаю, що будь-яка
сучасна PhD повинна містити міждисциплінарний підхід. Як
необхідний мінімум у міжнародній науковій практиці пропоную
розглянути міждисциплінарний підхід, який базується всього
на двох особливих дисциплінах: УДК 51 (математика) та
УДК 004 (програмування). В якості особливих вони обрані
тому, що будь-яка інша дисципліна базується на чистій
математиці, а для проведення будь-якого моделювання
(перевірки перевірки теорії), потрібен фундаментальний курс програмування.

Цікаво, що третій із цих індикаторів можна інвертувати
і при цьому оптимізаційний вектор (максимальна точність,
максимальна оптимальність та мінімальна складність) теж
виявиться цікавим об'єктом розгляду. В суті це вимоги,
 що висуваються до бібліотек програмного забезпечення:
вони повинні бути максимально формальними з доказами
властивостей їх моделей, повинні бути оптимально
змодельовані або володіти оптимальним екстрактом
в інші моделі, але при цьому також повинні бути
максимально простими або мінімально складними у
загальній картині компонентів. Філософія N2O точно
відповідає такому альтернативному оптимізаційному
вектору, але вирішуючи завдання технологічних стеків
не можна стати топовим програмістом, потрібно
збільшувати складність, наприклад у бік складних
соціо-інформаційних систем, як ERP системи управління
підприємством, де десятки тисяч таблиць кожна по
сотні полів звична складність для бізнес-аналітиків,
що працюють з такими продуктами як SAP S/3.

Забавно, що в езотеричному буддизмі Тибету,
ці три критерії застосовуються в настановах
з візуалізації у вищих тантрах: ви повинні
максимально детально (до волосся і фактури
тканин одягу --- принцип максимальної точності)
представити дуже складний об'єкт (18-руке божество
в союзі з дружиною, свитою та аттрибутами --- принцип
максимальної складності) максимально швидко (бажано
миттєво --- принцип оптимальності). Учні, які можуть
це робити не "на словах", а "на ділі" - вбудовуючи
картинку силою думки і уяви навіть не на січень очі,
а прямо на кору головного мозку, випалюючи зображення
у своєму мисленні на простирадлі уявного всесвіту ---
вважаються топовими медитаторами . Для наочного прикладу
хочу показати приблизну якість візуалізації, що
утримується в просторі свого мислення, якого може
досягти середній за здібностями європеоїд-медитатор
за 3 роки наполегливої практики:

Напевно, тому досі не існує жодного просвітленого
європейця! Ця картинка також наочна демонстрації
того, що ви можете досягти за 3 роки навчання (думаю,
що у відділу PhD хлопців із NASA, підготовка до рендеру
цієї картинки зайняла три роки).

\normalsize
