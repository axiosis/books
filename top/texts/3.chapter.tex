\section{Простір професійного розвитку}

\subsection{Структура курсу}

Не то, чтобы это была какая-то новость, уверен многие придерживаются такой карты топового программиста, но я возьму на себе смелость открыть это тайное знание. Начну описание курса с известной мемной картинки:

\subsection{Дракон}

Единорогами называют тех программистов, которые одинаково хорошо владеют CSS скажем, а также могут полностью построить любой сложности тонкий или толстый клиент не ограничиваясь HTML5, но и переходя в SVG или WPF, или DirectX или OpenGL.

Фулстек программистами называют специалистов по построению информационных систем на границе с единорогами (которые обычно не занимаются процессиногом, инфраструктурой, сетями и защитой).

\subsection{Лямбдагарбха}

Следующий уровень — это платформообразующий уровень, который включет язык программирования, рантайм и аппаратуру. Обычно взрослые академические языки создаются сразу с рантаймом, поэтому назовем эту секцию уровень университетского профессора, а секцию рантайма (ОС) и аппаратура назовем предпринимательской, так как ОС обычно продают вместе с железом и все кто это пытался продвигать на рынок можно приравнять к бодхисаттвам. Последние известные лямбдагарбхи — это древние автора первых Лисп машин и XEROX PARC.

\subsection{Гротендік}

На абсолютном уровне программисты (в том числе и топовые) являются математиками, поэтому тут можно отметить ядро которое было открыто Квилленом — модельные категории, в которых работали не только медалисты Филдса — Воеводский и сам Квиллен, но которые являются также основным инструментом современных теоретико-типовых математиков как Шульман. Предмет изучающий модельные категории Квиллен назвал гомотопической алгеброй, при помощи которой была построена не только модель алгебраической топологии самим Квилленом, но и А1-теория гомотопий Воеводского. Все это крышуется Гротендиком, как мультидисциплинарным программистом абсолютного уровня (топ-математиком).

\subsection{Будда}

Без лишней скромности, любой программист который смог не только представить, но и успеть порабать за жизнь на всех уровнях, может считать себя Буддой программирования, или как мы скромно называем таких пацанов — хуй с горы.

\normalsize
