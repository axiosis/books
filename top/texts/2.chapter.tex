\section{Мислення}

\subsection{Що таке мислення}

Перед тем как начинать процесс обучения неплохо было бы пару слов сказать об основном инструменте в процессе изучения – человеческом мышлении. Минуя физические стороны мышления сразу хочется поговорить о его когнитивных свойствах.

\subsection{Характеристики чистого мислення}

Первое и главное свойство мышления – это существенность -- определяющая характеристика существа. Интегральная высшая форма, которая управляет всеми подсистемами и воспринимается существом, майндстримом или аватаром. В одном теле может жить несколько майндстримов, и некоторые из них вполне могут быть программистами! Если вы мыслите – вы существо.

Вторая когнитивная характеристика мышления, которую можно почувствовать в медитациях – это абсолютная сферическая открытость во всех направлениях и его безграничность. Такая характеристика мышления навевает мысли об изоморфизме мышления и пространства. С физической точки зрения, мышление – это сложная система квантовых полей, которые наслаиваются на квантовый, молекулярный уровень, нервную систему, поэтому долго доказывать не нужно, что мышление как квантово-механическая система, распространяется на всё пространство.

Условно существует два раздела высшей медитации, первый из которых называется разделом мышления, а второй разделом пространства. Первый раздел посвящен техникам работы с феноменами, аналитической медитации, работе с мышлением с точки зрения майндстрима, глазным упражнениям, развитию ощущения перспективы, работе с воображением, визуализациям. Второй раздел посвящен техникам работы с мышлением с точки зрения пространства, где мышление ассоциируется с пространством, в котором оно пребывает, неаналитической медитации устремлении в бесконечность, медитации отдохновения.

Третья когнитивная характеристика мышления, которую можно воспринять на опыте – это его необсуловленность. Чем выше уровень развития мышления, тем выше его воля к свободе и необсуловленность, к перепроверке, критическому мышлению и переоцениванию. В своей полной свободе мышление свободно выбирать направленность и интенсивность потока, без резких перепадов и гормональных фонов, двигаясь по оптимальной траектории взрослении плода мышления на пути к всеведению.

Четвертая характеристика мышления – это непрерывность. Любые попытки остановить мышление приводят в место самоосознавания как несущей частоте ощущения присутствия себя в этом мире, в медитации. Даже в процессе сна, мышление не спит, а переходит в другое агрегатное состояние, более разреженное, порой бесформенное, нечеткое, мерцающее. Полный контроль над непрерывностью мышления, от которой нельзя отказаться и которую нельзя прекратить – задача топового программиста на пути к освобождению ресурсов для изучения программирования. Чем больше точность дискретизации у этого контроля – тем лучше. Контроль над непрерывностью мышления называется точностью мышления.

Пятая характеристика мышления – взаимозависимость. Вы как мышление – это продукт абсорбции других фрагментов мышлений или просто феноменов, поэтому обусловлены этим наследием. Вырваться за пределы этой традиции и раскопать инсайты на пути эволюции своего мышления – истинная драгоценность как награда за труд обучения. Когда вы становитесь мастером, обусловленность пропадает, вы реструктурируете себя заново исходя уже из личного опыта, построенного на череде инсайтов, по которым вы прыгаете на пути к мастерству. И даже их вы потом сможете удалить и забыть из своего мышления оставив только память о том, как нужно сразу делать правильно, возможно и не вспомните даже, когда вас спросят, как это вы так быстро помудрели, а зачем.

Эти пять характеристик послужат вам подсказками в каком ключе нужно мыслить о своем мышлении (первая производная) как инструменте познания, возможно для существ с высокими способностями это сразу прояснит некоторые моменты. Любая неспособность наблюдать эти характеристики в практических медитация или размышлениях о своем мышлении, говорит о том, что их нужно развивать, либо заняться йогой, пойти к психологу, развеяться с друзьями, пойти в бар, сесть на таблетки, стакан, всё по желанию – главное, чтобы сработало! Чек лист прошли переходим к рекомендациям и индикаторам.

\subsection{Коштовне намисто мислення}

В тибетской традиции существует шесть типов мышления или программ, которые считаются, что позитивно могут повлиять в целом на процесс изучения, размышления и медитации.

Щедрость в контексте мышления означает не жадничать в процессе изучения, не хвататься за все сразу, иметь методологию, с уважением относится к любой выбранной теме, раз она уже всплыла в медитации как гештальт, который все равно придется закрыть. Способность к репликации, преподаванию, возвратной контрибуции на пути поглощения информации – это щедрость мышления.

Дисциплина означает, что мышление должно придерживаться какого-то спортивного, желательно олимпийского режима, слишком хаотичные режимы мышления не будут способствовать обучению, поэтому приступать к эволюции своего мышления нужно, когда гормональный фон может оставаться ровным значительное время, это необходимо для глубоких медитаций, без которых невозможен прогресс.

Терпение – это способность переносить трудности в процессе обучения. Есть материал, который может не закрываться годами, но к нему, все равно придется возвращаться, ведь назад дороги нет, выбран путь топового программиста. На пути может быть слишком много инсайтов и слишком много воодушевления, которое может создавать гормональный фон, который не всегда можно контролировать, пересиживать на бенче такие периоды – это терпение.

Усердие – означает с неподдельным интересом изучать предметы, поэтому правильно их расположить очень важно. Возможно, именно для вас существует своя последовательность предметов, каждый из которых в отдельным момент времени вы будете изучать с максимальным усердием. С этим придется работаться, каждому индивидуально.

Фокусировка – фокусировка, или концентрация, или медитация, или шаматха – это основной режим работы программиста. Вот вы сели за комп, поставили чашку с кофе, протерли дисплей, всосались в пиксели, запустили шелл – вы сфокусированы на работе, это медитация.

Мудрость – это система накопленных инсайтов, которая формирует новые структуры мышления, новую его топологию. Эта система может переписывать старые неэффективные и невалидные структуры, над которыми мы смеемся повзрослев. Мышление мудрости – это мышление основанное исключительно на таких проверенных рафинированных структурах, которые положены в фундамент нашего существа.

\subsection{Отрути мислення}

Самые три неблагоприятные формы мышления по моей личной классификации.

Инертность мышления – это колесо медитации. Будучи раз запущена некоторая привычка, уходит в автоматический режим на подсознание – это инертность. Если бы не было инертности мышления мы бы не смогли учиться. Хотя это полезное свойство мышления, иногда бывает плохо, когда плохо – нужно отлавливать. Понятно, что дебажить свое мышление, которое спрятано в подсознании не учат в школах, придется работаться самому.

Лень. Слишком интенсивное мышление может перерасти в затяжную рекреационную прокрастинацию, которая сменится ленью. Наблюдение за видимым прогрессом необходимо, каким бы не был охуенный отдых нужно возвращаться за программирование, обновлять мотивацию, если нужно каждый день.

Безразличие. Корень всех ядов, жадности и прочего. Если вам все вдруг стало безразлично – это очень плохо, но не смертельно. Иногда может перерасти в экзистенциальный кризис, но мы же с вами уже договорились, что тело, йогу и таблетки и свое самочувствие вы берет на себя, с меня только рекомендации по процессу обучения. Запущенное безразличие -- это тупость.

Ничего выходящего за перечисление доблестей хорошего успешного студента здесь нет. Как и в первом случае постоянно применяем технику размышления, изучения и медитации к этим видам мышления, так же как к основным характеристикам мышления. Постоянно проходим валидацию своего мышления в соответствии с выбранными индикаторами.



\normalsize
