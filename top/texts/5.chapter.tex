\section{Теоретичні основи}

\subsection{Метафілософія}

Дуже коротко про сучасні філософії. Як визначити філософію предикативно,
то це науа, що вивчає наступний перелік питань:
1) як жити добре в достатку та гедонізмі максимально довго всім і не померти від воєн та метеоритів;
2) реальний світ та інші питання гнесеології;
3) свобода волі;
4) етика;
5) математика;
6) музика;
6) література,
--- це всі питання, або мовні набори та форми, що цікавлять сучасних філософів.

У цій нотатці ми спробуємо побудувати формальну систему
і на її прикладі показати інтерпретацію трьох ліній
передачі сучасної філософії: європейської чи континентальної
філософії --- школи, яка задала початок глобальній
інтерпретації світу та реконструкції мов, у тому числі
й математики; східної філософії як приклад особливої школи,
на прикладі якої ми будуватимемо модель; та аналітичної чи
англосаксонської філософії, що формалізується сучасною математикою.

\subsection{Європейска філософія}

На наш погляд, головне питання європейської філософії --- це Good Life.
Як жити, як жити добре самому, у соціумі, які цілі можуть стояти перед
індивідом та видом, баланс етики та етика балансу. Європейська
філософія народила геометорію, психоаналіз, навчила людей не
боятися свободи, трансформувати агресію, бути більш зрілою
істотою, і під вінець свого розвитку поставила питання про
мову та мовну гру, як основний інструмент рефлексуючої свідомості.

Мова перестала мати ґрунт, вона стала просто візерунками, семантика
яких втрачена, філософія стала формою літературного мистецтва.

Представники континентальної філософії: Арістотель, Платон, Кант,
Декарт, Ніцше, Фрейд, Юнг, Юм, Хайдегер, Адорно, Хабермас, Делез.

\subsection{Тибетська філософія}

У східній філософії центральним питанням є визволення себе
і інших, в першу чергу від різних форм страждання. Ця філософія
має чітку систему, яка нерозривно пов'язана з тілесними та
розумовими практиками, і вижила протягом тисячоліть у
законсерованому гірському плато. Тут також порушуються
питання етики та свободи волі, але основний наголос робиться
на інтелектуальні та неконцептуальні вправи, що ведуть до
безпосереднього переживання простору.

Деякі формулювання східної філософії, такі як недвійність
всіх феноменів піддаються формалізації в гомотопічній
теорії типів (використовуючи методи аналітичної філософії),
що спонукало до подальших досліджень у галузі формалізації
езотеричних теорій.

Представники східної (тибетської) філософії: Атіша, Нагарджуна,
Бхававівека, Камалашила, Шантаpакшита, Арьядева, Бyддхапаліта,
Чандракірті, Цонкапа, Міпам, Лонгченпа.

\subsection{Аналітична філософія}

Аналітична філософія народжена в математиці, рання аналітична
філософія починається напевно з Лейбніца, Ньютона та Ейлера.
Пізня аналітична філософія починається З Фреге і за списком:
Рассел, Уайтхед, Дедекінд, Пеано, Гільберт, Фон-Нейман, Каррі,
Акерманн, Карнап, Сколем, Пост, Гедель, Черч, Берньє, Тюрінг,
Кліні, Россер, Мак-Лейн Ловір, Гротендік, Скотт, Джояль,
Терньє, Мартін-Леф, Мілнер, Жирар, Плоткін, Рейнольдс,
Бакус, Барр, Барендрехт, Лер'є, Силі, Кокан, Х'юет, Ламбек,
Воєводський, Еводі, Шульман, Шрайбер.

Якщо описати двома словами головне питання аналітичної
філософії --- це мова простору. Побудова мови, яка дасть
формальний фундамент не лише математики та роздумів, а й самої філософії.

\subsubsection{Мова простору}

Формальні підстави мови роздумів, математики (усієї) та фізики (всесвіту).

\subsubsection{Мовні фреймворки}

Мовні фреймворки для менш формальних (з парадоксами) та нечітких (стохастичних) систем.

\subsubsection{Конкретні соціальні дифузійні моделі}

Прикладна філософія Використання мовних фреймворків для опису конкретних феноменів.

\normalsize
