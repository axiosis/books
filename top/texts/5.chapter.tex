\section{Теоретичні основи}

\subsection{Метафілософія}

Дуже коротко про сучасні філософії. Якшо визначити філософію предикативно, то це
науа, що вивчає наступний перелік питань: 1) як жити добре в достатку та гедонізмі
максимально довго всім і не померти від воєн та метеоритів; 2) реальний всесвіт
та інші питання гнесеології; 3) свобода волі; 4) етика; 5) математика; 6) музика; 6) література,
--- це все питання, або мовні набори та форми, що цікавлять сучасних філософів.

В этой заметке мы попытаемся построить формальную систему и на ее примере показать интерпретацию трех линий передачи современной философии: европейской или континентальной философии -- школы, которая задала начало глобальной интерпретации мира и реконстуирования языков, в том числе и математики; восточной философии как пример особенной школы, на примере которой мы будем строить модель; и аналитической или англосаксонской философии, которая формализируется современной математикой.

\subsection{Європейска філософія}

На наш взгляд главный вопрос европейской философии -- это Good Life. Как жить, как жить хорошо самому, в социуме, какие цели могут стоять перед индивидом и видом, баланс этики и этика баланса. Европейская философия родила геометорию, психоанализ, научила людей не бояться свободы, трансформировать агрессию, быть более зрелым существом, и под венец своего развития поставила вопрос о языке и языковой игре, как основополагающем интрументе рефлексирущего сознания.

Язык перестал иметь почву, он стал просто узорами, семантика которых утрачена, философия стала формой литературного искусства.

Представители континентальной философии: Аристотель, Платон, Кант, Декарт, Ницше, Фройд, Юнг, Юм, Хайдегер, Адорно, Хабермас, Делёз.

\subsection{Тибетська філософія}

В восточной философии центральным вопросом является освобождение, себя и других, в первую очередь от различных форм страдания. Эта философия обладает четкой системой, которая неразрывно связана с телесными и умственными практиками, и выжила на протяжении тысячилетий в законсерированом горном плато. Здесь также поднимаются вопросы этики и свободы воли, но основной упор делается на интеллектуальные и неконцептуальные упражнения ведущие к непосредственному переживанию пространства.

Некоторые формулировки восточной философии, такие как недвойственность всех феноменов поддаются формализации в гомотопической теории типов (используя методы аналитической философии), что и побудило к дальнейшим исследованиям в области формализации эзотерических теорий.

Представители восточной (тибетской) философии: Атиша, Нагарджуна, Бхававивека, Камалашила, Шантаpакшита, Аpьядэва, Бyддхапалита, Чандpакиpти, Цонкапа, Мипам, Лонгченпа.

\subsection{Аналітична філософія}

Аналитическая философия рождена в математике, ранняя аналитическая философия начинается наверно с Лейбница, Ньютона и Эйлера. Поздняя аналитическая философия начинается С Фреге и по списку: Рассел, Уайтхед, Дедекинд, Пеано, Гильберт, Фон-Нейман, Карри, Акерманн, Карнап, Сколем, Пост, Гёдель, Черч, Бернье, Тюринг, Клини, Россер, Мак-Лейн, Ловир, Гротендик, Скотт, Джояль, Тернье, Мартин-Лёф, Милнер, Жирар, Плоткин, Рейнольдс, Бакус, Барр, Барендрехт, Лерье, Сили, Кокан, Хьюет, Ламбек, Воеводский, Эводи, Шульман, Шрайбер.

Если описать в двух словах главный вопрос аналитеческой философии --- это язык пространства. Построение языка, который даст формальный фундамент не только математике и размышениям, но и самой философии.

\subsubsection{Мова простору}

Формальные основания языка размышлений, математики (всей) и физики (вселенной).

\subsubsection{Мовні фреймворки}

Языковые фреймворки для менее формальных (с парадоксами) и нечетких (стохастических) систем.

\subsubsection{Конкретні соціальні дифузійні моделі}

Прикладная философия. Использование языковых фреймворков для описания конкретных феноменов.

\normalsize
