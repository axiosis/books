\section{Поклоніння простору}

\subsection{Топовий програміст}

Якщо мої передплатники і просять про якусь масштабну контрибуцію,
то це монографію на тему «як стати топовим програмістом».
Хоча таке формулювання інфантильне, воно досить добре
відображає сутність запитуваного: детальний розгляд професії
програміста, стратегію вивчення предмета виходячи з особистого
досвіду, розбавлений автентичною філософією.

\subsection{Висловлювання нескіченної поваги}

Перед тим як розпочати розповідь про професію програміста
насамперед хочеться висловити пошану предмету вивчення та
практики програмування, а саме формальним математичним
обчислювальним побудовам, у яких можливе програмування
в принципі. В останні роки стало зрозуміло, що простір
цих побудов настільки глибокий, що може поглинути не
тільки всі дискретні програми всіх формальних граматик,
а й континуальну математику, у якій робота з простором
йде на іншому, більш фундаментальному рівні. Тому без зайвого
перебільшення можна сказати, що саме простір народжує
мовну групу мов, які є первісною матрицю всіх без винятку
мов програмування.

Принцип глибокої поваги до предмету, який
майстер повинен реалізувати є одним із секретних ключів
східної філософії. Позаяк програмування народжується з
простору феноменологічних побудов, що ведуть до абстрактної
класифікації просторів та логік з ними пов'язаних, то утримання
у фокусі мети вивчення простору та програмування як
практичного людського процесу з цим пов'язаним є
головним завданням на шляху вивчення. Тому, без недооцінки та
зайвого перебільшення можна сказати, що простягання або поклоніння,
як прояв поваги до самого простору, як об'єкту вивчення,
виглядає для мене логічним. Я простягаюся перед простором.

\subsection{Перевірка мотивації}

Важливою характеристикою, яка, хочете вірте, а хочете ні,
впливає на процес вивчення мистецтва програмування, є чистота мотивації.
Якщо розглянути граничный популярний споживацький приклад, то він буде виглядати так:
ваша мотивація полягає у збільшенні своїх навичок програміста для
досягнення матеріальних благ і підвищення конкурентоспроможності
на ринку праці. Повна нісенітниця, така мотивація вприває на критерії
вибору об'єктів вивчення і це може завести вас у ситуацію, коли вам
50 років і ви пишете на Core Java для якось швейцарского банку.
Очевидно, що люди, які просили у мене цей текст, не очікують чогось подібного.

Свої ілюзії про легкість цього шляху можна відразу відкинути.
Цей шлях по-самурайськи скдадний і на нім сходили з розуму не тільки
випускники прикладної математики, немало людей перегоріло на
підприємствах від перенавантаження та неконтрольованості інформації.
Тому 10 років ув'язнення з постійним виділеним
каналом в інтернет на повному зовнішньому забезпеченні ---
ідеальний ресурс, який я рекомендував би виділити для успішної
підготовки на майстра програмування.

Чому такий великий термін буде пояснено в наступних частинах.
10 років цілком адекватний інтервал навчання для лікаря, то чому для програміста повинно бути менше.
Кількість мов якими кваліфікований програміст володіє на практиці може сягати тисяч,
за кожною з них стоїть теорія, своя логіка і своя математика нею породжена.
Це не просто латина, есперанто та пару мов романо-германської групи.
Половину цього часу можна проводити в реальних проектах, типу інтернатури,
але мов і матеріалів так багато, що для топового програміста 10 років можна виділити тільки на теорію.

Так, можна і в 50 років влаштуватися на галєру «цифровим сантехніком»,
але це теж ніяк не попадає під курс топового програміста, який повинен
покривати широкий діапазон дисциплін: від створення процесорів, асемблерів,
компіляторів, операційних систем, систем управління базами даних, мережевих протоколів,
сервісів, шин та додатків до теоретико-типових верифікаторів математичних
моделей та теорем, сертифікованих компіляторів, систем доведення теорем.

Мотивація настільки важлива, що без правильної мотивації висувати будь-які претензії
про марно втрачені 10 років життя абсолютно безрезультатно, сертифікат відкликається.
Як перевірити чистоту мотивації і наскільки точні можуть бути рекомендації?
Можу лиш сказати, що видо повинні бути достатньо чесним перед самим собою, адже програмування
--- це складний виснажливий процес, а мислення --- найвища форма управління організмом, тому
вади в його роботі можуть призвезти до непоправних наслідків.

Якщо крім програмування ви нічого не вмієте, то непогано було би розвинутив в собі первні
стратегії відступу: мінімальні техніки управління диханням та дієтою, легкий спорт без
фанатизму, трохи йоги, можливо активні види спорту. Якщо ви вважаєте, що у цілому
ви психічно стабільна людина, то пригутуйтеся до сюрпризів на шляху осягнення загадок
простоу без внутрішної чистоти намірів.

Моєю особистою мантрою, з якою я вивчаю програмування --- це посвята резульатів своє
роботи людям та всім істотам, не нашкодивши нікому без виключення.
Взагалі вивчення програмування мало кому може зашкодити та може мати форму глибокого відлюдництва
святого монаха. Хоча є виключення, програмісти, а особливо гарні програмісти,
в своїй більшості не жорстокі істоти, і їх надмірна агресія і сердитість, направлена
в позитивному ключі інстроспекції є двигуном аутичного осягнення потаємностей професії програміста.

Взагалі, якшо мотивація алзмазної візниці привести усі істот до абсолютного просвітлення
знається вам занадто езотеричної, то хочи би стара етично норма інженерів минулого 
«не нашколь, а краще допоможи людям» є тим мінімомум, який необхідно перевіряти перед кожною сесією
програмування. Уявіть собі, що ви з рвінням лева кладете на вівтар просвітлення 10 років
самоосвіти в області програмування зі скрині свого життя для того аби принести користь людям та суспільству.
Без подібної мотивації вам просто не бути звідки черпати енергію для щоденних вправ в програмуванні та мисленні.

\subsection{Всевідання як джерело натхнення}

Главная черта характера, которая необходима в человеке, чтобы стать топовым программистом – это предрасположенность к изучению и исследованию феноменов, их анализу, синтезу и абстракции. Это желание разобрать и исследовать игрушку должно так глубоко находиться в сознании, что кажется, будто ребенок уже рождается с этим даром быстро разбираться в феноменах при должной интенсивной нагрузке на нейросеточку. Другими словами – это хакерство, если вы любите исследовать системы, разбираться в программном коде, понимаете, как работают процессоры, знаете, как работает логика и математика, то вы уже можете стать топовым программистом. Желание построить максимально точную модель феномена должно быть гипертрофированным, оно должно быть незакрытым гештальтом, который не дает вам спать по ночам пока вы его не закроете. Именно эта фанатичная одержимость конвертируется в то, что будет дровами в нашем костре просветления на пути к всеведению в мире программирования. Откуда взялось всеведение? Это вторая сторона медали главного источника хакерского вдохновения. Если при локальном рассмотрении феноменов главной мыслью должно быть построить максимально точную модель феномена, то при фокусировке в бесконечность к краям горизонта, это желание проявляется в виде максимально быстрого познания вообще всех феноменов и их универсальные принципы устройства. Такой мета-хакерский трансцедентальный полу-фрический майндсет необходим для понимания того, насколько абстрактными и широкими могут быть вызовы на пути познания глубинных языков, на которых написана наша вселенная.
\\
Так, как языки программирования используются во всех сферах человеческой деятельности, то топовый программист совершенно точно должен разбираться во всех доменных моделях, всех типах и всех математиках, которые возникают в разных языках программирования. Обычно, университетские 5 лет я бы рекомендовал провести как раз в охвате всех математик и всех видов языков программирования, перед тем как погрузиться в фундаментальную математику и системное программирование. Вообще хорошее современное образование уровня PhD автоматически подразумевает свободное владение языковым и математическим обеспечением в исследовании вселенной, так что ничего такого, что не требуют топовые университеты, курс топового программиста в этой части всеведения здесь не требует. Требуется полная автономность на уровне полетел в космос и восстановил все знания и навыки при необходимости в кратчайшие сроки путем легкого воспоминания.
\\
Каждый раз поглощая какой-то пласт информации вы высвобождаете огромное пространство свободы, которое либо заполняется новыми неисследованными пластами, либо освобождается абсолютно, если вы уже все пласты поглотили. Но когда вы полностью исчерпаете всю карму, тогда буддахуд придет автоматически, так что это уже программа максимум. Ведь после того как вы изучили какой-то предмет и дали несколько мастер классов по нему, вы просто листаете все книги по нему, по которым учились и это все для вас даже не букварь, потому что букварь вы уже сами написали, это для вас просто шум деревьев в лесу. Вы полностью исчерпали этот предмет, стали мастером в нем, вы видите уже все сечения глобулярных фазовых пространств, имеете на руках несколько моделей и прототипов. Это состояние всеведения. Желание этого состояния – необходимый компонент топового программиста.
\\
Если вы построили какое-то пространство феноменов наполнив их смыслом и остаетесь там в комфортной среде ограниченного знания, вы уже теряете топовую мотивацию как компонент всеведения. Не устремив свое мышление в бесконечность вы не сможете увидеть весь ландшафт и правильно расставить приоритеты в поглощении научных дисциплин, чтобы совершить «прыжки Тигра» между этими приоритетными реперными точками.
\\
Только исследователи, которые преисполнены врождённого желания строить новые теории и языковые пространства наделены семенем творчества, которое ведет к топовой реализации. Междисциплинарный подход может появиться только в условиях широкого профиля. Никто никогда не ставил задачу составить курс для подготовки людей, которых можно было бы назвать культовыми хакерами, поэтому и требования к подготовке должны быть запредельными. Только олимпийское желание всеведения может реально приблизить вас к нему.

\normalsize
