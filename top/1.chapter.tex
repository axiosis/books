\section{Поклоніння простору}

\subsection{Топовий програміст}

Якщо мої передплатники і просять про якусь масштабну контрибуцію,
то це монографію на тему «як стати топовим програмістом».
Хоча таке формулювання інфантильне, воно досить добре
відображає сутність запитуваного: детальний розгляд професії
програміста, стратегію вивчення предмета виходячи з особистого
досвіду, розбавлений автентичною філософією.

\subsection{Висловлювання нескіченної поваги}

Перед тим як розпочати розповідь про професію програміста
насамперед хочеться висловити пошану предмету вивчення та
практики програмування, а саме формальним математичним
обчислювальним побудовам, у яких можливе програмування
в принципі. В останні роки стало зрозуміло, що простір
цих побудов настільки глибокий, що може поглинути не
тільки всі дискретні програми всіх формальних граматик,
а й континуальну математику, у якій робота з простором
йде на іншому, більш фундаментальному рівні. Тому без зайвого
перебільшення можна сказати, що саме простір народжує
мовну групу мов, які є первісною матрицю всіх без винятку
мов програмування.

Принцип глибокої поваги до предмету, який
майстер повинен реалізувати є одним із секретних ключів
східної філософії. Позаяк програмування народжується з
простору феноменологічних побудов, що ведуть до абстрактної
класифікації просторів та логік з ними пов'язаних, то утримання
у фокусі мети вивчення простору та програмування як
практичного людського процесу з цим пов'язаним є
головним завданням на шляху вивчення. Тому, без недооцінки та
зайвого перебільшення можна сказати, що простягання або поклоніння,
як прояв поваги до самого простору, як об'єкту вивчення,
виглядає для мене логічним. Я простягаюся перед простором.

\subsection{Перевірка мотивації}

Важливою характеристикою, яка, хочете вірте, а хочете ні,
впливає на процес вивчення мистецтва програмування, є чистота мотивації.
Якщо розглянути граничный популярний споживацький приклад, то він буде виглядати так:
ваша мотивація полягає у збільшенні своїх навичок програміста для
досягнення матеріальних благ і підвищення конкурентоспроможності
на ринку праці. Повна нісенітниця, така мотивація вприває на критерії
вибору об'єктів вивчення і це може завести вас у ситуацію, коли вам
50 років і ви пишете на Core Java для якось швейцарского банку.
Очевидно, що люди, які просили у мене цей текст, не очікують чогось подібного.

Свої ілюзії про легкість цього шляху можна відразу відкинути.
Цей шлях по-самурайськи скдадний і на нім сходили з розуму не тільки
випускники прикладної математики, немало людей перегоріло на
підприємствах від перенавантаження та неконтрольованості інформації.
Тому 10 років ув'язнення з постійним виділеним
каналом в інтернет на повному зовнішньому забезпеченні ---
ідеальний ресурс, який я рекомендував би виділити для успішної
підготовки на майстра програмування.

Чому такий великий термін буде пояснено в наступних частинах.
10 років цілком адекватний інтервал навчання для лікаря, то чому для програміста повинно бути менше.
Кількість мов якими кваліфікований програміст володіє на практиці може сягати тисяч,
за кожною з них стоїть теорія, своя логіка і своя математика нею породжена.
Це не просто латина, есперанто та пару мов романо-германської групи.
Половину цього часу можна проводити в реальних проектах, типу інтернатури,
але мов і матеріалів так багато, що для топового програміста 10 років можна виділити тільки на теорію.

Так, можна і в 50 років влаштуватися на галєру «цифровим сантехніком»,
але це теж ніяк не попадає під курс топового програміста, який повинен
покривати широкий діапазон дисциплін: від створення процесорів, асемблерів,
компіляторів, операційних систем, систем управління базами даних, мережевих протоколів,
сервісів, шин та додатків до теоретико-типових верифікаторів математичних
моделей та теорем, сертифікованих компіляторів, систем доведення теорем.

Мотивація настільки важлива, що без правильної мотивації висувати будь-які претензії
про марно втрачені 10 років життя абсолютно безрезультатно, сертифікат відкликається.
Як перевірити чистоту мотивації і наскільки точні можуть бути рекомендації?
Можу лиш сказати, що видо повинні бути достатньо чесним перед самим собою, адже програмування
--- це складний виснажливий процес, а мислення --- найвища форма управління організмом, тому
вади в його роботі можуть призвезти до непоправних наслідків.

Якщо крім програмування ви нічого не вмієте, то непогано було би розвинутив в собі первні
стратегії відступу: мінімальні техніки управління диханням та дієтою, легкий спорт без
фанатизму, трохи йоги, можливо активні види спорту. Якщо ви вважаєте, що у цілому
ви психічно стабільна людина, то пригутуйтеся до сюрпризів на шляху осягнення загадок
простору без внутрішної чистоти намірів.

Моєю особистою мантрою, з якою я вивчаю програмування --- це посвята резульатів своє
роботи людям та всім істотам, не нашкодивши нікому без виключення.
Взагалі вивчення програмування мало кому може зашкодити та може мати форму глибокого відлюдництва
святого монаха. Хоча є виключення, програмісти, а особливо гарні програмісти,
в своїй більшості не жорстокі істоти, і їх надмірна агресія і сердитість, направлена
в позитивному ключі інстроспекції є двигуном аутичного осягнення потаємностей професії програміста.

Взагалі, якшо мотивація алзмазної візниці привести усі істот до абсолютного просвітлення
знається вам занадто езотеричної, то хочи би стара етично норма інженерів минулого 
«не нашколь, а краще допоможи людям» є тим мінімомум, який необхідно перевіряти перед кожною сесією
програмування. Уявіть собі, що ви з рвінням лева кладете на вівтар просвітлення 10 років
самоосвіти в області програмування зі скрині свого життя для того аби принести користь людям та суспільству.
Без подібної мотивації вам просто не бути звідки черпати енергію для щоденних вправ в програмуванні та мисленні.

\subsection{Всевідання як джерело натхнення}

Головна риса характеру, яка необхідна в людині, щоб стати
топовим програмістом --- це схильність до вивчення та дослідження феноменів,
їх аналізу, синтезу та абстракції. Це бажання розібрати і
досліджувати іграшку має так глибоко перебувати у свідомості,
що здається, ніби дитина вже народжується з цим даром і
швидко розбиратися у феноменах за належного інтенсивного
навантаження на нейросіточку. Іншими словами --- це хакерство,
якщо ви любите досліджувати системи, розбиратися в програмному
коді, розумієте, як працюють процесори, знаєте, як працює
логіка та математика, то ви вже можете стати топовим програмістом.
Бажання побудувати максимально точну модель феномена має
бути гіпертрофованим, воно має бути незакритим гештальтом,
який не дає вам спати ночами, поки ви його не закриєте.
Саме ця фанатична одержимість конвертується в те, що буде
дровами у нашому вогнищі просвітлення на шляху до всезнавства
у світі програмування. Звідки взялося всезнання?
Це друга сторона медалі головного джерела натхнення хакера.
Якщо при локальному розгляді феноменів головною думкою має
бути побудувати максимально точну модель феномену, то при
фокусуванні в нескінченність до країв горизонту, це бажання
проявляється у вигляді максимально швидкого пізнання всіх
феноменів і їх універсальні принципи пристрою. Такий
мета-хакерський трансцедентальний напів-фрічний майндсет
необхідний для розуміння того, наскільки абстрактними і
широкими можуть бути виклики на шляху пізнання глибинних
мов, якими написано наш всесвіт.

Так, як мови програмування використовуються у всіх сферах
людської діяльності, то топовий програміст абсолютно точно
повинен розбиратися у всіх доменних моделях, усіх типах та
всіх математиках, які виникають у різних мовах програмування.
Зазвичай, університетські 5 років я б рекомендував провести
якраз у охопленні всіх математик та всіх видів мов програмування,
перед тим як поринути у фундаментальну математику та системне
програмування. Взагалі хороша сучасна освіта рівня PhD автоматично
має на увазі вільне володіння мовним та математичним
забезпеченням у дослідженні всесвіту, так що нічого
такого, що не вимагають топові університети, курс
топового програміста в цій частині всезнавства тут не вимагає.
Потрібна повна автономність на рівні полетів у космос і
відновив усі знання та навички за потреби у найкоротші
терміни шляхом легкого спогаду.

Щоразу поглинаючи якийсь пласт інформації ви вивільняєте
величезний простір свободи, який або заповнюється новими
недослідженими пластами, або звільняється абсолютно, якщо
вже всі пласти поглинули. Але коли ви повністю вичерпаєте
всю карму, тоді буддахуд прийде автоматично, тож це вже
програма максимум. Адже після того, як ви вивчили якийсь
предмет і дали кілька майстер класів по ньому, ви просто
гортаєте всі книги по ньому, за якими навчалися і це все
для вас навіть не буквар, тому що буквар ви вже самі написали,
це для вас просто шум дерев у лісі. Ви повністю вичерпали цей
предмет, стали майстром у ньому, ви вже бачите всі перерізи
глобулярних фазових просторів, маєте на руках кілька моделей
і прототипів. Це стан всезнавства. Бажання цього стану ---
необхідний компонент топового програміста.

Якщо ви побудували якийсь простір феноменів, наповнивши
їх змістом і залишаючись там у комфортному середовищі
обмеженого знання, ви вже втрачаєте топову мотивацію
як компонент всезнання. Не спрямувавши своє мислення
в нескінченність, ви не зможете побачити весь ландшафт
і правильно розставити пріоритети в поглинанні наукових
дисциплін, щоб здійснити <<стрибки Тигра>> між цими
пріоритетними реперними точками.

Тільки дослідники, які сповнені вродженого бажання будувати
нові теорії та мовні простори, наділені насінням творчості,
що веде до топової реалізації. Міждисциплінарний підхід може
виникнути лише за умов широкого профілю. Ніхто ніколи не
ставив завдання скласти курс для підготовки людей, яких
можна було б назвати культовими хакерами, тому й вимоги
до підготовки мають бути позамежними. Тільки олімпійське
бажання всезнання може реально наблизити вас до нього.

\normalsize
