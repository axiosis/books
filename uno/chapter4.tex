\chapter{Бібліотека середовища виконання}
\epigraph{Присвячується автору формальної системи F}{Жану-Іву Жирару}

Після побудови в розділі 3 формального середовища виконання,
яке складається з операційної системи у якій виконуються
CPS-інтерпретатори з формальною системою вводу-виводу IO, можна зразу
переходити до базової бібліотеки середовища виконання.

Даний розділ формалізує інтерфейс прикладного програмування та систему
бібліотек часу виконання для забезпечення потреб побудови гетерогенних систем та сервісів.

\section*{Вступне слово}
Так чи інакше для дослідження будь-якої базової бібліотеки середовища
виконання доведеться зустрітися з теорією яка стоїть за System F.
Навіть базова бібліотека фундаментальної вищої мови PTS в сутності
потребує для свого кодування лише системи Ф, так як є безпосереднім
портом з мови Haskell. Тому доречно використовувати у якості проміжної
типової системи систему Ф Жирара як цільову систему для
екстрактів з вищих мов, таких як HTS (якщо такі екстракти існують для окремих програм).

\section{Загальні принципи}
N2O.DEV — це формальна філософія та інженерна вправа водночас.
Вона обмежує автора бути ефективним та точним не втрачаючи
при цьому повноти та функціональності. Це накшталт внутрішньої
дисципліні при проектуванні програмного забезпечення. Ця філософія
багато років застосовується на практиці для побудови систем SYNRC,
та визначає стандартний мінімальний набір для демонстрації однієї
з сучасних моделей реактивного веб-програмування, яка включає: веб-сокет
веб-фреймвок з бінарною серілізацією, пушами та контролем DOM елементів
зі сторони сервера. N2O.DEV вчить будувати прості та надійні системи
на будь-якій мові програмування.

\section{Формальна специфікація}
Формальне середовище виконання визначає структуру операційних середовищ (runtimes) як
операційну систему лямбда-інтерпретаторів які працюють на паралельному
обчислювальному середовищі (ядрах процесорів). Кожне з ядер процесорів
виконує в нескінченному циклі команди лямбда-інтерпретаторів, переключаючи
через певний проміжок часу на потік команд іншого інтерпретатора. Таке
визначення дає змогу вбудувати цю структуру у віртуальну машину Erlang:
1) Головний процес додатку; 2) Супервізор додатку; 3) Проміжні
супервізори; 4) Кінцеві пул процесорів повідомлень.

Тут визначена специфікація програмного забезпечення усіх рівнів прикладної моделі
для підприємств на функціональних мовах програмування.
Ця специфікація визначає правила побудови WebSocket сервера,
бінарного серіалізатора та веб-фреймворку визначеному
формальними протоколами. Промислові версії також підтримують
систему управління бізнес-процесами та ефективне сховище
з глобальним простором ключів.

\subsection*{Серверні та клієнтські мови}
Мови програмування розділені на чотири рівня як для клієнтської
(мобільної та веб розробки) так і для серверної розробки (бекенд
системи). Нульовий рівень — тотальні формальні алгебраїчні мови
програмування, що забезбечують повноту, функціональність та доведення
властивостей прогрм згідно сучасних уявлень про математичне моделювання
та системи залежних типів побудованих на розшаруваннях. Перший
рівень — формальні функціональні мови програмування, як правило
System F, System Fω які успішно використовуються в промисловості
та забезпечують достатньо формальний запис який піддається
масштабуванню у великих командах завдяки потужному ядру компіляторів.
Другий рівень — неформальні (без формальної операційної чи денотаційної
семантики) чи формальних верифікаторів, які проте успішно
використовуються в промисловості, можуть бути як з розвиненими
системами типів з узагальненими шаблонами та типами сумами,
так і однотипними мовами програмування з динамічною типізацією.
Третій рівень — мови які погано піддаються масштабуванню у
промисловому виробництві (на основі спостережень за власним досвідом).

\begin{table}[h]
\begin{center}
\caption{Перелік кваліфікаційних рівніх верифікації}
\tabcolsep7pt\begin{tabular}{lcccc}
\hline
\textbf{Сторона/Рівень} & \textbf{Приклади мов} \\
\hline
    клієнт/3 & JavaScript, Lua \\
    клієнт/2 & Swift, Kotlin, TypeScript \\
\rowcolor{LightGray}
    клієнт/1 & UrWeb, OCaml, PureScript \\
\rowcolor{LightGray}
    клієнт/0 & Kind, PTS \\
    \hline
    сервер/3 & PHP, Python, Perl, Ruby \\
    сервер/2 & Erlang, Elixir \\
\rowcolor{LightGray}
    сервер/1 & SML, Scala, Haskell, F\#, Rust, Hamler \\
\rowcolor{LightGray}
    сервер/0 & Coq, Agda, Lean, MLTT/HTS \\
\hline
\end{tabular}
\end{center}
\end{table}
\end{lstlisting}

\subsection*{Обрані мови реалізації}

Тут перелічені мови, на яких реалізовано та апробовано N2O.DEV.

\textbf{Standard ML}\footnote{\url{https://github.com/o1}}. В академічних цілях Марат Хафізов створив
за специфікаєю N2O/NITRO порт на мови Standard ML (SML/NJ та MLton).
Ця робота представлена Github організацією O1 в структурі N2O.

\textbf{Haskell}\footnote{\url{https://github.com/o3}}. Перший експеримент з формалілазції N2O в систему F
була робота Андрія Мельникова. Пізніше, більш повну версію з NITRO протокол
запропонував Марат Хафізов, ця версія представлена на Github як організація O3.

\textbf{F\#}\footnote{\url{https://ws.erp.uno}}. Також у якості вправи Siegmentation Fault зробив порт NITRO
на мову F\# разом з ETF кодуванням. Ці напрацювання представлені в Github
організації O61.

\textbf{Lean}\footnote{\url{https://github.com/o89}}. У якості більшо формальної платформи з залежними типами,
мова Lean 4 від Леонардо де Мура з Microsoft. Siegmentation Fault автор порта,
який представлений Github організацією O89 та сайтом lean4.dev.

\textbf{Erlang}\footnote{\url{https://github.com/synrc}}. Основна промислова платформа, яка в повній мірі реалізує
специфікацію N2O.DEV.

\textbf{Elixir}\footnote{\url{https://github.com/synrc}}. Адаптація N2O.DEV для мови Elixir.
Основна імплементація не змінена, постійно підтримується публікація пакетів на HEX.PM.

\textbf{Hamler}. Нова формальна платоформа на базі PureScript для віртуальної машини Erlang.
В подальшому цей розділ буде присвячений імплементації та специфікації
на мові Hamler (варіація PureScript з бекендом в Erlang Core).

\begin{table}[h]
\begin{center}
\caption{Перелік мов для яких існує версія базової бібліотеки}
\begin{tabular}{lcc}
\hline
\textbf{Мова/Платформа} & \textbf{Набір реалізованих компонент} \\
\hline
Erlang/OTP & N2O, BPE, KVS, NITRO, MAD, FORM \\
Elixir/OTP & N2O, BPE, KVS, NITRO, FORM \\
\hline
\rowcolor{LightGray}
Hamler/OTP & RT, BASE, N2O, BPE, KVS, NITRO \\
\hline
Standard ML & N2O, NITRO, ETF \\
Haskell & N2O, NITRO, ETF \\
F\# & N2O, NITRO, ETF \\
Lean 4 & N2O, NITRO, MAD, ETF \\
\hline
\end{tabular}
\end{center}
\end{table}

Авторськими вважаються імплементації на мовах Erlang (Elixir) та Hamler (PureScript).
Інші представлені імплементації вважаються сертифікаваними формальними референсними моделями.

Протягом 8 років автор практикувався аби адаптувати бібліотеку середовища виконання
до основних формальних або напівформальних мов System F, які принаймі написані
на мовах які теж є мовами System F (SML, F\#, Haskell). Сторонніми дослідниками
була навіть зроблена адаптація для Lean 4\footnote{\url{https://o89.github.io/lean4.dev/}}.
В процесі цьої багаторічної вправи стало зрозумілим, що досягти на виробництві
тієї якості, яка досягається в середовищі виконання Erlang/OTP майже неможливо.
Тому модель базової бібліотеки, спочатку адаптувалася з оригінальної мови Erlang (2013)
для мови Elixir (2018), і потім для мови Hamler (2021), що є варіацією PureScript (System $F_\omega$).
Всього існує 7 моделей для 7 мов програмування базової бібліотеки представленої в цій роботі.

\section{Пакети формального середовища}
Тут представлена модель бібліотеки формального середовища виконання, яке
скаладається з JIT-інтерпретатора, потужної SMP-системе акторів з неблокуючими
курсорами черг повідомлень для системи процесів 1:1 (процес-черга, кожен процес
має свою чергу). Така спрощена модель дещо відрізняється від моделі CSP, CCS
та класичного числення процесів (Pi Calculus) проте протягом цих 8 років на практиці
стало зрозумілим, що модель Erlang/OTP більш гнучка до масштабування та стійка до помилок.

Іншими словами модель в System F базової бібліотеки
середовища виконання формально визначає цей прошарок уніфікованого мовного середовища.
А модель процесів хоча формально не відображає семантику коіндукції (стерта інформація),
проте досі синтаксис мовного середовища представленого в розділі 2 є сумісним з моделлю
Erlang/OTP яка є основною платформою, що підтримується у виробництві.

\subsection{Структури даних BASE}
Структури даних представлені додатком BASE. Основні модулі додатку:
List, Array, Atomics, Binary, Bool, Char, Counters, DateTime, Digraph,
Enum, Eq, Float, Int, Map, Maybe, Ord, OrdDict, OrdSet, Pid, Queue,
Read, Record, Regex, Set, Time, Tuple.

Сигнатури додатка BASE максимально сумісні з базовою бібліотекою Hamler.

\subsection{Сервіси середовища виконання RT}
Сервіси середовища виконання представлені додатком RT, іменні простори якого
дещо відрізняються від відповідних сигнатур базової бібліотеки мови Hamler.

\subsubsection{Database}
Модулі простору Database: Mnesia, ETS.

\begin{lstlisting}
axiom all : IO (List TableId)
axiom browse : IO String
axiom delete : Π (k: U), TableId -> k -> IO Unit
axiom first : Π (t: TableId) (k: U), IO (Maybe k)
axiom last : Π (k: U), TableId -> IO (Maybe k)
axiom foldl : Π (v acc: U), (v -> acc -> acc) -> acc -> TableId -> IO acc
axiom foldr : Π (v acc: U), (v -> acc -> acc) -> acc -> TableId -> IO acc
axiom insert : Π (v: U), TableId -> v -> IO Boolean
axiom lookup : Π (k v: U), TableId -> k -> IO (List v)
axiom member : Π (k: U), TableId -> k -> IO Boolean
axiom new : Atom -> TableOptions -> IO TableId
axiom next : Π (k: U), TableId -> k -> IO (Maybe k)
axiom prev : Π (k: U), TableId -> k -> IO (Maybe k)
axiom rename : TableId -> Atom -> IO Atom
axiom take : Π (k v: U), TableId -> k -> IO (List v)
axiom match : Π (a v: U), TableId -> a -> IO (List v)
axiom slot : Π (v: U), TableId -> Integer -> IO (Maybe (List v))
\end{lstlisting}

\subsubsection{Filesystem}
Модулі простору Filesystem: Dir, File, FilePath, IO, Active.

\subsubsection{OS}
Модулі простору OS: Env, Init, Shell.

\subsubsection{Process}
Модулі простору Process: Application, Supervisor, Dict, GenServer, Spawn, Process, Timer.

\subsubsection{Network}
Модулі простору Network: TCP, UDP, TLS, Inet, WebSocket.

\newpage
\subsection{Прикладні протоколи N2O}
Сервіси веб-сокет сервера, представлені додатком N2O: N2O, PI, Proto,
Ring, MQTT, WS, TCP, Heart, Syn, FTP, NITRO, ETF, GCM, Session, Cache.

Модуль N2O пропонує ряд сервісів зручних та вивірених в промисловій роботі для побудови
сервісних протоколів що вбудовуються в цикли сучасних TCP, QUIC, UDP, MQTT, WebSocket серверів,
та в процесі своєї роботи можуть стартувати додаткові процеси для обробки інформації під
супервізією середовища виконання.

\begin{lstlisting}
axiom pickle : Binary -> Binary
axiom depickle : Binary -> Binary
axiom encode : Π (k: U), k -> Binary
axiom decode : Π (k: U), Binary -> IO k
axiom reg: Π (k: U), k -> IO k
axiom unreg : Π (k: U), k -> IO k
axiom send : Π (k v z: U), k -> v -> IO z
axiom getSession : Π (k v: U), k -> IO v
axiom putSession : Π (k v: U), k -> v -> IO v
axiom getCache : Π (k v: U), Atom -> k -> IO v
axiom putCache : Π (k v: U), Atom -> k -> v -> IO v
\end{lstlisting}

Тут залишено лише саме необхідне, але не настільки тривіальне аби бути іграшковим.
Всі функції всії сервісів можуть підмінятися в ході виконання.
В сутності тут зібрані сервіси:
1) бінарної серіалізації Erlang Term Format (ETF), функції encode/decode;
2) функції симетричного шифрування AES-GCM/128 pickle/depickle;
3) функції Pub/Sub внутрішнього Erlang реєстра (syn/gproc/global, тощо);
4) функції роботи з персональними сесіями, які захищені зашифрованими токенами;
5) функції роботи з глобльним кеш-сервісом для бізнес-об'єктів.

Модуль N2O.PI абстрагує користувача від надмірного
фольклору Supervisor та пропонує простіший протокол запуску
асинхронних процесів.

\begin{lstlisting}
data PI  = PI String Atom Atom Atom Integer RestartType
data Sup = Ok Pid String | Error String

axiom start : PI -> IO Sup
\end{lstlisting}

\newpage
\subsection{Сховище даних KVS}
Сервіси сховища даних представлені додатком KVS: KVS,
Stream, ST, Rocks, Mnesia, FS.

\begin{lstlisting}
axiom get : Π (f k v: U), f -> k -> IO (Maybe v)
axiom put : Π (r: U), r -> IO StoreResult
axiom delete : Π (f k: U), f -> k -> StoreResult
axiom index : Π (f p v r: U), f -> Atom -> v -> List r
\end{lstlisting}

\begin{lstlisting}
data Reader = Reader Integer Binary ETF String Integer
data Writer = Writer Integer Binary ETF String Integer
data StoreResult = Ok Integer String Binary
                 | Error Integer String Binary

axiom next : Reader -> IO Reader
axiom prev : Reader -> IO Reader
axiom take : Reader -> IO Reader
axiom drop : Reader -> IO Reader
axiom save : Reader -> IO Reader
axiom append : Π (f r: U), f -> r -> IO StoreResult
axiom remove : Π (f r: U), f -> r -> IO StoreResult
\end{lstlisting}

\subsection{Бізнес-процеси BPE}
Сервіси системи управління бізнес-процесами
представлені додатком BPE: BPE, Event, Action, Process, Hist, Flow.

\begin{lstlisting}
axiom start : Proc -> IO Sup
axiom stop : String -> IO Sup
axiom next : ProcId -> IO ProcRes
axiom load : ProcId -> IO ProcRes
axiom proc : ProcId -> IO ProcRes
axiom assign : ProcId -> IO ProcRes
axiom persist : ProcId -> Proc -> IO ProcRes
axiom amend : Π (k: U), ProcId -> k -> IO ProcRes
axiom discard : Π (k: U), ProcId -> k -> IO ProcRes
axiom modify : Π (k: U), ProcId -> k -> Atom -> IO ProcRes
axiom event : ProcId -> String -> IO ProcRes
axiom head : ProcId -> IO Hist
axiom hist : ProcId -> IO (List Hist)
axiom step : ProcId -> Atom -> IO (List Hist)
axiom docs : ProcId -> IO (List Tuple)
axiom events : ProcId -> IO (List Tuple)
axiom tasks : ProcId -> IO (List Tuple)
axiom doc : Tuple -> ProcId -> IO (List Tuple)

data ProcId = String
data Proc = Proc ProcId String
data ProcRes = Ok Integer String Binary
             | Error Integer String Binary

\end{lstlisting}

\subsection{Контрольні елементи протоколу NITRO}
Сервіси веб-фреймворка, представлені додатком NITRO: NITRO, Combo,
Edit, Form, Input, Table, Actions, Render.

\begin{lstlisting}
axiom q : Π (k: U), Atom -> k
axiom qc : Π (k: U), Atom -> k
axiom jse : Maybe Binary -> Binary
axiom hte : Maybe Binary -> Binary
axiom wire (actions: List Action) : IO (List Action)
axiom render (content: Either Action Element) : Binary
axiom insert_top (dom: Atom) (content: List Element) : IO (List Action)
axiom insert_bottom (dom: Atom) (content: List Element) : IO (List Action)
axiom update (dom: Atom) (content: List Element) : IO (List Action)
axiom clear (dom: Atom) : IO Unit
axiom remove (dom: Atom) : IO Unit
\end{lstlisting}

NITRO – це Nitrogen-подібний веб фреймворк для Erlang/OTP.
Він може бути використаний не лише у веб-додатках, а ще
і в консольних програмах, у яких потрібно зробити рендеринг HTML5.

Nitrogen Elements – це трансформатор шаблонів HTML для
мови Erlang, у якому всі HTML теги рендеряться з Erlang рекордів.

Працюючи з N2O вам взагалі не потрібно користуватись HTML.
Натомість ви визначаєте вашу сторінку у вигляді
Erlang рекордів, відповідно сторінка генерується
з перевіркою типів, під час компіляції. Це класичний
підхід CGI для компільованих сторінок, який надає всі
переваги перевірки помилок під час компіляції,
та визначає DSL для клієнт- та серверного рендерингу.

Nitrogen Elements, за своєю природою, є примітивними UI
елементами управління, які можуть бути використані для
побудови Nitrogen сторінок з внутрішнім DSL Erlang-а.
Вони компілюються в HTML та JavaScript. Поведінка всіх
елементів контролюється на стороні сервера, а весь
зв'язок між веб-переглядачем та сервером здійснюється
за допомогою WebSocket каналів. Отже, вам не потрібно
використовувати POST запити чи HTML форми. Тобто:

\begin{lstlisting}
 #textbox { id=userName, body= <<"Anonymous">> },
 #panel { id=chatHistory, class=chat_history }
\end{lstlisting}

згенерує наступний html:

\begin{lstlisting}
 <input value="Anonymous" id="userName" type="text"/>
 <div id="chatHistory" class="chat_history"></div>
\end{lstlisting}

А

\begin{lstlisting}
  nitro:update(loginButton,
      #button{id = loginButton,
          body = "Login",
          postback = login,
          source = [user, pass]});
\end{lstlisting}

згенетурє наступне:

\begin{lstlisting}
  qi('loginButton').outerHTML='<button id=\"loginButton\"
    type=\"button\">Login</button>'; { var x=qi('loginButton');
    x && x.addEventListener('click',function (event){
      event.preventDefault(); { if (validateSources(['user','pass'])) {
      ws.send(enc(tuple(atom('pickle'),bin('loginButton'),
        bin('b840bca20b3295619d1157105e355880f850bf0223f2f081549dc
        8934ecbcd3653f617bd96cc9b36b2e7a19d2d47fb8f9fbe32d3c768866
        cb9d6d85700416edf47b9b90742b0632c750a4240a62dfc56789e0f5d8
        590f9afdfb31f35fbab1563ec54fcb17a8e3bad463218d6402f1304'),
        [tuple(tuple(string('loginButton'),bin('detail')),[]),
        tuple(atom('user'),querySource('user')),
        tuple(atom('pass'),querySource('pass'))]))); }
    else console.log('Validation Error'); }});};
\end{lstlisting}

Для подальної інформації дивіться актуальную
документацію\footnote{\url{https://n2o.dev/ua/deps/nitro/index.html}}

\section{Висновки}

Кожна мова, на якій реалізовано N2O.DEV, має вбудовувати
свою філософію максимально природно та компактно. Якщо
потрібен якийсь шар між базовою бібліотекою мови,
його можна надати, але, якщо можливо, його слід
зменшити до нуля. У деяких випадках деякі частини
базової бібліотеки можна замінити кращою альтернативою.
N2O має надати клієнтську бібліотеку-компаньйон,
зазвичай реалізовану на іншому наборі мов клієнта:
JavaScript, Swift, Kotlin. Якщо ви все зробили
правильно, значення N2O не повинно перевищувати
500 LOC на будь-якій мові.
