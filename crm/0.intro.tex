\chapter{Огляд МІА: СЕД}

МІА:СЕД --- рішення для підтримки введення даних, індексування, обробки
документів, управління доступом, маршрутизації документів, системної
інтеграції та зберігання.

В основі базової версії інформаційної системи представлені:

--- Консистентність --- одноразова реєстрація документів, без потенційного
дублювання, та невалідного стану.

--- Персистентність --- єдина інформаційна база для централізованого зберігання
документів, їх додатків (копії паперових версій) та індексації.

--- Паралельність — можливість паралельного виконання різних операцій з
метою скорочення часу руху документів і підвищення оперативності їх
виконання.

--- Індексація --- ефективно організована система пошуку документу.

--- Оркестрація --- можливість визначення формальних бізнес-процесів, згідно
яких функціонує документообіг підприємства.

--- Звітність --- розвинена система звітності, що дозволяє контролювати рух
документу в процесі документообігу, вимірювати показники ефективності,
та на основі них змінювати бізнес-процеси системи.

Принципи для розгортання МІА:СЕД в середовищах визначають
можливості розвитку базової версії системи:

--- Неперервність --- система повинна експлуатуватися тривалий проміжок часу,
протягом якого усі зміни повинні бути відтворювані по трансакційному
журналу подій.

--- Відмовостійкість --- робота по ненадійних каналах зв’язку та в умовах
обмежених ресурсів.

--- Відкритість --- МІА:СЕД базується на відкритому програмному забезпеченні;
система надає API для можливого доопрацювання та інтеграції з іншими
системами.

--- Розширюваність --- розширення, доповнення та оновлення програмнотехнічних засобів без змін програмного і інформаційного забезпечення,
зміна функцій і складу системи без зупинки та порушення її
функціонування.

--- Розподіленість --- підтримка роботи з документами у територіальнорозподілених організаціях, а також взаємодії з віддаленими користувачами
по відкритих протоколах.

--- Гетерогенність --- робота зі стаціонарних чи мобільних пристроїв, що мають
вихід в Інтернет, незалежно від операційної системи, встановленої на них, у
режимі 24/7, підтримка сучасних стандартів.

