\chapter{Можливості МІА: СЕД}

У відповідності з основними принципами організації електронного
документообігу, МІА:СЕД здійснює:

\section{Централізоване управління документами}

Централізоване управління документами --- МІА:СЕД дозволяє оперативно
змінювати форми документів, що використовуються в організації.

\section{Підтримка життєвого циклу документів}

Підтримка життєвого циклу документів --- МІА:СЕД дозволяє контролювати
життєвий цикл документів з урахуванням вимог корпоративного середовища, а
також галузевих стандартів і законодавства:
1) реєстрація документів за різними групами (вхідні, вихідні, внутрішні,
звернення громадян, адвокатські запити тощо), з можливістю розподілу
кожної з груп документів на довільні підгрупи;
2) робота з дорученнями (видача, виконання і контроль);
3) робота з проєктами документів (створення, редагування зі збереженням
попередніх версій, погодження та затвердження (підписання), реєстрація
документу, створеного на основі проекту, підтримка форматів DOC, DOCX,
PDF);
4) реалізація розсилки по різних каналах зв’язку;
5) зберігання та архівування;
6) оперативний багатокритеріальний пошук як по всім можливим полям
документів (базова версія), так і по складним логічним умовам
(опціонально);
7) побудова аналітичної звітності за параметрами.

\section{Колективна робота}

Колективна робота --- МІА:СЕД дозволяє організувати спільну роботу над
документом в межах єдиної інфраструктури.

\section{Забезпечення конфіденційності}

Забезпечення конфіденційності --- МІА:СЕД забезпечує можливість
підписувати документи за допомогою електронного підпису і зашифровувати їх.

\section{Управління доступом}

Управління доступом --- МІА:СЕД дозволяє розмежувати повноваження, доступ
до документів і дій над ними співробітникам організації та здійснювати контроль
за доступом до документів.

\section{Оркестрація документів}

Оркестрація документів --- МІА:СЕД забезпечує автоматичну передачу
документу потрібній особі та дозволяє:
1) проектування бізнес-процесів для документообігу, логічних маршрутів
документів з можливістю послідовного та паралельного їх виконання;
2) відправку документів як за типовими, раніше спроектованими, так і за
новими, визначеними користувачем у процесі виконання завдання,
маршрутами;
3) підтримка стандарту BPMN 2.0 та відкритого візуального конструктора
бізнес-процесів bpmn.io.

\section{Організація зберігання документів}

Організація зберігання документів --- МІА:СЕД забезпечує централізоване
управління зберіганням доку- ментів, компоненти:
1) сховище атрибутів --- реєстр контрольно-моніторингових карток (РК) —
набір метаданих і реквізитів, передбачених стандартами українського
діловодства;
2) сховище документів --- робота з файлами: перегляд, редагування, видалення,
підписання електронним підписом, розмежування прав доступу (як на рівні
РК, так і самого електронного документа);
3) сервісів повнотекстової індексації --- організація індексування документів
для подальшого їхнього швидкого пошуку.

\section{Формування справ}

Формування справ --- МІА:СЕД дозволяє формувати номенклатури справ,
правила списання документа в справу.

\section{Архівна справа}

Архівна справа --- МІА:СЕД дозволяє формувати централізоване, структуроване
й систематизоване зберігання незмінних оригіналів електронних документів в
архівному фонді організації на основі законів і правил ведення архівів.
2.10 Інтеграція з системами --- МІА:СЕД може бути інтегрована з системами
управління підприємством (ERP, CRM), системою електронної взаємодії
СЕВ ОВВ v.2.0 та іншими системами відповідно до процедурних і технічних
вимог, що встановлені нормативними актами.

\section{Моніторинг}

Моніторинг --- МІА:СЕД надає можливість проведення комплексного та
безперервного збору, обробки, систематизації та аналізу інформації про стан
виконання управлінських рішень (електронних документів) в установі.

\section{QR-код та штрих-кодування документів}

QR-код та штрих-кодування документів --- МІА:СЕД надає можливість друку
унікальних кодів на документ.

\section{Довідники}

Довідники --- підтримка і синхронізація довідників.

\section{Шина}

Шина --- внутрішня шина для системних повідомлень та нагадувань, а також
для забезпечення взаємодії між продуктами МІА:СЕД.



