\section{Формалізація буддизму}

Зараз я дам вам відчути смак математичної формальної філософії по-справжньому!
А то вам може здатися, що це канал з формальної математики, а не
формальної філософії. Я ж вважаю, що якщо формальна філософія не
спирається на формальну математику, то гріш ціна такій формальній філософії.

\begin{lstlisting}
module buddhism where
import path
\end{lstlisting}

Сьогодні ми будемо формалізувати поняття недвоїстості в буддизмі,
яке пов'язане одразу з багатьма концепціями на рівнях Сутри, Тантри
та Дзогчена: поняттям взаємозалежного виникнення та поняттям порожнечі
всіх феноменів (Сутра Праджняпараміти). Класичний приклад із
розчленовуванням тіла ставить питання, коли тіло перестає бути
людиною-істотою, якщо від нього почати відрубувати шматки м'яса (ми
буддисти любимо і лілеєм такі уявні образи-експерименти) або іншими
словами, щоб відрізнити тіло від не-тіла, нам потрібен двомісний
предикат (родина типів), функція, яка може ідентифікувати конректні
два еклемпляри тіла. Практично йдеться про ідентифікацію двох об'єктів,
тобто про звичайний тип-рівность Мартіна-Льофа.

За фреймворк візьмемо концепти Готтлоба Фреге, згідно з визначенням,
концепт - це предикат над об'єктом або, іншими словами, Пі-тип Мартіна-Льофа,
індексований тип, сім'я типів, тривіальне розшарування тощо. Де об'єкт x з o
належить концепту, якщо сам концепт, параметризований цим об'єктом,
населений p(o) : U (де p : concept o).

\begin{lstlisting}
concept (o: U): U
   = o -> U
\end{lstlisting}

Концепт p повинен надавати приклад чи контрприклад розрізнення,
тобто щоб визначити тіло це чи не тіло ще, поки ми його розчленовуємо,
нам потрібно як мінімум два шматки: тіло і не тіло як приклади ідентифікації.
Таким чином, недвоїстість може бути представлена як рівність між усіма
розшаруваннями (предекатами над об'єктами).

\begin{lstlisting}
nondual (o: U) (p: concept o): U
   = (x y: o) -> Path U (p x) (p y)
\end{lstlisting}

Отже, недвоїстість усуває різницю між прикладами і контрприкладами
на примордіальному рівні мандали MLTT, тобто ідентифікує всі концепти.
Сама ж ідентифікація класів об'єктів, які належать різним концептам --- це умова,
що стискає всі об'єкти в точку, або стягуваний простір, вершина конуса
мандали MLTT, або, іншими словами, порожнеча всіх феноменів виражена
як тип логічної одиниці, який містить лише один елемент.

\begin{lstlisting}
allpaths (o: U): U
   = (x y: o) -> Path o x y
\end{lstlisting}

Формулювання буддійської теореми недвоїстості, яка поширюється
всі типи учнів (тупих, середніх і тямущих), може звучати так:
недвоїстість концепту є спосіб ідентифікації його об'єктів. Сформулюємо
цю саму теорему в інший бік: спосіб ідентифікації об'єктів задає
предикат неподвоїстості концептів. Туди - ((p: concept o) -> nondual o p) -> allpaths o,
Сюди - allpaths o -> ((p: concept o) -> nondual o p). І доведемо її!
Як видно з сигнатур нам лише треба побудувати функцію транспорту між
двома просторами шляхів: (p x) =U (p y) і x =o y. Скористаємося
приведенням шляху до стрілки (coerce) та конгруентності (cong) з
базової бібліотеки.

\begin{lstlisting}
forward (o:U): ((p: concept o) -> nondual o p) -> allpaths o
   = \(nd: (p: concept o) -> nondual o p) (a b: o) ->
      coerce (Path o a a) (Path o a b) (nd (\(z:o)->Path o a z) a b) (refl o a)

 backward (o:U): allpaths o -> ((p: concept o) -> nondual o p)
   = \(all: allpaths o)(p: concept o)(x y: o) -> cong o U p x y (all x y)
\end{lstlisting}

Як бачите, теоремка про порожнечу всіх феноменів вийшла на кілька рядків,
які демонструють: 1) основи формальної філософії та швидке занурення в
область математичної філософії; 2) гарний приклад до першого розділу
HoTT на простір шляхів та модуль path.



