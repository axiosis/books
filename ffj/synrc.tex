% Copyright (c) 2022 Synrc Research Center

\usepackage[osf]{mathpazo}
\usepackage{musicography}
\usepackage{microtype}
\usepackage{stmaryrd}
\usepackage{color,colortbl}
\usepackage{epigraph}
\usepackage[mathcal,mathbf]{euler}
\usepackage{amsmath,amssymb,amsthm,wasysym}
\usepackage{graphicx,sidecap,tikz}
\usepackage{fullwidth}
\usepackage[T1]{fontspec}
\usepackage{hyphenat}
\usepackage{ifthen}
\usepackage{yfonts}
\usepackage{tikz-cd}
\usepackage[english,russian]{babel}
\usepackage{tabstackengine}
\usepackage{graphicx}
\usepackage{cite}
\usepackage{hyperref}
\usepackage{moreverb}
\usepackage{listings}
\usepackage{caption}
\usepackage{newunicodechar}

\usepackage[english,russian]{babel}
\usepackage{xeCJK}
\usepackage{fontspec}
\usepackage{graphicx,changepage,txfonts}
\usepackage{listings}
\usepackage{newunicodechar}
\usepackage{amsmath}
\usepackage{hyphenat}
\usepackage[top=18mm, bottom=25.4mm,
            inner=15mm,outer=18mm,
            paperwidth=142mm, paperheight=200mm]{geometry}

\hyphenation{}
\fontencoding{T1}
\addto\captionsrussian{\renewcommand{\contentsname}{Зміст}}

\newunicodechar{ꑭ}{ꑭ}
\newunicodechar{❤}{❤}

\newcommand{\includeimage}[2]
{\begin{figure}[h!]
\centering
\includegraphics[width=\textwidth]{#1}
\caption{#2}
\end{figure}
}

\newcommand*{\titleFFJ}
{
\newfontfamily{\cyrillicfont}{Geometria}
\setmainfont{Geometria}
    \begingroup
        \thispagestyle{empty}
        \hspace*{0.15\textwidth}
        \rule{1pt}{\textheight}
        \hspace*{0.05\textwidth}
        {
        \parbox[c][][s]{0.75\textwidth}
        {
             \vspace{-15cm}
            \setmainfont{DDC Uchen}
             \textsc{}
            \\
             \textsc{\noindent
             \setmainfont{Geometria}
             Максим Сохацький \\ [0.3\baselineskip]
             \\
             \Large
             Формальна філософія \\[0.5\baselineskip]
             \\
             \small
             Свідомість \\
             Розшарування Хопфа \\
             Формалізація буддизму \\
             Хроматична теорія гомотопій \\
             Геометрія в модальній HoTT \\
             Lean конференція \\
             Категорії Квілена \\
             Модальна гомотопічна теорія \\
             Метафілософія \\
             Прикладна математика \\
             Абелеві категорії \\
             Мова простору \\
             Суперпростір \\
             Топовий програміст \\
             Формальна Йогачара \\
             }
        }}
    \endgroup
    \setmainfont{Geometria}
}

\lefthyphenmin=1
\hyphenpenalty=100
\tolerance=6000

\fontencoding{T1}
\newfontfamily{\cyrillicfont}{Geometria}
\setmainfont{Geometria}

\newcommand{\ua}{\setmainfont{Geometria}}
\newcommand{\ti}{\setmainfont{DDC Uchen}}

\usetikzlibrary{matrix}
\usetikzlibrary{babel}

\usetikzlibrary{arrows,positioning,decorations.pathmorphing,trees}

\addto\captionsrussian{\renewcommand{\contentsname}{Зміст}}
\addto\captionsrussian{\renewcommand{\bibname}{Список використаних джерел}}
\addto\captionsrussian{\renewcommand{\chaptername}{Розділ}}
\addto\captionsrussian{\renewcommand{\tablename}{Таблиця}}

\newlength\tindent
%\setlength{\tindent}{\parindent}
%\setlength{\parindent}{0pt}
\renewcommand{\indent}{\hspace*{\tindent}}

\def\lstlanguagefiles{syntax.tex}
\lstset{language=infinity}
