\section{Хроматична теорія гомотопій}

Поговоримо про хроматичну теорію гомотопій. Я маю на увазі,
що ви трохи знайомі з терією категорій і топологією.

Отже, передусім, припустимо, ви вірите в те, що намагатися
класифікувати топологічні простори гомотопічним типом не марна витівка.
Це практично неможливо, проте ціль ця шляхетна. Щоб полегшити собі завдання,
ми будемо розглядати базові простори, які можуть бути створені приєднанням клітин.
Вони іноді називаються CW-комплексами чи клітинними комплексами. Коли я
говорю "склеювання клітин", я маю на увазі конструювання пушауту для
конуса $D^n \leftarrow S^{n-1} \rightarrow X$, де $X$ --- деякий простір,
$D^n$ --- $n$-диск, а $S^{n-1}$ --- його межа. Буквально - склеювання диска на його кордоні.

Виявляється, що будь-який гарний простір (компактно згенерований хаусдорфово) може
бути побудований таким чином, починаючи з простих точок, хоча вам, можливо,
доведеться приєднати нескінченну кількість осередків. Іншими словами, якщо
єдиними будівельними блоками, які ми маємо, є осередки, такі як $D^n$, ми
можемо, аж до гомотопії, створювати найкрасивіші топологічні простори (наприклад,
усі ті, що з'являються у таких додатках, як диференціальна геометрія).

Отже, щоб зрозуміти, які нові простори ми можемо побудувати з існуючого простору
X приєднанням осередків, достатньо знати всі способи, якими сфера може безперервно
відображатися сама в собі (бо ми прикріплюємо осередки з використанням сфер відображення).
Ви, напевно, вже знаєте, що безліч способів відображення сфери в інший топологічний
простір X називається гомотопічними групами X. Тому, якщо ми можемо обчислити
гомотопічні групи, ми можемо класифікувати (гарні) простори з точністю до гомотопій.
Звичайно, ви також можете знати, що обчислювати гомотопічні групи топологічних
просторів дійсно дуже, дуже складно. Зокрема, найпростіше питання, яке ви можете
поставити --- це які простори можна отримати склейками будь-яких сфер ($S^n$
до $S^m$ для будь-яких n і m). Це буде набір груп із двома індексами n і m,
один для розмірності сфери домену та один для розмірності сфери кодомена.
Знання цього могло стати непоганим стартом.

Погані новини: ми не знаємо ці групи, і ніколи можливо не дізнаємось. Ми знаємо
безліч їх, і ми хороші в обчисленні груп для фіксованих n і m, якщо добре
постараємося, але несхоже щоб там був певний очевидний патерн для генерації
всіх таких груп. Ну добре, ми можемо принаймні спробувати, і сподіватися,
що ми побачимо прикольні штуки дорогою вирішення цього завдання. Коли це
все починалося, не було відомо, наскільки це буде складним (сходить до Пуанкаре),
тому ми просунулися досить далеко, перш ніж зрозуміли, що справа погана. Зрештою,
хроматична гомотопічна теорія є спробою розбити вищеописані гомотопічні групи
сфер на будівельні блоки, які легше зрозуміти і з якими легше працювати.
Слово "хроматичний" відноситься до складових довжин хвиль, в які "розкладається" біле світло.

Сподіваюся, ви знаєте, що для сфери $S^n$ існує відображення "ступеня p",
яке обертає сферу $S^n$ навколо себе p разів. Уявімо це відображення як
$p: S^n \rightarrow S^n$. Це в точності те p, що ви побачите, коли згадаєте,
що n-та гомотопічна група сфери $S^n$ --- це цілі числа $\mathbb{Z}$.
Зауважте, що це відображення генерує ціле сімейство відображень $S^n \rightarrow S^n$,
задане ітеруванням p . Тобто. $p^k: S^n \rightarrow S^n$ для будь-якого k.
Таким чином у нас обчислилися деякі гомотопічні групи, але вони не такі вже й цікаві.
Одну річ, яку ми можемо зробити --- це причепити клітину уздовж цього
відображення (або будь-якої його ітерації), щоб отримати новий простір,
який я запишу як $V(0)$, або $S^n\ mod\ p$. Зауважте, що пушаут, який визначає
причеплену клітину, зробив вихідне відображення p гомотопним нулю (або гомотопічно
тривіальним). Зауважте, що $V(0)$ це лише (n+1)-сфера приклеєна до n-сфери,
існує включення нижньої сфери $S^n$ в $V(0)$, назвемо це i, і відображення
$V(0) \rightarrow S^{n+1}$, яке стягує нижню сферу, назвемо це q. Таким чином,
якби я мав інше відображення $f: \Sigma V(0) \rightarrow V(0)$, де $\Sigma$ ---
надбудова (суспензія), тоді я міг би зліва закомпозити це з i, а потім
праворуч закомпозити з q ($i \cdot f \cdot q$), щоб отримати нове
відображення $S^{n+1} \rightarrow S^{n+1}$, яке було б свого роду
породженим за допомогою f. Зверніть увагу, що роль, яку відіграє
надбудова, полягає у збільшенні розмірності нижньої сфери $V(0)$. Інакше ми
мали б відображення $S^n \rightarrow S^{n+1}$, і будь-яке таке відображення було б
гомотопічно тривіальним (знецінюючи наші дії).

Причина, через яку ми це все робимо, полягає в тому, щоб просто знайти ДЕЯКІ
елементи гомотопічних груп сфер. Загалом, нам знадобилося чимало часу, щоб
знайти БУДЬ-ЯКІ елементи гомотопічних груп сфер, а тим більше спробувати
обчислити ВСІ їх. Таким чином, це була велика справа, коли люди як Адамс і Тода,
змогли показати що, ТАК, є відображення $\Sigma V(0) \rightarrow V(0)$ (тут упускаються деталі,
насправді вам потрібна більше ніж одна надбудова). Більше того, це відображення,
назвемо його $A: \Sigma V(0) \rightarrow V(0)$, може бути итерировано нескінченне
число разів, не стаючи при цьому гомотопічно тривіальним. І щоразу, коли ми ітеруємо,
ми збільшуємо розмірність. Отже, у нас є вся родина відображень сфер, що
виходять з (ітеруючих) А. Під ітеруванням, я маю на увазі, що у мене є
відображення $A: \Sigma V(0) \rightarrow V(0)$, тому я можу отримати
відображення $\Sigma A: \Sigma\Sigma V(0) \rightarrow \Sigma V(0)$,
а потім інше відображення
$\Sigma\Sigma A: \Sigma\Sigma\Sigma V(0) \rightarrow \Sigma\Sigma V(0)$, і так до нескінченності,
де розмір доменної сфери буде ставати все більше і більше. Якщо це спрацювало
одного разу, то чому б не зробити це знову? Так само, як ми затюніли p раніше,
давайте візьмемо коядро відображення A (яке насправді називається корозшаровуванням).
Іншими словами, стягуйте все, що потрапляє в A, даючи нам точну послідовність
просторів $V(0) \rightarrow V(0) \rightarrow V(0)/A$.

Давайте перейменуємо $V(0)/A = V(1)$. Тепер вам просто потрібно повірити,
що $V(1)$ виходить шляхом приєднання осередків (змішуючи склеювання, які ми
зробили для побудови $V(0)$), і тому все ще існують відображення з нижньої
сфери i: $S^k \rightarrow V(1)$ і фактор групи до верхньої сфери $q: V(1) \rightarrow S^k$.
Сміт показав, що існує інше відображення $B: \Sigma V(1) \rightarrow V(1)$,
яке ми можемо ітерувати стільки разів, скільки ми захочемо, і ми отримаємо
ІНШЕ велике сімейство відображень між сферами. Отже, у нас є два великі
сімейства відображень сфер (всіх можливих вимірів!), Я назву їхню A-родину
і B-сімейство. Виявляється, якщо ви знову "затюніте" B, ви отримаєте новий
простір, назвемо його $V(2)$, і ви можете повторити цей процес ще раз.
І справді, існує карта $C: \Sigma V(2) \rightarrow V(2)$, що дає нам інше
сімейство, я назву його C-сімейством гомотопічних груп сфер. Тому тут
виникають два природні питання: 1) чи можна повторювати це нескінченно?
2) чи отримаємо ми всі гомотопічні групи сфер?

У певному сенсі перлина хроматичної гомотопічної теорії --- це позитивна
відповідь на ці два питання (знову ж таки, упускаючи багато деталей).
Це насправді є зміст теорем Нільпотентності та Періодичності Девіннаца,
Хопкінса та Сміта. У своїй основі "хроматична" ідея полягає в тому, що
ці сімейства, які ми отримали A, B і C, є початком стратифікації
гомотопічних груп сфер, і існує нескінченний список цих сімейств.
Таким чином, це різновид глобальної структурної теореми для цих розсипаних
та заплутаних груп. Ми ще не знаємо їх усіх, але знаємо, як вони
організовані. І враховуючи, наскільки вони складні, це справді велика гра!

Ще раз трохи про це все, що є ідеєю простих ідеалів, локалізації
та К-теорій Морави. Також тут буде трохи алгебраїчної геометрії.
Ви можете пам'ятати, що якщо ви хочете дізнатися про пучку на $Spec(\mathbb{Z})$,
достатньо знати про нього в кожному простому числі, тобто кожної "локалізації" 
цілих чисел $\mathbb{Z}$. Таким чином, хроматична стратифікація говорить про те,
що на відміну від алгебраїчних різноманітності, інформація про які може
бути зібрана з цілих простих чисел, інформація про комплекси CW може бути
зібрана з цих так званих хроматичних шарів. Насправді, ці хроматичні шари
не просто стратифікують гомотопічні групи сфер, вони стратифікують всю
категорію кінцевих CW-комплексів. Оскільки схема може бути розбита на
її p-локальні частини для кожного простого p, простір може бути розбитий
на його хроматичні частини. Це зміст Thick Subcategory Theorem, яка є
частиною теореми про Нільпотентність та Періодичність (наслідком).
У ній говориться, що, знаходячи цю стратифікацію лише з сфер, ми фактично
стратифікували всі кінцеві комплекси клітин. Тут під кінцевим клітинним
комплексом, я маю на увазі простір, побудований шляхом виконання кінцевого
числа приклеювань осередків (починаючи з порожнечі). Більше того, для тих
з вас, хто знає, що це означає, існують теорії когомологій, які говорять
про те, на якому СЛО є цей простір. І ці теорії когомологій є так званими
К-теоріями Морави $K(n)$ для натуральних n. Тому кінцевий клітинний
комплекс має "тип", який є натуральним числом, і це число говорить
мені, до якого хроматичного шару належить кінцевий клітинний комплекс.
Це може бути обчислено, тому що це просто перший n, для якого $K(n)$
когомології вашого простору відмінні від нуля.

$K(0)$ відповідає ступеню відображення p, $K(1)$ відповідає відображенню A, $K(2)$
відповідає відображенню B і т.д. Ці відповідності я не розглядатиму
тут детально. Але, в будь-якому випадку, з'являється така, дійсно
красива, картина, де ці $K(n)$ говорять нам про те, як розділити
категорію кінцевих клітинних комплексів на дрібніші, зручніші шари.
Фактично, подібно до того, як ви можете локалізувати схему або пучок
для простого p, ви можете локалізувати будь-який простір $K(n)$ для
будь-якого n. І якщо ви знаєте його $K(n)$ локалізацію для кожного n,
тоді ви можете зібрати цей простір по шматках, і ви будете знати все
про цей простір. Це невелика частина того, що люди мають на увазі,
коли кажуть, що теорія Морави --- це "прості числа" (стабільної) гомотопічної
категорії. По суті, вони --- "місця", в яких ми можемо локалізувати, щоб
отримати локальну інформацію, яку ми хочемо зібрати в глобальну інформацію.

Я повинен додати, що в основному все, що я знаю про це, я дізнався
з "помаранчевої книги", або "Нільпотентність та періодичність у Стабільній
теорії гомотопії" Дугласа Равенела. Це справді красива, приголомшлива книга,
і я всіляко її рекомендую.
\\
\\
\\
\textsc{\footnotesize Джонатан Бірдслей}
