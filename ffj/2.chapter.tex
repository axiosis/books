\section{Хопф}

Візьмемо людський мозок. Нехай кожен нейрон --- це вершина симпліціального
комплексу (тріангульованого простору), навіть не комплексу, а симпліціального
множини (теж що і комплекс, але з інформацією про орієнтацію, тут орієнтація ---
це категорна дуальність, перевертання n-стрілок), так як різниця потенціалів
передається по дендрону від нейрона до нейрона у певному напрямку. Беремо
радіоактивні ізотопи (найкраще взяти помічений ЛСД), пропускаємо через
енцефалічний бар'єр і будуємо симпліційний комплекс. Чому ми взяли симпліційні
множини, а не спрямовані графи, тому що групи нейронів утворюють згустки,
в яких енергія зв'язку настільки сильна, що можна говорити про компактність
клітин на n-рівнях. Якщо спробувати згенерувати рендомний мозок, з урахуванням
статистичних даних, ми отримаємо симпліційний комплекс розмірності 3 (три) загалом.
Якщо ж ми візьмемо очікування розмірності за реальними конкретними мізками,
ми отримаємо кількість вимірів комплексу рівним приблизно 8 (восьми). Такі
многовиди відомі як Калабі Яу, а простір в якому живуть всі фізичні симетрії
стандартної моделі міститься в групі Е8, яка в гомотопічній інтерпритації
розрізається на 4 розрашування Хопфа. Моделювання нейромереж багатовимірними
комплексами це маст хев сучасного теоретичного АI, як у симуляції (медичній),
так і в прикладному моделюванні. У комп'ютер віжині вже, до речі. Критерій
Сохацького: якщо комплекс нейромережі має розмірність менше 8, чекати на
самозароджене АI там безглуздо. У процесі навчання ми зможемо спостерігати
зміну комплексу у реальному часі та можливе навіть підвищення розмірностей.

Взагалі, теорія симпліціальних множин має багато ізоморфізмів: теорія інфініті
категорій (відразу кілька моделей, квазікатегорії, про них піде мова в наступних
постах), теорія струн, і т.д. Забезпечення відкритого нескінченного
глобулярного (n-розмірного) когерентного (аналог композиції на n-рівнях)
простору – чиста геометрія. В геометрію ми виходимо завжди, якщо щось узагальнюємо
на нескінченності. Наприклад у теоретичній інформатиці, а саме теорії типів ---
у нас є два розділи: теорія типів та їх поліноміальні функтори (звичайні індуктивні
типи) з одного боку, і, з іншого боку — гомотопічна теорія типів (де є глобулярні
рівності, завдяки викинутому ета-правилу Id типа) та їх вищих індуктивних
типів (або CW-комплексів, тому що будь-який CW-комплекс можна виразити через HIT і навпаки).
