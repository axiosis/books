\addtocontents{toc}{\protect\newpage}

\chapter{Система вищих мов}
\epigraph{Присвячується автору MLTT}{Перу Мартіну-Льофу}

П'ятий розділ описує розвиток концептуальної моделі системи доведення теорем як сукупності:
послідовності формальних мов програмування, кожна наступна з яких, складніша за попередню,
має свою операційну семантику, та наслідує усі властивості попередніх мов послідовності.

\section*{Вступне слово}
Батьком всіх сучасних теорій типів, зокрема HoTT, в яких працюють сучасні формальні
математики можна вважати Пера Мартіна-Льофа. В його теорії мова ділиться на типи,
кожен з яких визначається 4 правилами: конструктори та деконструктори,
та рівняння які визначають як відбуваються обчислення та гарантують унікальність.
Таке кодування природнім чином відповідає кодуванню ізоморфізмів та відображає
глибинну категорну семантику типових систем. Ця робота цілком відповідає
моделі типів Мартіна-Льофа та показує те тільки конфігурацію конкретного спектра мов,
але і визначає модель для опису цього спектру.

\section{Чиста система типів PTS}
IEEE\footnote{IEEE Std 1012-2016  --- V\&V Software verification and validation} стандарт
та регуляторні документи ESA\footnote{ESA PSS-05-10 1-1 1995 -- Guide to software verification and validation}
визначають інтрументи та підходи до виробничого процесу верифікації та валідації.
Найбільш розвинені та потужні засоби вимагають застосування математичних мов та нотацій.
Ера верифікованої математики була започаткована верифікатором AUTOMATH\cite{deBruijn83} (де Брейн) розробленого
під керівництвом де Брейна, а також розвиток теорії типів Мартіна-Льофа\cite{Lof84}.
Сьогодні ми маємо Lean, Coq, F*, Agda мови які використовують числення
конструкцій, Calculus of Constructions\cite{Coq88} (CoC)
та числення індуктивних типів (Calculus of Inductive Constructions\cite{Pfenning89} (CiC).
Пізніше учень де Брейна, Хенк Барендрехт класифікував послаблені чисті
системи типів по трьом осям та візуазізував це за допомогою лямбда-куба\cite{Henk93}.
Чисті мови програмування вже були імплементовані раніше
(Morte\footnote{Gabriel Gonzalez. Haskell Morte Library \url{https://github.com/Gabriel439/Haskell-Morte-Library}} Габріеля Гонзалеза, Henk\cite{Erik97} Еріка Мейера).
Чисті системи типів це системи з однм $\Pi$-типом (або ще і $\Sigma$ як в ECC\cite{Ore92}, Оре),
з можливими розширеннями, такими як PTS$^\infty$ зі зліченою кількістю всесвітів\cite{Tonpa18} (Сохацький),
Cedile з self-типами\cite{Fu14}\cite{Stump17} (Стамп, Фу), система з К-правилами\cite{Barthe95} (Барте).

Головна мотивація чистих систем --- це простота аналізу ядра верифікатора,
можливість застосування сильної нормалізації та довірена зовнішня верифікація
та сертифікація завдяки простоті верифікатора (type checker), це означає, що
алгоритм верифікації повинен бути настільки простим, аби можна було
просто імплементувати його на будь-якій мові програмування. Приклади застовування
тут можуть бути:
1) формальна мова блокчейн контрактів (Pluto
   \footnote{Rebecca Valentine. Formal Specification of the Plutus Core Language. 2017.
             \url{https://iohk.io/research/papers/#JT5XKNBP}});
2) сертифіковані обчислення для інтерпретаторів;
3) платіжні системи.

\subsection{Генерація сертифікованих програм}
Згідно ізоморфізму Каррі-Говарда-Ламбека або інтерпретації Брауера-Гейтінга-Колмогорова
існує взаємноознозначна відповідність між доведеннями теорем (або пруфтермами)
та лямбда функціями в теорії типів Мартіна-Льофа\cite{Lof84}.
Так як специфікація та доведення її відоповідності для певної програми
відбувається за допомогою мови з залежними типами, ми можемо екстрагувати
цільову імплементацію (зі стертою інформацією про типи) сертифікованої программи
в довільну мову програмування. У якості такої цільової мови підходять
майже усі інтерпретатор безтипового лямбда числення, такі як JavaScript,
Erlang, PyPy, LuaJIT, K.

Більш розвинені практики та підходи до кодогенерації та екстрагуванню
сертифікованих програм полягає у генерації С++ чи Rust програм, або програм
для нижчих систем лямюда-кубу, таких як System F або System F$_\omega$.
У цій роботі представлений екстракт в мову Erlang у якості цільового інтерпретатору.

\textbf{PTS синтаксиси}. Мінімальне ядро з однією аксіомою
сприймає декілька лямбда ситаксисів.
Перший синтаксис сумісний з системою програмування
$morte$\footnote{http://github.com/Gabriel439/Haskell-Morte-Library}, та походить від неї.
Інший синтаксис сумісний з синтаксисом $cubical$\footnote{http://github.com/mortberg/cubicaltt}.
Планувалося також підтримати синтаксис $caramel$\footnote{https://github.com/MaiaVictor/caramel}.

\begin{equation}
\begin{cases}
Sorts = U.\{i\},\ i : Nat\\
Axioms = U.\{i\} : U.\{inc\ i\}\\
Rules = U.\{i\} \leadsto U.\{j\} : U.\{max\ i\ j\}\\
\end{cases}
\end{equation}

Мова програмування Ом -- це мова з залежними типами, яка є розширенням
числення конструкцій (Calculus of Constructions, CoC) Тері Кокана. Саме з числення
конструкцій починається сучасна обчислювальна математика. В додаток до CoC,
наша мова Ом має предикативну ієрархію індексованих всесвітів. В цій мові немає
аксіоми рекурсії для безпосереднього визначення рекурсивних типів. Однак в цій мові
вцілому, рекурсивні дерева та корекурсія може бути визначена, або як кажуть, закодована.
Така система аксіом називається системою з однією аксіомою (або чистою системою), тому що в ній
існує тільки Пі-тип, а для кожного типу в теорії типів Мартіна Льофа існує п'ять
конструкцій: формація, інтро, елімінатор, бета та ета правила.

Усі терми підчиняються системі аксіом \textbf{Axioms} всередині
послідовності всесвітів \textbf{Sorts} та складність залежного
терму відповідає максимальній складності домена та кодомена
(правила \textbf{Rules}). Таким чином визначається простір всесвітів,
та його конфігурація може бути записана згідно нотації
Барендрехта для систем з чистими типами:

Проміжна мова чистої системи типів Ом базується на мові
Henk\cite{Erik97}, вперше описаній Еріком Мейером та Саймоном Пейтоном Джонсом в 1997 році.
Пізніше Габріель Гонзалез імплементував на мові Haskell
верифікатор з посиланням на Henk, та використував кодування Бома для нерекурсивного
кодування рекурсивних індуктивних типів. Ця мова базується лише на $\Pi$-типі,
$\lambda$-функції, її елімінатора аплікації, $\beta$-редукції та $\eta$-експансії.
Дизайн мови Ом нагадує дизайн мов Henk та Morte.
Ця мова призначена бути максимально простою (повна імплементація займає 300 рядків),
формально верифікованою, здатною продукувати сертифіковані програми та
розповсюджувати їх за межі комп'ютера по мережах та недовірених каналах зв'язку,
та компілювати (верифікувати та екстрагувати) на цільових платформах за допомогою
тієї ж мови Ом, можливо імплементовоної на іншій мові програмування та вбудованої
в основну систему.

\begin{table}[h]
\begin{center}
\caption{Перелік мов, інтерпретаторів та платформ для цільової компіляції}
\tabcolsep7pt\begin{tabular}{lcccc}
\hline
\rowcolor{ZimaBlue}
\textbf{Ціль} & \textbf{Джерело} & \textbf{Система типів} & \textbf{Застосування}\\
\hline
\rowcolor{LightGray}
CPS        & Per      & $\pi$+$\lambda$+$\mu$ calculi & середовище виконання \\
\rowcolor{LightGray}
BEAM       & Hamler     & System F            & системна бібліотека N2O.DEV \\
\rowcolor{LightGray}
BEAM       & PTS        & CoC$^\infty$        & фундаментальна мова \\
\rowcolor{LightGray}
BEAM       & HTS        & HoTT                & гомотопічна система \\
\hline
JavaScript & PureScript & System F & дитячі розваги \\
JVM        & Java       & F-sub\footnote{System F wit bounded quantification} & реліктова історія \\
JVM        & Scala      & System F$_\omega$   & маргінальна промисловість \\
CLR        & F\#        & System F$_\omega$   & маргінальна промисловість \\
GHC        & Haskell    & System F$_\omega$   & avoid success at all costs \\
GHC        & Morte      & CoC                 & фундаментальна мова Кокана \\
JavaScript & Kind       & Self-Types + IntNet & сучасна фундаментальна мова \\
GHC,OCaml  & Coq        & CiC                 & промисловий прувер \\
\hline
\end{tabular}
\end{center}
\end{table}

\subsection{Синтаксис}
Синтаксис PTS сумісний з численням конструкцій (CoC) Тері Кокана,
та такими мовами як Morte та Henk.
Однак в системі PTS присутній індекс для всесвітів який
представлений натуральними числами. Тут наведений синтаксис у BNF нотації

\begin{lstlisting}[mathescape=true]
     I := #list #nat
     U := * + * . #nat
     O := U + I + ( O ) + O O + O $\rightarrow$ O
        + $\lambda$ ( I : O ) $\rightarrow$ O
        + $\forall$ ( I : O ) $\rightarrow$ O
\end{lstlisting}

Тут + --- сумма виразів, '.' --- конкатенація терміналів без пробілу,
:= --- оператор визначення BNF-правила, \#empty, \#nat, \#list --- вбудовані типи BNF-нотації
--- синтаксичні елементи BNF нотації,
а *,:, $\rightarrow$, (, ), $\lambda$, $\forall$ -- термінали або синтаксичні елементи мови програмування.
Еквівалентне визначення як ініціальний об'єкт категорій $O_{PTS}$ або $O_\Pi$
який може вмістити цей синтаксис містить всі правила виводу
внутрішньої мови категорії.

\begin{lstlisting}[mathescape=true]
def pts (lang: U) : U
 := inductive { star             (n: nat)
              | var    (x: name) (l: nat)
              | pi     (x: name) (l: nat) (d c: lang)
              | lambda (x: name) (l: nat) (d c: lang)
              | app                       (f a: lang)
              }
\end{lstlisting}

\subsection*{Всесвіти}
Мова PTS$^\infty$ -- це лямбда числення з залежними типами вищого порядку,
розширення числення конструкцій Кокана, або системи P$_\omega$ Барендрехта,
з предикативною (імпредикативною) ієрархією індексованих всесвітів.
Це розширення мотивоване консистентністю\cite{Lof75} в залежній теорії типів та
неможливістю кодування парадоксів Жирара-Хуркенса-Рассела\footnote{Так званий парадокс голяра який виникає в системах $U : U$}. Також для
забезпечення консистентності в мові PTS відсутня аксіома \lstinline{Fixpoint}.

$$
    \mathrm{U_0} : \mathrm{U}_1 : \mathrm{U}_2 : \mathrm{U}_3 : ...
$$

Де $\mathrm{U_0}$ --- імпредикативний всесвіт,
   $\mathrm{U_1}$ --- перший предикативний всесвіт,
   $\mathrm{U_2}$ --- другий предикативний всесвіт,
   $\mathrm{U_3}$ --- третій предикативний всесвіт і т.д.

\begin{equation}
\tag{S}
\dfrac
{o : Nat}
{U_o}
\end{equation}

\subsubsection*{Предикативні всесвіти}
Всі терми підпорядковуються системі аксіом А для послідовності всесвітів S.
Складність R залежності термів дорівнює максимальнії складної термів з
яких складаєтсья формула (або вираз мови). Система всесвітів описується
згідно SAR-нотації Барендрехта. Зауважте, що предикативні всесвіти
несумісні в Бом кодуванням, але ви можете переключати предикативність.

\[
\tag{$A_1$}
\dfrac{i: Nat, j: Nat, i < j}{U_i : U_j}
\]

\[
\tag{$R_1$}
\dfrac{i : Nat, j : Nat}{U_i \rightarrow U_j : U_{max(i,j)} }
\]

\subsubsection*{Імпредикативні всесвіти}
Стягуваний імпредикативний простір внизу ієрархії є єдиним можливим розширенням
предикативної ієрархії для того аби вона залишалась консистентною. Однак
в чистій системі типів PTS підтримується ієрархія бескінечних імпредикативних всесвітів.

\begin{equation}
\tag{$A_2$}
\dfrac
{i: Nat}
{U_i : U_{i+1}}
\end{equation}

\begin{equation}
\tag{$R_2$}
\dfrac
{i : Nat,\ \ \ \ j : Nat}
{U_i \rightarrow U_{j} : U_{j}}
\end{equation}

\subsection*{Контексти}

Контексти моделюються словником з іменами змінних в верифікаторі.
Він може бути типизований як \lstinline{list Sigma}.
Правило елімінації тут не дається, після використання функції верифікації,
словник вивільняється з пам'яті.

\begin{equation}
\tag{Ctx-formation}
\dfrac
{}
{\Gamma : Ctx}
\end{equation}

\begin{equation}
\tag{Ctx-intro$_1$}
\dfrac
{\Gamma : Ctx}
{\emptyset : \Gamma}
\end{equation}

\begin{equation}
\tag{Ctx-intro$_2$}
\dfrac
{A : U_i,\ \ \ \ x : A,\ \ \ \ \Gamma : Ctx}
{(x : A)\ \vdash\ \Gamma : Ctx}
\end{equation}

\subsection{Операційна семантика}
Операційна семантика --- це правила обчислення,
або $\beta$-,$\eta$-правила фьюжену інтро-правила та елімінаторів.
для визначення яких необхідно визначити:
1) інтро-правила, їх тип (правило формації), та класс (тип правила формації);
2) правило елімінації та залежної елімінації (індукції).
Таким чином будемо вважати, що операційна семантика системи типів $O_{PTS}$
буде складатися з 5 правил: формації, інтро-правило,
залежний елімінатор (індукція), $\beta$-редукція або правило обчислення,
$\eta$-експансія або правило унікальності.

\begin{equation}
\tag{$\Pi$-formation}
\dfrac
{A:U_i\ ,\ x:A \vdash B : U_j}
{\Pi\ (x:A) \rightarrow B : U_{p(i,j)}}
\end{equation}

\begin{equation}
\tag{$\lambda$-intro}
\dfrac
{x:A \vdash b : B}
{\lambda\ (x:A) \rightarrow b : \Pi\ (x: A) \rightarrow B }
\end{equation}

\begin{equation}
\tag{$App$-elimination}
\dfrac
{f: (\Pi\ (x:A) \rightarrow B)\ \ \ a: A}
{f\ a : B\ [a/x]}
\end{equation}

\begin{equation}
\tag{$\beta$-computation}
\dfrac
{x:A \vdash b: B\ \ \ a:A}
{(\lambda\ (x:A) \rightarrow b)\ a = b\ [a/x] : B\ [a/x]}
\end{equation}

\begin{equation}
\tag{subst}
\dfrac
{\pi_1 : A\ \ \ \ u:A \vdash \pi_2 : B}
{[\pi_1/u]\ \pi_2 : B}
\end{equation}

Перелік теорем (специфікації) для чистої системи типів можуть бути
прямо вбудовані в теорію типів, таким чином ми отримуємо логічний фреймворк
для перевірки імплементації залежної теорії.

Доведення цих теорем дано в модулі базової бібліотеки розділу 3.
Також можна повитися на інші доведення \cite{Henk93}.
Рівняння обсислювальної семантики (бета та ета правила) визначаються
за допомогою Path-типів, які визначаються $O_=$ або $O_I$ мовним синтаксисом.

Ці рівняння обчислювальної семантики представлені тут як Path-тип в вищій мові.
В чисту систему типів PTS (з бескінечною кількістю всесвітів)
для завантаження файлів з локального довіреного сховища додається
remote конструктор в синтаксичне дерево. Рекурсія по цьому конструктору заборонена.

Індекси де Брейна діють локально в межах одного імені.
При додаванні існуючого імені в контекст збільшується індекс цього імені.
Таким чином PTS верифікатор чистої системи типів відрізняється від
канонічного приклада алгоритма верифікації CoC\cite{Coq88}. Він включає
наступні функції мовної категорії: підстановка,
зсув імені, нормалізація термів, рівність
за визначенням та верифікація.

\subsection*{Перевірка типів}
Для перевірки типів застосовується наступний алгоритм верифікації, який є основаю
усіх залежних систем. В чистих системах потрібно бути обережним з \lstinline{remote}
конструктором. Він використовуються для завантаження типів з локального довіреного сховища.
При дозволі рекурсії по \lstinline{remote} конструктору можливо реалізувати
self-типи\cite{Stump17}\cite{Fu14}.

\begin{lstlisting}[mathescape=true]
type (:star,N)     D $\rightarrow$ (:star,N+1)
     (:var,N,I)    D $\rightarrow$ :true = proplists:is_defined N B, om:keyget N D I
     (:pi,N,0,I,O) D $\rightarrow$ (:star,h(star(type I D)),star(type O [(N,norm I)|D]))
     (:fn,N,0,I,O) D $\rightarrow$ let star (type I D), NI = norm I
                         in (:pi,N,0,NI,type(O,[(N,NI)|D]))
     (:app,F,A)    D $\rightarrow$ let T = type(F,D),
                            (:pi,N,0,I,O) = T, :true = eq I (type A D)
                         in norm (subst O N A)
\end{lstlisting}

\subsection*{Індекси де Брейна}
Зсув переіменовує змінну N в контексті P, тобто додає одиницю для лічильника цієї змінної.

\begin{lstlisting}[mathescape=true]
  sh (:star,X)     N P $\rightarrow$ (:star,X)
     (:var,N,I)    N P $\rightarrow$ (:var,N,I+1) when I >= P
                       $\rightarrow$ (:var,N,I)
     (:pi,N,0,I,O) N P $\rightarrow$ (:pi,N,0,sh I N P,sh O N P+1)
     (:fn,N,0,I,O) N P $\rightarrow$ (:fn,N,0,sh I N P,sh O N P+1)
     (:app,L,R)    N P $\rightarrow$ (:app,L,R)
\end{lstlisting}

\newpage
\subsection*{Підстановка, нормалізація, рівність}
Підстановка заміняє змінну у виразі на певний терм.

\begin{lstlisting}[mathescape=true]
 sub (:star,X)     N V L $\rightarrow$ (:star,X)
     (:var,N,L)    N V L $\rightarrow$ V
     (:var,N,I)    N V L $\rightarrow$ (:var,N,I-1) when I > L
     (:pi,N,0,I,O) N V L $\rightarrow$ (:pi,N,0,sub I N V L,sub O N (sh V N 0) L+1)
     (:pi,F,X,I,O) N V L $\rightarrow$ (:pi,F,X,sub I N V L,sub O N (sh V F 0) L)
     (:fn,N,0,I,O) N V L $\rightarrow$ (:fn,N,0,sub I N V L,sub O N (sh V N 0) L+1)
     (:fn,F,X,I,O) N V L $\rightarrow$ (:fn,F,X,sub I N V L,sub O N (sh V F 0) L)
     (:app,F,A)    N V L $\rightarrow$ (:app,   sub F N V L,sub A N V L)
\end{lstlisting}

Нормалізація виконує підстановку при аплікаціях до функцій (бета-редукція)
за допомогою рекурсивного спуску по конструкторам синтаксичного дерева.

\begin{lstlisting}[mathescape=true]
norm (:star,X)     $\rightarrow$ (:star,X)
     (:var,X)      $\rightarrow$ (:var,X)
     (:pi,N,0,I,O) $\rightarrow$ (:pi,N,0,norm I,norm O)
     (:fn,N,0,I,O) $\rightarrow$ (:fn,N,0,norm I,norm O)
     (:app,F,A)    $\rightarrow$ case norm F of
                         (:fn,N,0,I,O) $\rightarrow$ norm (subst O N A)
                                    NF $\rightarrow$ (:app,NF,norm A) end
\end{lstlisting}

Рівність за визначенням перевіряє рівність Erlang термів.

\begin{lstlisting}[mathescape=true]
  eq (:star,N)        (:star,N)        $\rightarrow$ true
     (:var,N,I)       (:var,(N,I))     $\rightarrow$ true
     (:pi,N1,0,I1,O1) (:pi,N2,0,I2,O2) $\rightarrow$
          let :true = eq I1 I2
           in eq O1 (subst (shift O2 N1 0) N2 (:var,N1,0) 0)
     (:fn,N1,0,I1,O1) (:fn,N2,0,I2,O2) $\rightarrow$
          let :true = eq I1 I2
           in eq O1 (subst (shift O2 N1 0) N2 (:var,N1,0) 0)
     (:app,F1,A1)       (:app,F2,A2)   $\rightarrow$ let :true = eq F1 F2 in eq A1 A2
     (A,B)                             $\rightarrow$ (:error,(:eq,A,B))
\end{lstlisting}

\subsection*{Обмеження}
Обмеження:
1) неможливість визначити рекурсію та індукцію без fixpoint аксіоми;
2) кодування Бома повинно бути позитивно-рекурсивним;
3) неможливість побудови великого елімінатора, вивести тип з даних;
4) неефективність сім'ї лямбда кодувань вцілому (Парігот, Скотт, Бом).

\newpage
\subsection{Використання мови}
Тут буде показано використання мови PTS.

\begin{lstlisting}
> ./om help me
[{a,[expr],"to parse. Returns {_,_} or {error,_}."},
 {type,[term],"typechecks and returns type."},
 {erase,[term],"to untyped term. Returns {_,_}."},
 {norm,[term],"normalize term. Returns term's normal form."},
 {file,[name],"load file as binary."},
 {str,[binary],"lexical tokenizer."},
 {parse,[tokens],"parse given tokens into {_,_} term."},
 {fst,[{x,y}],"returns first element of a pair."},
 {snd,[{x,y}],"returns second element of a pair."},
 {debug,[bool],"enable/disable debug output."},
 {mode,[name],"select metaverse folder."},
 {modes,[],"list all metaverses."}]

> ./om print fst erase norm a "#List/Cons"
   \ Head
-> \ Tail
-> \ Cons
-> \ Nil
-> Cons Head (Tail Cons Nil)
ok
\end{lstlisting}


\subsection*{Екстракти}
Мова $PTS_\infty$ передбачає автоматичну генерацію сертифікованих програм в цільові платформи.
Сертифікая полягаю у візуальному доведенню одніє стрілки ізоморфізма
$\lambda$-функції в залежній теорії типів та $\lambda$-фунції в нетипизованому лямбда численні.

\begin{lstlisting}[mathescape=true]
ext (:var,X,N,F)      $\rightarrow$ (:var,X)
    (:app,A,B,N,F)    $\rightarrow$ (:call,N,ext(F,A,N),[ext(F,B,N)])
    (:fn,S,_,I,O,N,F) $\rightarrow$ (:fun,N,(:clauses,[{:clause,N,
                                [(:var,N,S)],[],[ext(F,O,N)]}]))
                    _ $\rightarrow$ []
\end{lstlisting}

Так працює функція екстракту в Erlang з системи типів PTS$^\infty$.
Erlang-версія Ом повинна бути зручна для використання для
віртуальних машин LING та BEAM. Оскільки цей екстракт генерує
AST дерево Erlang (подідбно до Elixir), результуючий код
подається повністью на весь стек оптимізаційного компілятора
Erlang. включаючи Erlang Core, тому весь модуль екстракта займає 30 рядків.

\newpage
\subsubsection{Інтерпретатори}
З практичної точки зору мова Ом є способом використання залежних типів
та специфікації, побудовані за їх допомоги на мові Erlang.
Завдяки глибокій інтеграції з Erlang вдалося мімізувати імплементацію системи до 300 рядків.
Екстракт в інтерпретатор $O_{PTS}$ (чи інші) є альтернативною опцією для Ом.
Також, мова Ом може бути легко портована на інші мови.

\subsubsection{LLVM}
Більш складна опція генерації сертифікованих програм --- це генерація машинного коду,
з використанням або без використання допоміжних проміжних мов таких як LLVM та MIR.
Тому що для цього потрібно верифікувати модель асемблера та процесора а також
його оптимізатора, так як зі складністю синтаксичного дерева росте складність
та велична терму-доведення будь-яких властивостей.

\subsubsection{FPGA}
Інша, не менш складна, або ще більш складна опція є безпосередня генерація
VHDL моделей (наприклад, clash).

\newpage
\section{Система індуктивних схем Paulin-Mohring}
Індуктивні синтаксиси та кодування можуть підтримуватися за допомогою системи модулів.
Кожна система модулів може самостійно (у вигляді ефектів), або за допомогою лямбда кодувань
попередньої мови PTS рівня, зберігати та оперувати індуктивними типами даних.

\subsection{Синтаксис}
\begin{lstlisting}[mathescape=true]
def := data id tele = sum + id tele : exp = exp +
       id tele : exp where def
exp := cotele*exp + cotele $\rightarrow$ exp + exp $\rightarrow$ exp + (exp) + app + id +
       (exp,exp) + \ cotele $\rightarrow$ exp + split cobrs + exp .1 + exp .2

  0 := #empty         imp    := [ import id ]
brs := 0 + cobrs      tele   := 0 + cotele
app := exp exp        cotele := ( exp : exp ) tele
 id := [ #nat ]       sum    := 0 + id tele + id tele | sum
ids := [ id ]         br     := ids $\rightarrow$ exp
cod := def dec        mod    := module id where imp def
dec := 0 + codec      cobrs  := | br brs
\end{lstlisting}

Індуктивні синтаксиси будуються на телескопах Диб'єра,
конструкторах сум, та їх елімінаторах.

\newpage
\subsection{Поліноміальні функтори}
Існує два види формальної рекурсії: 1) перша з найменшою нерухомою точкою
(як $F_A(X) = 1 + A \times X$ або $F_A(X) = A + X \times X$), іншими словами
рекурсія з базою (термінується $1$ або $A$). Списки та дерева є
прикладами таких рекурсивних структур з nil та leaf термінальними
конструкторами (або рекурсивні суми).
2) друга з найбільшою нерухомою точкою, або рекурсія без бази
(як $F_A(X) = A \times X$) --- така рекурсія не термінована на рівні типів,
та моделює нетерміновані послідовності, процеси тощо (або рекурсивні добутки).
Кодування найменшою нерухомою точкою ще називається кодуванням
добре-визначиними деревами або кодування поліноміальними функторами.
\\
\\
\noindent Натуральні числа: $\mu\ X \rightarrow 1 + X$\\
Списки елементів A: $\mu\ X \rightarrow 1 + A \times X$\\
Лямбда числення: $\mu\ X \rightarrow 1 + X \times X + X$\\
Потоки: $\nu\ X \rightarrow A \times X$\\
Потенційно нескінченний список елементів A: $\nu\ X \rightarrow 1 + A \times X$\\
Скінченне дерево: $\mu\ X \rightarrow \mu\ Y \rightarrow 1 + X \times Y = \mu\ X = List\ X$\\

Для цих кодувань існує аналог кодування Чорча, який розповсюджує
кодування чистими функціями з нетипизованого лямбда числення до $\Pi$-типу.
Таке кодування називається кодуванням Бома-Беррардуччі, а просто кодування Бома.
Воно дозволяє кодувати індуктивні типи даних $\Pi$-типами чистими функціями.
Проте як було показано Жеверсом\cite{Geuvers01} неможливо побудувати принцип
індукції в чистих системх без використання в явному чи прихованому
вигляді \lstinline{Fixpoint} аксіоми. Також неможливо побудувати
J елімінатор Id типу закодованого в Бом кодуванні, а також
елімінатори гомотопічних примітивів, наприклад елімінатори гомотопічного
візрізка як \lstinline{funExt}, \lstinline{homotopy}.

\newpage
\subsection{Кодування Бома}
Тип даних List над даним типом А, може бути представлений як ініціальні алгебра
$(\mu L_A, in)$ функтору $L_A(X) = 1 + (A \times X)$. Позначається $\mu L_A = List(A)$.
Функції-конструктори $nil: 1 \rightarrow List(A)$ та
$cons: A \times List(A) \rightarrow List(A)$ визначені як
$nil = in \circ inl$ та $cons = in \circ inr$, таким чином $in = [nil,cons]$.
Для кожних двох функцій $c: 1 \rightarrow C$ та $h: A \times C \rightarrow C$,
катаморфізм $f = \llparenthesis [c,h] \rrparenthesis : List(A) \rightarrow C$
є унікальним розв'язком системи рівнянь:

$$
\begin{cases}
  f \circ nil  = c \\
  f \circ cons = h \circ (id \times f)
\end{cases}
$$

де $f = foldr(c,h)$. Маючи це, ініціальна алгебра представлена функтором
$\mu (1 + A \times X)$ та сумою морфізмів
$[1 \rightarrow List(A), A \times List(A) \rightarrow List(A)]$
як катаморфізму. Використовуючи це кодування, List-тип в базовій бібліотеці мови $O_{PTS}$
буде мати наступну форму:

$$
\begin{cases}
 foldr = \llparenthesis [ f \circ nil , h] \rrparenthesis, f \circ cons = h \circ (id \times f)\\
 len = \llparenthesis [ zero, \lambda\ a\ n \rightarrow succ\ n ] \rrparenthesis \\
 (++) = \lambda\ xs\ ys \rightarrow \llparenthesis [ \lambda (x) \rightarrow ys, cons ] \rrparenthesis (xs) \\
 map = \lambda\ f \rightarrow \llparenthesis [ nil, cons \circ (f \times id)] \rrparenthesis
\end{cases}
$$

\begin{lstlisting}[mathescape=true]
def list (A: U) : U
 := inductive { cons (x: A) (cs: list A)
              | nil
              }
\end{lstlisting}

\begin{cases}
list = \lambda\ ctor \rightarrow \lambda\ cons \rightarrow \lambda\ nil \rightarrow ctor\\
cons = \lambda\ x\ \rightarrow \lambda\ xs \rightarrow \lambda\ list \rightarrow \lambda\ cons \rightarrow\ \lambda\ nil \rightarrow cons\ x\ (xs\ list\ cons\ nil)\\
nil = \lambda\ list \rightarrow \lambda\ cons \rightarrow \lambda\ nil \rightarrow nil\\
\end{cases}

\begin{lstlisting}[mathescape=true]
axiom map (A B: U) (f: A -> B) : list A -> list B
axiom length (A: U): list A -> nat
axiom append (A: U): list A -> list A -> list A
axiom foldl (A B: U) (f: B -> A -> B) (Z: B): list A -> B
axiom filter (A: U) (p: A -> bool) : list A -> list A
\end{lstlisting}

\begin{cases}
len = foldr\ (\lambda\ x\ n \rightarrow succ\ n)\ 0\\
(++) = \lambda\ ys \rightarrow foldr\ cons\ ys\\
map = \lambda\ f \rightarrow foldr\ (\lambda x\ xs \rightarrow cons\ (f\ x)\ xs)\ nil\\
filter = \lambda\ p \rightarrow foldr\ (\lambda x\ xs \rightarrow if\ p\ x\ then\ cons\ x\ xs\ else\ xs)\ nil\\
foldl = \lambda\ f\ v\ xs = foldr\ (\lambda\ xg\rightarrow\ (\lambda \rightarrow g\ (f\ a\ x)))\ id\ xs\ v\\
\end{cases}

\newpage
\subsection{Операційна семантика}
Традиційно індуктивні типи входять в операційну семантику систем побудованих
та основі MLTT-80 (таких як кубічні системи, або система HTS).
Оригінально Мартін-Льоф виразив індуктивні дерева через $W$-типи, для яких
також потрібно $0$-тип, $1$-тип, та $2$-тип. Така система є менш потужною ніж система типів Полін-Морін,
оскільки не дозволяє виразити загальні схеми індукції та взаємну рекурсію. Хоча з іншого
боку не потребує термінейшин чекера, стріктлі позітів чекера, та взаємної рекурсивності,
що дозволяє доводити семантику такого мовного ядра значно простіше.

\begin{equation}
\tag{$W$-formation}
\dfrac
{A:Type\ \ \ \ x:A\ \ \ \ B(x):Type}
{W (x:A) \rightarrow B(x) : Type}
\end{equation}

\begin{equation}
\tag{$W$-intro}
\dfrac
{a:A\ \ \ \ t: B(a) \rightarrow W}
{sup(a,t) : W}
\end{equation}

\begin{equation}
  \begin{equation}
  \tag{$W$-elim}
  \dfrac
    {\begin{split}
     & w: W \vdash C(w) : Type \\
     & x:A,\ u:B(x) \rightarrow W, \\
         & { \begin{split}
              & v:\Pi (y:B(x)) \rightarrow C(u(y)) \vdash c(x,u,v):C(sup(x,u))
           \end{split}
         }
     \end{split}
    }
    {w:W \vdash wrec(w,c):C(w)}
\end{equation}


\begin{equation}
   \begin{equation}
   \tag{$W$-$\beta$}
      \dfrac{
         \begin{split}
           & w: W \vdash C(w) : Type \\
           & x: A,\ u:B(x) \rightarrow W \\
           & { \begin{split}
               & v: \Pi (y:B(x)) \rightarrow C(u(y)) \vdash c(x,u,v):C(sup(x,u))
               \end{split}
          }
     \end{split}}
    {\begin{split}
    & x:A,\ u:B(x) \rightarrow W \vdash wrec(sup(x,u),c) \\
    & =\ c(x,u,\lambda (y:B(x)), wrec(u(y),c)) : C(sup(x,u))
    \end{split}
  }
\end{equation}

\newpage
\section{Гомотопічна система типів HTS}
Гомотопічна система типів, принаймі без вищих індуктивних типів, які є функтором на інших мовних категорій,
формально моделюється досніповою (предпучковою) семантикою на категорії де Моргана $I: \square_n^{op} \rightarrow Set$.
Так як тут теж є своя аплікація та лямбда, то можна сказати, що це ще
одна лямбда система, але для безпосередньох маніпуляції багатовимірними кубами,
використовуючи при цьому логіку де Моргана.

\subsection{Синтаксис}
Тут подано повний синтаксис разом з вищими індуктивними типами.

\begin{lstlisting}[mathescape=true]
   sys := [ sides ]          side := (id=0)$\rightarrow$exp+(id=1)$\rightarrow$exp
  form := form\/f1+f1+f2    sides := #empty+cos+side
   cos := side,side+side,cos  mod := $\textbf{module}$ id $\textbf{where}$ imps dec
    f1 := f1/\f2               f2 := -f2+id+0+1
   imp := $\textbf{import}$ id                 brs := #empty+cobrs
   app := exp exp             tel := #empty+cotel
  imps := #list imp         cotel := (exp:exp) tel
    id := #list #nat          dec := #empty+codec
    u2 := $\textbf{glue}$+$\textbf{unglue}$+$\textbf{Glue}$                   u1 := $\textbf{fill}$+$\textbf{comp}$
   ids := #list id             br := ids$\rightarrow$exp+ids@ids$\rightarrow$exp
 codec := def dec
 cobrs := | br brs
   sum := #empty+id tel+id tel|sum+id tel<ids>sys
   def := $\textbf{data}$ id tel=sum+id tel:exp=exp+id tel:exp $\textbf{where}$ def
   exp := cotel*exp+cotel$\rightarrow$exp+exp$\rightarrow$exp+(exp)+id
          (exp,exp)+\cotele$\rightarrow$exp+$\textbf{split}$ cobrs+exp$\textbf{.1}$+exp$\textbf{.2}$+
          $\langle$ids$\rangle$exp+exp@form+app+u2 exp exp sys+u1 exp sys
\end{lstlisting}

Тут термінали := (визначенния), + (сума типів), \#empty (пустий тип), \#nat (тип натуральних чисел),
\#list (тип списків) --- є частинами BNF мови. Термінали
$\rvert$, :, *, $\langle$, $\rangle$, (, ), =, $\backslash$, /, -, $\rightarrow$, 0, 1, @, [, ],
$\mathbf{module}$, $\mathbf{import}$,
$\mathbf{data}$, $\mathbf{split}$, $\mathbf{where}$, $\mathbf{comp}$, $\mathbf{fill}$,
$\mathbf{Glue}$, $\mathbf{glue}$, $\mathbf{unglue}$,
$\mathbf{.1}$, $\mathbf{.2}$,
а також термінал $,$ є терміналами мови верифікатора гомотопічної системи типів.
Ця мова включає в себе: індуктивні типи, вищі індуктивні типи, оператори склеювання
для всесвітів та типів з відповідними елімінаторами. Усі ці концепції, та їх
моделі більш формально та детально описані у наступному розділі 3.

Система не повинна бути обмежена мовами та синтаксисами, ми покажемо як приклад,
підтримку гомотопічної мови з інтервалом [0,1] сумісної з $cubical$ та з пітримкою індуктивних
синтаксисів та кодувань попереднього рівня.

\newpage
\section{Структура верифікатора}

На відміну від одноаксіоматичного верифікатора $\mathbf{Henk}$,
який містить тільки один індексований всесвіт $U_i$, рівність
за визначенням для примітивів єдиного $\Pi$-типу, та функцію
верифікації $\tau = \mathbf{type}$, верифікатори $\mathbf{Per}$
і $\mathbf{Anders}$ містять додатково $\Sigma$-тип для контекстів
та телескопів, більш деталізовану функцію типізації $\tau$,
та багато інших досніпових модулів, крім $\Pi$-типу, але які
теж підпорядковуються системі типів Мартіна-Льофа.

\subsection*{Космос $\mathbb{N}$-індексованих всесвітів $\omega$ }

В теорії типів всі сигнатури всіх типів живуть в ієрархіях
всесвітів індексованих натуральними числами. Множина таких
ієрархій називається космосом. В імплементаціях $\mathbb{N}$
завжди реалізовано як Big Integer. Верифікатор $\mathbf{Anders}$ має
космос, що складається з двох ієрархій всесвітів $\omega = \{ \mathbf{V}_i, \mathbf{U}_i \}$.

\subsection*{Рівність \defequ\ з точністю до $\alpha$-$\beta$ конверсій}

Рівність за визначенням (інтенсіональна) двох термів означає, що за допомогою серії альфа та бета
перетворень можна довести що терми дорівнюють посимвольно (бінарно). Саме ця функція повинна бути
імплементована для всіх типів у верифікаторі. Програми, які доводять рівність двох термів
в теорії самого верифікатора за допомогою конструкторів рефлексивності називаються екстенсіональними рівностями.
Якшо це відбувається для $=$-типу у всесвітах $V_i$ --- це називаються пропрозиційними рівностями,
а якшо в інших ідентифікаційних системах у всесвітах $U_i$ -- називаються типовими рівностями.
Інтерналізація інтенсіональної рівності за визначенням всередині теорії за
допомогою $\Pi$-типів називається рівністю Лейбніца і є виводимою у всіх фібраційних верифікаторах.

\subsection*{Функція верифікації $\tau$}

Головна функція верифікації  розпадається на систему взаємозалежних функцій
$\tau = \{ \mathbf{infer},\mathbf{app}, \mathbf{check}, \mathbf{act}, \mathbf{conv}, \mathbf{eval} \}$,
які повинні бути імплементовані для кожного типу, вбудованого в верифікатор.


\newpage
\subsection*{Контексти та телескопи \sigmatype }

В теорії типів контексти, як алгебраїчні послідовності які містять сигнатури, які теж у свою
чергу складаються з послідовністей пар, що складаються з імені змінної та її типу, визначаються
$\Sigma$-типами.

\subsection*{Досніпові модулі \presheaf\ вбудованих типів}

Кожен досніповий модуль повинен бути представлений у вигляді п'яти
синтаксичних примітивів: 1) формації; 2) конструкції; 3) елімінації;
4) обчислювальності; 5) унікальності. Ці примітиви повинні бути узгоджені
в сенсі Мартіна-Льофа та представлені у цій статті,
як документація на бібліотеку верифікатора, як у тому числі дає формальне визначення
примітивам в конкретній теорії $\mathrm{\int} = \{ \Pi, \Sigma, =, \mathbf{W},
\mathbf{0}, \mathbf{1}, \mathbf{2}, \mathbf{Path}, \mathbf{Glue} \}$.

\section{Висновки}
Як апогей, система HTS є фінальною категорією,
куди сходяться всі стрілки категорії мов. Кожна мова та її категорія
мають певний набір стрілок ендоморфізмів, які обчислюють, верифікують,
нормалізують, оптимізують програми своїх мов.
Стрілки виду $e_i: O_{n+1} \rightarrow O_n$ є екстракторами, які понижають систему типів,
при чому $O_{CPS} = O_0$.

Базова бібліотека мови Ерланг у яку проводиться основний
естракт, поставляється з дистрибутивом Erlang/OTP. Базова бібліотека
$O_{PTS}$ наведена в репозиторії Github\footnote{\url{https://github.com/groupoid/henk}}.
Гомотопічна базова бібліотека відповідає термінальній мові $O_{CCHM}$, та теж відкрита
на Github\footnote{\url{https://github.com/groupoid/anders}}.
Останні два розділи присвячені математичному моделюванню математики на цій мові.

