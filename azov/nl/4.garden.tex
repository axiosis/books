\section{Сад націй}

\subsection{Концепція просвітленого царства}

Кожен з нас, як розумна істота знаходиться на свому рівні усвідомлення реальності.
Незалежно від цих наших моделей реальності ми як істоти прагнемо жити в
достатку та щасті щодня, це властивість кожної живої, навіть не мислячої, істоти.
Бажання самі виникають в нас і ведуть наше єство до їх втілення, ми транслюємо
свою волю, доки вона не стикається з волями інших істот і це не впливає на
загальну максиму філософії --- жити щасливо і вдостатку не тільки індивідуально,
а і в планетарному масштабі, курсуючи на величезному космічному кораблі під назвою
планета Земля, що обертається навколо космічного ядерного реактору, зірки Сонце,
що падає стрімголів по спіралі в чорну діру нашої галактики, Чумацького Шляху.
\\
\\
Еволюція сформувала наше суспільство через війни народів, держав та націй, які
до цих пір воюють та мирно співіснують між собою. Бажання розвинених націй жити
щасливо, які віднайшли етнологічну модель взаємодії у вигляді міжнародних органів
та організацій, сформувало мову політики, що визначає націю, як народ з політичними
правами, де одне з базових прав це право на життя. Ми здійснюємо та включаємо у
всесвітній парламент націй усі етноси, раси та культури. Однак наші ресурси обмежені
і отрути нашого мислення: відраза, бажання та байдужість --- призводять до воєн.
\\
\\
Питання етичної війни є ключовим для виживання людського виду, перший аналіз
цього питання в західній культурі прийнято приписувати блаженному Августину,
який розмірковув про теорію справедливої війни. Можно довго теоретизувати,
але емпірично планета зазнала двох пасток.
\\
\\
Перша пастка --- це пацифізм (або запрошення ворога до свого знищення),
який продемонстрував Тибет. Хоча юридично немає жодного підтвердження,
що Тибет це вільна країна, скоріше навпаки, це не означає, що Тибет як
жертва геноциду та вигнання в 1958 році повинен бути забутий та засимельований Китаєм.
Український народ, який перебував під постійним тероризмом та знищенням державності з
боку північних диких племен які постійно мімікрували на політичній арені націй, добре
знає, що геноцид 1933 року не можна пояснювати поганою поведінкою українців, які
перебували під окупацією. Помилка, яку допустив Понтифік 2022, під назвою пацифізм
або запрошення ворога до свого знищення демонструє інфантильність західної релігійної
політики та її знаходженні на рівні Чемберлена 1939.
\\
\\
Друга пастка --- це надмірна жорстокість та бажання до повного знищення ворога.
Хоча ми як суспільство вижили, користуючись цією максимою, це не означає що вона
повинна бути покладена в етику воєн майбутнього. Сучасні війни користуються етичними
поведінковими нормами знешкодження ворога та його розброєння з проявленням гуманності
та без тотальних руйнувань. Ми бачимо на власні очі що надмірна жорстокість та бажання
винищення нашої нації вертається на той рівень який ми вже називаємо геноцидом. Ідея
виправдання війни не повинна бути її причиною, війна для захищення своєї нації, яка
ведеться під офіційним прапором, з юридичним забезпеченням парламенту націй є
праведною війною. Зараз ми заглибилися в ще одну нору-відгалуження другої пастки,
ми думали що організація-модератор, світовий тритейський орган вже побудований,
але, на жаль, його вже доведеться будувати нашим дітям.
\\
\\
Хоча тибетський геноцид був здійснений як запрошення ворога до свого знищення,
в тибетскій філософії та деяких традиціях є поняття націєтворення та просвітленного
царства, священної мандали з головним центром та офісом Гуру Рінпоче в столиці на
Славетній Мідноцвітній Горі та її представництвами на нашій планеті Земля, та
арміями у вигляді несамовитих воїнів-божеств Херук та Кродхішвар. Якшо у тибетців
є зображення настільки несамовитих божеств, то їх армія --- це точно ЗСУ.

\newpage
Фундамент та запорука засвоєння основ етики --- це освіта. Те, що на заході
називється Liberal Arts, в Гімалаях називається університетом Наланди. Це
коплекс предметів які є обов'язковими для націєтворення, що включають
вивчення природничих наук, соціальних наук та творчих наук: музика, філологія, філософія,
математика, образотворче мистецтво. Тибетська культура розповсюджується далеко за межі
Тибетського Автономного Округу та містить міцні корені в Бутані та провінції Сичуань (Китай).
Бутан --- це приклад такого просвітленого царства, побудованого згідно тибетським поняттям
націєтворення, яке доєдналось до парламенту нації ООН в 1971 році та отримало індійський
протекторат та захист від китайської експансії в тибетський регіон.
\\
\\
Користуючись аграрними аналогіями, якшо націєтворців, які культивують народи
у нації, ми бачимо як садовників, що підстригають дерева та кущі, в тибетскій
культурі наше мислення як істоти так і називається --- плодом, що повинен
дозріти. Кожен тибетець зрощує плід свого мислення на благо усіх живих істот.
Завдання кожної істоти це розвинення свого мислення до максимального рівня,
яке здатне вмістити в себе серця всіх істот, заповнити своє мислення їх щастям.
Традиційно, викладанням освітніх предметів займалися в монастирях лами, у той
час, коли сучасний набір профеcій необхідний для націєтворення розширюється до
інтелігенції як то лікарі, викладачі, політики, журналісти, науковці, інженери,
художники, артисти. Солдати, в умовах геноциду, кожен усвідомленний громадянин
своєї нації повинен бути несамовитим божеством, так само, як кожен представник
тибетського народу є духовним практиком. Єдиний шлях до виживання нації яку
систематично намагаються знищити, --- це вибрати шлях діамантового спокою та твердості,
джерелом яких повинно стати повне усвідомлення процесів та своєї ролі в цій війні.
\\
\\
Кожен, хто може, повинен оновити систему своїх цінностей та спрямувати її на
виживання своєї частини української культури. Передбачається, що кожен українець,
що читає цей текст, вже знає сам для себе, що таке українська культура, інакше
читач буде зневажливо ставитись до цього тексту та не зможе його розкодувати.
\\
\\
Задача кожного майстра --- передати свої знання достойним учням. Задача кожного
автора --- написати свій магнум опус. Кожного артиста --- знайти свою сцену.
Кожного дослідника --- відкрити спокій, результат та сатисфакцію. Як народ, ми
маємо право жити, проте наше завдання зберегти націю як політично-правову
культуру орнаментовану творчістю українців.
\\
\\
Система функціонування української держави та українського націоналізму
як освітньої державотворчої системи за межами часу та простору повинна
бути випробувана нашими визвольними змаганнями аби довести своє право
на існування в середовищі максимально агресивних викликів у вигляді
божевільних націй. Вона повинна містити абсолютну етику, яка дозволить
ставати на нашу сторону союзникам з моральними нормами максимального
спектру вимог, наш моральний стандарт повинен бути найвищим. Ми повинні
демонструвати смиренність там, де несамовитість очікується допустимою.
Ми повинні демонструвати милосердя там, де безпека нашого виживання важливіша.
Але ми повинні жити. Українці повинні жити.
\\
\\
\textsc{\footnotesize 21 червня 2022, Вишневе}

\normalsize
