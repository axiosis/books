\section{Прокльони}

Мене питають, як правильно проклинати, шо таке прокльони і чому вони
не працюють одразу. Розповідаю.
\\
\\
Проклинати треба століттями і не забувати про це. Встав, помився, помедитував
30 хв і почав проклинати, закінчив практику, посвятив заслуги. І так 100 років.
Тоді жодного малороса не залишиться, а не буде малоросів, не буде і росів.
\\
\\
Механіка прокльону в тому шо ви себе дистанціюєте від об'єкта прокльону і
намагаєтесь розібратися в собі: чому ви ненавидите свій об'єкт. І поступово
повинні вичавлювати з себе якості, якими він наділений, і які є у вас.
\\
\\
Набільший прокльон --- це чітке усвідомлення ситуації і бездоганна поведінка
з погляду внутрішнього спокою. Джерело бездоганного прокльону --- бездоганна
емпатія. Емпатія --- корінь етики. Етика --- корінь освіти. Освіта --- мета
держави. Держава --- основа нації.
\\
\\
Я, наприклад, кожен день проклинаю всіх хто знущається над тваринами з 5 років,
коли побачив вперше, як старші пацани копнули паршивого кота на 2 метри. В той
день я став буддистом і проклинаю кожного твариноневисника на регулярній основі.
Потім я почав проклинати тих, хто не вміє співчувати загалом. Потім неосвічених
людей. Тепер я гуру прокльонів. Коли ти мислиш бездоганно --- це і є
найпотужніжий прокльон який може бути у всесвіті.
\\
\\
Мислення --- це і є ненависть.\\
Ненависть до невігластва --- це і є мислення.
\\
\\
Гомотопічний транспорт: enc . dec = id, dec . enc = id. Якшо ти не
ненавидиш невігластво --- то ти не мислиш. Це на другому курсі PhD
студій по тибетології в університетах вивчають, до речі стосовно
наукового методу. А саме йдеться про літературний спадок, терма
Карма Лінгпи <<Тибетська Книга Мертвих>> 14 століття (найкращий
переклад від тата Уми Турман), де йдеться мова про мирних та
несамовитих божеств. Так от: джерело несамовитих божеств --- це ваш
мозок або ваші розумові здібності. Але я впевнений, що ви ж тут
і так всі освічені в мене підписники. Формально --- це область
літератури та мистецтва, які вивчають емпатію та етику. Тому
тибетологія та релігії --- це, по суті, літературні предмети.
Як і континентальна філософія.
\\
\\
P.S. для так званих адептів. Технічно не існує магічної психічної
енергії яка, передається від одноєї істоти до іншої: не дивлячись
на те, шо всі істоти перебувають в одному просторі, їхні карми
не перетинаються. Поняття карми, як і цей факт, були відомі
задовго до Гуатами Шак'ямуні. Тому якшо ви думаєте шо
прокльон --- це магічний фаєрбол, то ваш рівень розвитку
знаходиться на рівні РПЦ. То, що ми молимося, наприклад, Будді
Медицини Орджен Менла не означає, що ми здійснюємо потаємну
магію лікування і стираємо чиюсь негативну карму -- це фізично
неможливо, інакше Будда би це давно зробив хлопком однієї
долоні (по своїй сраці), це лише означає шо ми/ви на 1/108000000000 (ви
як одна зі 108 мільярдів людей шо існували на цій планеті) покращуєте
простір, де перебувють живі істоти і цим самим створюєте мізерну (потаємну)
умову для їхнього щастя (причину для себе може створити тільки сама
істота, або інакше кажучи, <<спасти себе сама>>). Як би так зробили
всі 108 мільярдів душ, то нірвана наступила би моментально.
\\
\\
\textsc{\footnotesize 3 серпня 2022, Кам'янець-Подільський}

\normalsize
