\section{Мета Держави}

\subsection{Неправильне розуміння освіти}

Цікаво, що сутність, яка заставляє експата купувати згущьонку в каліфорнійській бєрьозці,
сутність що заставляє купувати супутникову антену в 3-му тисячолітті, щоб дивитися
російські канали, і сутність яка заставляє істоту насолоджуватися вбивствами своїх
сусідів -- це всьо одне. І ім'я цій сутності російська душа. Вона наче трирічна
дитина, яка думає, що командує процесом і режисує дійсність, однак може засунути
в носа лампочку, або кинути кип'яток чаю в голову ні з того ні з сього, тому шо
воно просто ідіот. Ті, хто народилися в СССР, знають, шо коли ви в СССР, вас можуть
просто ні з того ні з сього просто вбити, чи кинути вас в чимось, чи виїбати
просто на вулиці. Цікаво, що сучасна російська молодь пишається цим еволюційним здобутком.
\\
\\
Росіяни --- це перший народ, якому дали гроші та технології, а вони все одно хочуть
вбивати людей і насолоджуватися цим. Їм не потрібна освіта, вони навіть не
розуміють шо це таке. Вони думають, що освіта --- це розрахунок балістичної траекторії
або когомології Хохшильда рахувати, паралельно з піною в роті на ютубі демонструвати
свою скалічену психіку.
\\
\\
Відсутність освіти народжує концепції євразійської небезпеки. Треба заборонити
векторні устремління в майбутнє як в комунізмі та утопіях типу Арестовича.
Коли ви занадто дивитесь вперед, ціна помилки --- мільйони життів. Тому у європейців
така консервативна зовнішня політика. Ну, крім німців, ті просто мразі.
Ті, хто планували століттями вперед, поклали мільйони життів на вівтар своєї ідеї.
І завжди це не спрацьовувало. Люди завжди вмирають ні за цапову душу на катовищі
політичних диктатур. Там немає ніякої філософії, філософія --- це не про те як
ефективно вбивати та хоронити.
Так само не можна шукати політичного вектору в минулому, що неперервно роблять
фашистські режими, відновлювачі імперій та собіратєлі земель в постіній ностальгії
за імпрерським минулим.
\\
\\
Єдиний спосіб етичної політики всесвітньої --- це невідкладна допомога людям тут і
зараз. Ми, українці, наприклад ціною свого життя годуємо частину планети нашим
зерном. Це достойна роль на планеті Земля, за яку не стидно виконувати. Ми не
будуємо фортець в ефемерному майбутньому та не шукаємо правди у варварських
часах своєї східноєвропейської імперії. Ми годуємо людей тут і зараз і захищаємо
Європу від вбивць та людожерів --- росіян. Немає ніякої іншої ідеї, ця ідея вже
достойна легенд у віках про те як маленька нація хліборобів захищає і годує пів
світу від вбивць і психічно хворих створінь які навалою лізуть на цивілізацію
століттями. Україна завжди існувала в романтизмі та бароко. Я консерватор цих
часів. Іншого часу для мене в українському контексті не існує, так само як
культурно і локально я живу в 70-х роках минулого століття. Арестович як психолог
повинен зрозуміти шо рана українців від заподіяної шкоди росіян глибша, ніж він думає.
Можливо як би він отримав освіту, зокрема в області української літератури,
він би зрозумів, що ці народи асимілюватися не можуть поки живе Гнів Українців
та дифузійні соціальні моделі.

\subsection{Освіта як мета держави}

Як сказав один сучасний український філософ на ютуб з династії філософів:
мета держави --- це освіта. Але отримати освіту в СССР було в повному обсязі
неможливо, зрозуміти що таке освіта мені допоміг університет Наланди та
власна освіта батьків-викладачів.
\\
\\
Якшо коротко, то корінь освіти --- це етика, здатність до емпатії, вміння та бажання
еволюціонувати задля гарного життя всього людства, всьо разом називається
коротко --- бути людиною.
\\
\\
Тобто головне --- це мотивація, навіщо взагалі покидати безпечний острів невігластва
та неандерталізму: покидати для гармонійного розвитку людства. Якшо мотивація інша,
то освітою це можна назвати лише умовно. Якшо кидати все і спасати державу, то окрім
операційних міністерств, стратегічним міністерством є міністерство освіти і науки.
Я виділив би декілька головних інститутів, які би я назвав <<головою>>, <<душею>> та <<творчістю>> нації.

\subsection{Інститут формальної математики та філософії}

Отже, перший інститут академії наук МОН --- це інститут
математики.\footnote{ \url{https://groupoid.space/institute/} --- п'ять кафедр інституту}, який повинен містити кафедру формальної філософії,
яка показує, що реальна, дійсна філософія можлива лише в рамках певної мови програмування.
Якшо ваша думка не може бути закодована або є білібердою, то ви не філософ.

\subsection{Інститут формальної літератури}

Якшо вона є білібердою, то це предмет розгляду інституту літератури, яка, як відомо, є душею
нації. Якшо ця біліберда знаходить відклик в серцях українців, то це предмет емпатії, якшо
вона робить їх кращими, то це об'єкт розгляду інституту формальної літератури. У той час
як інститут формальної математики формалізує думку, інститут формальної літератури
формалізує серце української нації засобами формалізації природніх людських мов та
машинної лінгвістики.

\subsection{Інститут формального мистецтва}

Третій найголовніший інститут --- це інститут музики, кіно та мистецства. Теж формальний,
а як ви думали, весь сучасний контент створюється формальними засобами та ретельно
моделюється. Часи спонтанної творчості пройшли. Настали часи корпоративної та спільної творчості.
\\
\\
Людина, яка не має освіти, у цих інститутах завжди буде ненавидіти людину, яка у цих
інститутах освіту отримувала. Тому перше, що потрібно вичавити радянській людині з
себе --- це відчуття класової ненависті до багатих та розумних.
\\
\\
Більшість українців зразу цю енергію кинули в роботу, не завжди в еволюцію над собою,
але в піклування про свої родини та безпеку. В що інвестували росіяни та кому вони
віддали управління своєю свідомістю ми бачимо в онлайн. Суцільне порушення дхарма
ліцензії. Після того, як впав совок та десятки мільйонів зазомбованих рабів-дебілів,
які лікуються зєльонкою, гірчичниками, заперечують віруси, ще недавно заряджали воду
по телевізору та ходили на стадіони на кашпіровського, вивільнилися для нових керманичів їхньої психіки.
\\
\\
В буддизмі кажуть, що аби карма почала негативно визрівати, вона повинна бути
належним чином запечатана, ви повинні:
1) мати мотивацію (хотіти вбити українців),
2) здійснити фінансування або виконання (сплатити податки), та
3) отримати насолоду від цього (хоча б на мить посміхнутися та зрадіти).
Ваші це думки чи не ваші --- це не важливо для вашої свідомості, так само це
і не важливо для вашого тіла. Ви опиняєтеся автоматично у середовищі, яке
ви самі для себе викохали. На шляху до абсолютного дна у вас буде два
попередження: втрата технологій (тварини), та голод (духи). Ну а після вже ади.
Як ви любите. Найнижчий ад чекає тих, хто образив за своє життя хоч одного
справжнього гуру. Релігійні тексти, зокрема тибетська філософська поезія,
очевидно є об'єктом розгляду інституту формальної літератури.
\\
\\
У Мадг'яміці є пару теорем, зокрема теорема про пустотність всіх феноменів,
які формалізуються в сучасній гомотопічній теорії типів, як це було показано
в додатку до дисертації Фавонії. Так, що деякі семантичні моделі транспортовані
в інститут математики навіть.
\\
\\
Російська література --- це 100-серійний серіал про соціопатів вбивць
типу MINDHUNTER на лупі про божевільних росіян, які або шось вкрали в
голандців, або в колабі з кайзером вкотре намагаються завоювати світ.
Або, коли не вийшло, ниють і аналізують свою скалічену психіку шляхом
генерації безкінечної низки блювотиннячих епосів розміру як простір
їхнього невігластва. Глибокий аналіз російської літератури може
призвести в результаті тільки до виправдяння вчинків вбивць. Ніякої
високої моралі, яка би гарантувала достатній рівень емпатії для початкового
саморозвитку особистості, російська література не пропонує. Навпаки, вона
розширює простір можливостей до законів фізики. А вбивати і гвалтувати
дітей, виправдовуючи це, або не розуміючи своєї ролі у цьому дійстві, --- це
не найогидніше шо може придумати їхня психіка. Впевнений: росіяни придумають
через 100 років нові методи залякування. Я 40 живу з неперервним аналізом
російської душі і мушу сказати, що це --- наймерзенніше, що можна аналізувати.
Завжди є якась риса яка об'єднує найвіддаленіших маргіналів, в основному ---
це дві безвимірні речі: необмежена прихована ненависть до українців та
бескінечне невігластво. Бескінечне невігластво --- це плід промивання
мізків ідеею про багатостраждальний СССР, про мінімалізм в комунізмі
і інший неоархітектурний брєд. Росіяни експлуатують ідею пригноблених
народів, прикриваючи свою цивілізаційну обмеженість та нездатність до
функціонування як народу. Коли в раба, народженогов СССР, є тільки спогади
про його дружину, народжену в рабстві та його раба-дитину, --- стати повноцінним
членом світової спільноти важко. Необхідно закрити цей рабський гештальт.
Дуже важко експлуатувати образ знедоленого багатостраждального народу коли
твій народ почав три геноциди за одне століття.

\newpage
\subsection{Корінь нації}

Ці три інститути (математики, літератури, музики кіно і мистецтва) є тим,
що формує ядро нації. Нація може існувати в повному сенсі, коли є культура
того, про що тут написано. Якшо цієї культури немає, про націю говорити рано,
треба спочатку закінчити ПТУ, потім 5 років університет, потім 5 років робота,
потім 5 років дисертація, потім визнання та успіх, а потім вже приходьте до
Сохацького вихвалятися!
\\
\\
Зауважте, що Бутан не має міністерства економіки навіть, проте маю націю,
сучасну за мірками недорозвитку загальної цивілізації кульутру, та сидить
за столом ООН. Тому що тибетський народ є володарем та утримувачем етики.
Етика --- це корінь освіти, освіта --- це корінь нації.
\\
\\
Так, що можна жити в лісі, мати філософію, якій позаздрять німці, культуру
рівня світових сцен, та міжнародну політику, але не можна жити в лісі,
кожні 100 років починати загарбницькі війни і казати, що в тебе є нація та культура.
\\
\\
Так само як тибетська культура гартувалась під тиранією китаю, українська
культура гартувалась під тиранією росії. Тому навряд чи комусь вдасться
знищити українську націю коли небудь. Карму нашого болю та ненависті
ми можемо розділити тільки з народами, які переживали геноциди.
\\
\\
\textsc{\footnotesize 16 липня 2022, Кам'янець-Подільський}

\normalsize
